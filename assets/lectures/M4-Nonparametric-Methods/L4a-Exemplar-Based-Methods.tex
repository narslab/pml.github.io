%\documentclass[smaller, handout]{beamer}
\def\bmode{0} % Mode 0 for presentation, mode 1 for a handout with notes, mode 2 fo% r handout without notes
\if 0\bmode
\documentclass[smaller]{beamer}
\else \if 1\bmode
\immediate\write18{pdflatex -jobname=\jobname-Handout-Notes\space\jobname}
\documentclass[smaller,handout]{beamer}
\usepackage{handoutWithNotes}
\pgfpagesuselayout{2 on 1 with notes}[letterpaper, landscape, border shrink=4mm]
\else \if 2\bmode
\immediate\write18{pdflatex -jobname=\jobname-Handout\space\jobname}
\documentclass[smaller,handout]{beamer}
\fi
\fi
\fi

%\setbeamertemplate{section in head/foot}{}
%\setbeamertemplate{section in head/foot shaded}{\textcolor{white}{\insertsectionhead}}
%%%%%%%%%%%%%%%%%%%%%%%%%%%%%%%%%%%%%%%%%%%%%%%%%%%%%%%%%%%%%%%%%%%%%%%%%%%%%%%%%%%%%%%%%%%%%
\newcommand{\coursetitle}{CEE 697M: Probabilistic Machine Learning}
\newcommand{\longlecturetitle}{M4 Nonparametric Methods:\\ L4a: Exemplar-based methods}
\newcommand{\shortlecturetitle}{L4a: Exemplar-Based Methods}
\newcommand{\instructor}{Jimi Oke}
\newcommand{\lecturedate}{Wed, Apr 19, 2023}
%%%%%%%%%%%%%%%%%%%%%%%%%%%%%%%%%%%%%%%%%%%%%%%%%%%%%%%%%%%%%%%%%%%%%%%%%%%%%%%%%%%%%%%%%%%%%

 
 
% \usepackage[T1]{fontenc} 
% \usepackage{lmodern} 
%\usepackage{etex}
 %\newcommand{\num}{6{} }

% \usetheme[
%   outer/progressbar=foot,
%   outer/numbering=fraction,
%   block=fill,
%   inner/subsectionpage=progressbar
% ]{metropolis}
\usetheme{Madrid}
\useoutertheme[subsection=false]{miniframes} % Alternatively: miniframes, infolines, split
\useinnertheme{circles}
% %\useoutertheme{Frankfurt}
% \usecolortheme{beaver}
% %\useoutertheme{crane}
% %\useoutertheme{metropolis}
\usepackage[backend=biber,style=authoryear,maxcitenames=2,maxbibnames=99,safeinputenc,url=false, eprint=false]{biblatex}
%\addbibresource{bib/references.bib}
% \AtEveryCitekey{\iffootnote{{\tiny}\tiny}{\tiny}}

% %\usepackage{pgfpages}
% %\setbeameroption{hide notes} % Only slides
% %\setbeameroption{show only notes} % Only notes
% %\setbeameroption{hide notes} % Only notes
% %\setbeameroption{show notes on second screen=right} % Both

% % \usepackage[sfdefault]{Fira Sans}

% % \setsansfont[BoldFont={Fira Sans}]{Fira Sans Light}
% % \setmonofont{Fira Mono}

% %\usepackage{fira}
% %\setsansfont{Fira}
% %\setmonofont{Fira Mono}
% % To give a presentation with the Skim reader (http://skim-app.sourceforge.net) on OSX so
% % that you see the notes on your laptop and the slides on the projector, do the following:
% % 
% % 1. Generate just the presentation (hide notes) and save to slides.pdf
% % 2. Generate onlt the notes (show only nodes) and save to notes.pdf
% % 3. With Skim open both slides.pdf and notes.pdf
% % 4. Click on slides.pdf to bring it to front.
% % 5. In Skim, under "View -> Presentation Option -> Synhcronized Noted Document"
% %    select notes.pdf.
% % 6. Now as you move around in slides.pdf the notes.pdf file will follow you.
% % 7. Arrange windows so that notes.pdf is in full screen mode on your laptop
% %    and slides.pdf is in presentation mode on the projector.

% % Give a slight yellow tint to the notes page
% \setbeamertemplate{note page}{\pagecolor{yellow!5}\insertnote}\usepackage{palatino}

% %\usetheme{metropolis}
% %\usecolortheme{beaver}
 \usepackage{tipa}
% \usepackage{enumerate}
\definecolor{darkcandyapplered}{HTML}{A40000}
\definecolor{lightcandyapplered}{HTML}{e74c3c}

% %\setbeamercolor{title}{fg=darkcandyapplered}

% \definecolor{UBCblue}{rgb}{0.04706, 0.13725, 0.26667} % UBC Blue (primary)
% \definecolor{UBCgrey}{rgb}{0.3686, 0.5255, 0.6235} % UBC Grey (secondary)

% % \setbeamercolor{palette primary}{bg=darkcandyapplered,fg=white}
% % \setbeamercolor{palette secondary}{bg=darkcandyapplered,fg=white}
% % \setbeamercolor{palette tertiary}{bg=darkcandyapplered,fg=white}
% % \setbeamercolor{palette quaternary}{bg=darkcandyapplered,fg=white}
% % \setbeamercolor{structure}{fg=darkcandyapplered} % itemize, enumerate, etc
% % \setbeamercolor{section in toc}{fg=darkcandyapplered} % TOC sections
% % \setbeamercolor{frametitle}{fg=darkcandyapplered,bg=white} % TOC sections
% % \setbeamercolor{title in head/foot}{bg=white,fg=white} % TOC sections
% % \setbeamercolor{button}{fg=darkcandyapplered} % TOC sections

% % % Override palette coloring with secondary
% % \setbeamercolor{subsection in head/foot}{bg=lightcandyapplered,fg=white}

%\usecolortheme{crane}
% \makeatletter
% \setbeamertemplate{headline}{%
%   \begin{beamercolorbox}[colsep=1.5pt]{upper separation line head}
%   \end{beamercolorbox}
%   \begin{beamercolorbox}{section in head/foot}
%     \vskip1pt\insertsectionnavigationhorizontal{\paperwidth}{}{}\vskip1pt
%   \end{beamercolorbox}%
%   \ifbeamer@theme@subsection%
%     \begin{beamercolorbox}[colsep=1.5pt]{middle separation line head}
%     \end{beamercolorbox}
%     \begin{beamercolorbox}[ht=2.5ex,dp=1.125ex,%
%       leftskip=.3cm,rightskip=.3cm plus1fil]{subsection in head/foot}
%       \usebeamerfont{subsection in head/foot}\insertsubsectionhead
%     \end{beamercolorbox}%
%   \fi%
%   \begin{beamercolorbox}[colsep=1.5pt]{lower separation line head}
%   \end{beamercolorbox}
% }
% \makeatother

% Reduce size of frame box
\setbeamertemplate{frametitle}{%
    \nointerlineskip%
    \begin{beamercolorbox}[wd=\paperwidth,ht=2.0ex,dp=0.6ex]{frametitle}
        \hspace*{1ex}\insertframetitle%
    \end{beamercolorbox}%
}


%\setbeamercolor{frametitle}{bg=darkcandyapplered!80!black!90!white}
%\setbeamertemplate{frametitle}{\bf\insertframetitle}

%\setbeamercolor{footnote mark}{fg=darkcandyapplered}
%\setbeamercolor{footnote}{fg=darkcandyapplered!70}
%\Raggedbottom
%\setbeamerfont{page number in head/foot}{size=\tiny}
%\usepackage[tracking]{microtype}


% %\usepackage[sc,osf]{mathpazo}   % With old-style figures and real smallcaps.
% %\linespread{1.025}              % Palatino leads a little more leading

% % Euler for math and numbers
% %\usepackage[euler-digits,small]{eulervm}
% %\AtBeginDocument{\renewcommand{\hbar}{\hslash}}
\usepackage{graphicx}
\usepackage{multirow}
\usepackage{booktabs}
\usepackage{graphbox}
\usepackage{animate}
\usepackage{media9}
\usepackage{adjustbox}

% %\mode<presentation> { \setbeamercovered{transparent} }

\setbeamertemplate{navigation symbols}{}
\makeatletter
\def\beamerorig@set@color{%
  \pdfliteral{\current@color}%
  \aftergroup\reset@color
}
\def\beamerorig@reset@color{\pdfliteral{\current@color}}
\makeatother


% %=== GRAPHICS PATH ===========
\graphicspath{{./m5-images/}}
% % Marginpar width
% %Marginpar width
% %\setlength{\marginparsep}{.02in}


% %% Captions
% % \usepackage{caption}
% % \captionsetup{
% %   labelsep=quad,
% %   justification=raggedright,
% %   labelfont=sc
% % }

% \setbeamerfont{caption}{size=\footnotesize}
% \setbeamercolor{caption name}{fg=darkcandyapplered}

% %AMS-TeX packages

\usepackage{amssymb}
\usepackage{amsmath}
\usepackage{amsthm}
\usepackage{mathtools} 
\usepackage{bm}
\DeclareMathOperator*{\argmax}{arg\,max}
\DeclareMathOperator*{\argmin}{arg\,min}
% \usepackage{color}

% %https://tex.stackexchange.com/a/31370/2269
\usepackage{cancel}
\renewcommand{\CancelColor}{\color{red}} %change cancel color to red
\makeatletter
\let\my@cancelto\cancelto %copy over the original cancelto command
\newcommand<>{\cancelto}[2]{\alt#3{\my@cancelto{#1}{#2}}{\mathrlap{#2}\phantom{\my@cancelto{#1}{#2}}}}
% redefine the cancelto command, using \phantom to assure that the
% result doesn't wiggle up and down with and without the arrow
\makeatother


% % \usepackage{comment}
\usepackage{enumerate}
\usepackage{hyperref}
% \usepackage{minitoc,array}
% \definecolor{slblue}{rgb}{0,.3,.62}
\hypersetup{
    colorlinks,%
    citecolor=blue,%
    filecolor=blue,%
    linkcolor=blue,
    urlcolor=blue
}

% \usepackage{epstopdf}
% \epstopdfDeclareGraphicsRule{.gif}{png}{.png}{convert gif:#1 png:\OutputFile}
% \AppendGraphicsExtensions{.gif}

% %\usepackage{listings}

% %%% TIKZ
\usepackage{forest}
\usepackage{tikz}
\usepackage{tikz-3dplot}
\usepackage{pgfplots}
\usepackage{pgfplotstable}
% \usepackage{pgfgantt}
\usepackage{neuralnetwork}

\usetikzlibrary{fit,arrows,arrows.meta,shapes,positioning,shapes.geometric}
\usetikzlibrary{decorations.markings}
\usetikzlibrary{shadows,automata}
\usetikzlibrary{patterns}
\usetikzlibrary{trees,mindmap,backgrounds}
%\usetikzlibrary{circuits.ee.IEC}
\usetikzlibrary{decorations.text}
% % For Sagnac Picture
% \usetikzlibrary{%
%     decorations.pathreplacing,%
%     decorations.pathmorphing%
% }
% \tikzset{no shadows/.style={general shadow/.style=}}
% %
% %\usepackage{paralist}

\tikzset{
  font=\Large\sffamily\bfseries,
  red arrow/.style={
    midway,red,sloped,fill, minimum height=3cm, single arrow, single arrow head extend=.5cm, single arrow head indent=.25cm,xscale=0.3,yscale=0.15,
    allow upside down
  },
  black arrow/.style 2 args={-stealth, shorten >=#1, shorten <=#2},
  black arrow/.default={1mm}{1mm},
  tree box/.style={draw, rounded corners, inner sep=1em},
  node box/.style={white, draw=black, text=black, rectangle, rounded corners},
}

% %%% FORMAT PYTHON CODE
% %\usepackage{listings}
% % Default fixed font does not support bold face
% \DeclareFixedFont{\ttb}{T1}{txtt}{bx}{n}{8} % for bold
% \DeclareFixedFont{\ttm}{T1}{txtt}{m}{n}{8}  % for normal

% % Custom colors
% \definecolor{deepblue}{rgb}{0,0,0.5}
% \definecolor{deepred}{rgb}{0.6,0,0}
% \definecolor{deepgreen}{rgb}{0,0.5,0}

% %\usepackage{animate}

% % Python style for highlighting
% % \newcommand\pythonstyle{\lstset{
% % language=Python,
% % basicstyle=\footnotesize\ttm,
% % otherkeywords={self},             % Add keywords here
% % keywordstyle=\footnotesize\ttb\color{deepblue},
% % emph={MyClass,__init__},          % Custom highlighting
% % emphstyle=\footnotesize\ttb\color{deepred},    % Custom highlighting style
% % stringstyle=\color{deepgreen},
% % frame=tb,                         % Any extra options here
%     % showstringspaces=false            % 
% % }}

% % % Python environment
% % \lstnewenvironment{python}[1][]
% % {
% % \pythonstyle
% % \lstset{#1}
% % }
% % {}

% % % Python for external files
% % \newcommand\pythonexternal[2][]{{
% % \pythonstyle
% % \lstinputlisting[#1]{#2}}}

% % Python for inline
% % 
% % \newcommand\pythoninline[1]{{\pythonstyle\lstinline!#1!}}

% %\usepackage{algorithm2e}

\newcommand{\eps}{\epsilon}
\newcommand{\bX}{\mb X}
\newcommand{\by}{\mb y}
\newcommand{\bbe}{\bm\beta}
\newcommand{\beps}{\bm\epsilon}
\newcommand{\bY}{\mb Y}

\newcommand{\osn}{\oldstylenums}
\newcommand{\dg}{^{\circ}}
\newcommand{\lt}{\left}
\newcommand{\rt}{\right}
\newcommand{\pt}{\phantom}
\newcommand{\tf}{\therefore}
\newcommand{\?}{\stackrel{?}{=}}
\newcommand{\fr}{\frac}
\newcommand{\dfr}{\dfrac}
\newcommand{\ul}{\underline}
\newcommand{\tn}{\tabularnewline}
\newcommand{\nl}{\newline}
\newcommand\relph[1]{\mathrel{\phantom{#1}}}
\newcommand{\cm}{\checkmark}
\newcommand{\ol}{\overline}
\newcommand{\rd}{\color{red}}
\newcommand{\bl}{\color{blue}}
\newcommand{\pl}{\color{purple}}
\newcommand{\og}{\color{orange!90!black}}
\newcommand{\gr}{\color{green!40!black}}
\newcommand{\lbl}{\color{CornflowerBlue}}
\newcommand{\dca}{\color{darkcandyapplered}}
\newcommand{\nin}{\noindent}
\newcommand*\circled[1]{\tikz[baseline=(char.base)]{
            \node[shape=circle,draw,thick,inner sep=1pt] (char) {\small #1};}}

\newcommand{\bc}{\begin{compactenum}[\quad--]}
\newcommand{\ec}{\end{compactenum}}

\newcommand{\p}{\partial}
\newcommand{\pd}[2]{\frac{\partial{#1}}{\partial{#2}}}
\newcommand{\dpd}[2]{\dfrac{\partial{#1}}{\partial{#2}}}
\newcommand{\pdd}[2]{\frac{\partial^2{#1}}{\partial{#2}^2}}
\newcommand{\pde}[3]{\frac{\partial^2{#1}}{\partial{#2}\partial{#3}}}
\newcommand{\nmfr}[3]{\Phi\left(\frac{{#1} - {#2}}{#3}\right)}
\newcommand{\Err}{\text{Err}}
\newcommand{\err}{\text{err}}

\DeclarePairedDelimiter\ceil{\lceil}{\rceil}
\DeclarePairedDelimiter\floor{\lfloor}{\rfloor}

%%%% GREEK LETTER SHORTCUTS %%%%%
\newcommand{\la}{\lambda}
\renewcommand{\th}{\theta}
\newcommand{\al}{\alpha}
\newcommand{\G}{\Gamma}
\newcommand{\si}{\sigma}
\newcommand{\Si}{\Sigma}


\pgfmathdeclarefunction{poiss}{1}{%
  \pgfmathparse{(#1^x)*exp(-#1)/(x!)}%
  }

\pgfmathdeclarefunction{gauss}{2}{%
  \pgfmathparse{1/(#2*sqrt(2*pi))*exp(-((x-#1)^2)/(2*#2^2))}%
}

\pgfmathdeclarefunction{expo}{2}{%
  \pgfmathparse{#1*exp(-#1*#2)}%
}

\pgfmathdeclarefunction{expocdf}{2}{%
  \pgfmathparse{1 -exp(-#1*#2)}%
}

\newcommand{\mb}{\mathbb}
\newcommand{\mc}{\mathcal}
\newcommand{\tr}{^{\top}}
\newcommand{\empt}[2]{$#1^{( #2 )}$}
\newcommand{\pe}{\pause}
% \usepackage{pst-plot}

% \usepackage{pstricks-add}
% \usepackage{auto-pst-pdf}   

% \psset{unit = 3}

% \def\target(#1,#2){%
%  {\psset{fillstyle = solid}
%   \rput(#1,#2){%
%     \pscircle[fillcolor = white](0.7,0.7){0.7}
%     \pscircle[fillcolor = blue!60](0.7,0.7){0.5}
%     \pscircle[fillcolor = white](0.7,0.7){0.3}
%     \pscircle[fillcolor = red!80](0.7,0.7){0.1}}}}
% \def\dots[#1](#2,#3){%
%     \psRandom[
%       dotsize = 2pt,
%       randomPoints = 25
%     ](!#2 #1 0.04 sub sub #3 #1 0.04 sub sub)%
%      (!#2 #1 0.04 sub add #3 #1 0.04 sub add)%
%      {\pscircle[linestyle = none](#2,#3){#1}}}


%%%%%%%%%%%%%%%%%%%%%%%%%%%%%%%%%%%%%%%%%%%%%%%%%%%
%%%%%%%%%%%%%%%%%%%%%%%%%%%%%%%%%%%%%%%%%%%%%%%%%%%
\title[\shortlecturetitle]{ {\normalsize \coursetitle}
  \\ \longlecturetitle}
\date[\lecturedate]{\footnotesize \lecturedate}
\author{{\bf \instructor}}
\institute[UMass Amherst]{
%\titlegraphic{\hfill
  \begin{tikzpicture}[baseline=(current bounding box.center)]
    \node[anchor=base] at (-7,0) (its) {\includegraphics[scale=.3]{UMassEngineering_vert}} ;
  \end{tikzpicture}
  % \hfill\includegraphics[height=1.5cm]{logo}
}

%https://tex.stackexchange.com/questions/55806/mindmap-tikzpicture-in-beamer-reveal-step-by-step
  \tikzset{
    invisible/.style={opacity=0},
    visible on/.style={alt={#1{}{invisible}}},
    alt/.code args={<#1>#2#3}{%
      \alt<#1>{\pgfkeysalso{#2}}{\pgfkeysalso{#3}} % \pgfkeysalso doesn't change the path
    },
  }


% https://tex.stackexchange.com/questions/446468/labels-with-arrows-for-an-equation
% https://tex.stackexchange.com/a/402466/121799
\newcommand{\tikzmark}[3][]{
\ifmmode
\tikz[remember picture,baseline=(#2.base)] \node [inner sep=0pt,#1](#2) {$#3$};
\else
\tikz[remember picture,baseline=(#2.base)] \node [inner sep=0pt,#1](#2) {#3};
\fi
}

% \lstset{language=matlab,
%                 basicstyle=\scriptsize\ttfamily,
%                 keywordstyle=\color{blue}\ttfamily,
%                 stringstyle=\color{blue}\ttfamily,
%                 commentstyle=\color{gray}\ttfamily,
%                 morecomment=[l][\color{gray}]{\#}
%               }


%%% Local Variables:
%%% mode: latex
%%% TeX-master: t
%%% End:

              
\begin{document}
\maketitle
\begin{frame}
  \frametitle{Outline}
  \tableofcontents
\end{frame}



 
  
 
\section{Introduction}

\begin{frame}
  \frametitle{Nonparametric modeling}
  \pe
  Parametric models seek to estimate $p(\bm y|\bm\th)$ (unconditional case) or $p(\bm y|\bm x,\bm\th)$ (conditional case).

  \pe

  \begin{itemize}
  \item $\bm\th$ is a fixed-dimensinoal vector of \textbf{parameters}\pe
  \item Estimation is performed using a dataset $\mc D = \{(\bm x_{n},\bm y_{n}) : n=1:N\}$\pe
  \item There is an assumed functional form: $\bm y \sim f_{\bm\th}(\bm x)$
  \end{itemize}
\pe
\bigskip

\textbf{Nonparametric models} are defined based on similarity between a test input $\bm x$ at each training input $\bm x_{n}$: $d(\bm x, \bm x_{n})$
\pe
\begin{itemize}
\item No assumption of functional form on model parameters\pe
\item Effective number of parameters can grow with size of dataset $|\mc D|$\pe
\item Known as \textbf{exemplar-based models} (as training samples are used to make each future prediction)\pe
\item Other names: instance-based learning, memory-based learning
\end{itemize}
\end{frame}

\begin{frame}
  \frametitle{Exemplar-based models}
  \pe
  We will consider the following exemplar approaches: \pe
  \begin{itemize}
  \item K-nearest neighbors (KNN)
  \item Kernel density estimation
  \item Kernel [local] regression
  \end{itemize}
\end{frame}

\section{KNN}
\begin{frame}
  \frametitle{K nearest neighbor classifier}
  \pe Basic idea: classify new/test input $\bm x$ by assigning to most probable (majority) label in the neighborhood of
  $\bm x$ (closest examples) from the training set. \pe

  Thus, we estimate:\pe
  \begin{equation}
    p(y = c |\bm x, \mc D) = \fr1K \sum_{n\in N_{K}(\bm x,\mc D)}\mb I(y_{n}= c)
  \end{equation}
  \pe
  \begin{itemize}
  \item  $c$ class label\pe
  \item $K$: number of training samples in neighborhood  \pe
  \item $N_{K}(\bm x,\mc D)$: neighborhood of $\bm x$ (size $K$) based on dataset $\mc D$
  \end{itemize}
\end{frame}


  \begin{frame}
    \frametitle{Illustration of KNN}

    \begin{figure}[h!]
      \centering
      \includegraphics[width=.9\textwidth]{2-14}
      \caption{Illustration of the KNN approach on a training set of 12 observations and the resulting decision boundary. (ESL Fig 2.14)}
      \label{fig:knn}
    \end{figure}
  \end{frame}

  \begin{frame}
    \frametitle{Bias-variance trade-off in KNN}
    \pause
    \begin{figure}[h!]
      \centering
      \includegraphics[width=.9\textwidth]{2-16}
      \caption{Comparing decision boundaries $K = 1$ and $K=100$ for a dataset of 100 observations. Which model has
        lower bias? Which one gives a higher variance?  The Bayes decision boundary is the purple dashed line (ESL Fig
        2.16)}
      \label{fig:knn2}
    \end{figure}
  \end{frame}

    \begin{frame}
    \frametitle{Approximating Bayes decision boundary with KNN}
    \begin{figure}[h!]
      \centering
      \includegraphics[width=.5\textwidth]{2-15}
      \caption{KNN decision boundary with $K=10$ on the same training data set. (ESL Fig 2.15)}
    \end{figure}
  \end{frame}
 
  \begin{frame}
    \frametitle{Training and test error rates for KNN}
    \begin{figure}[h!]
      \centering
      \includegraphics[width=.7\textwidth]{2-17}
      \caption{KNN training error rate (blue, 200 observations) and test error rate (orange, 5000
        observations). Flexibility increases as $K$ decreases. Which $K$ should you choose? (ESL Fig 2.17)}
      \label{fig:knn3}
    \end{figure}
  \end{frame}

\begin{frame}
  \frametitle{KNN  considerations}
  \pe

  \begin{itemize}
  \item    To find the points in the neighborhood $N_{K}$, we need to determine the $K$-closest points to input $\bm x$. \pe
    This is done via a specified distance metric:\pe
 $     d(\bm x,\bm x') \in \mb R^{+}$
    \pe
    (e.g. Euclidean, Mahalanobis) \pe
  \item $K=1$ induces a Voronoi tessellation: partitioning of input sapce such that all points $\bm x \in V(\bm x_{n})$ are closer to $\bm x_{n}$ than to any other point \pe (From a modeling perspective, this is overfitting)
    \pe
    \begin{center}
      \includegraphics[width=.3\textwidth]{VoronoiDiagram}

      {\tiny Source: \url{https://package.elm-lang.org/packages/ianmackenzie/elm-geometry/latest/VoronoiDiagram2d}}
    \end{center}

    \pe
  \item Suffers under high dimensionality\pe
  \item Memory intensive
  \end{itemize}
  
\end{frame}

\begin{frame}
  \frametitle{KNN extension: open set recognition}
  \pe
  KNN is readily applicable to open set recognition (set of classes $\mc C$ not fixed).\pe
  \begin{itemize}
  \item Novelty/out-of-distribution/anomaly detection\pe
        \begin{center}
      \includegraphics[width=.3\textwidth]{anomaly}

      {\tiny Source: \url{https://www.intechopen.com/chapters/74393}}
    \end{center}

  \item Incremental/online/life-long/continual learning: any potentially new label is added to new class $C_{t+1}$; dataset augmented\pe
  \item Few-shot classification (for person re-identification or face verification)
    
  \end{itemize}
\end{frame}

\section{Metric learning}
\begin{frame}
  \frametitle{Distance metrics}
  \pe
  The [semantic] distance between points $\bm x$ and $\bm x'$ is specified by a distance metric $d(\bm x,\bm x')\in\mb R^{+}$. \pe

  \begin{itemize}
  \item Alternately, the similarity $s(\bm x,\bm x')$ can be computed \pe

  \item Distance/similarity required for KNN, unsupervised learning (e.g.\ clustering) among other tasks \pe

  \item  Common  metrics: \pe
  \begin{itemize}
  \item Euclidean distance: \pe
    \begin{equation}
      d_{\bm E}(\bm x,\bm x') = \sqrt{(\bm x-\bm x')\tr(\bm x- \bm x')}
    \end{equation}
    \pe
  \item Mahalanobis distance: \pe
    \begin{equation}
      d_{\bm M}(\bm x,\bm x') = \sqrt{(\bm x-\bm x')\tr\bm M(\bm x- \bm x')}
    \end{equation}
    \pe
    where $\bm M$ is the Mahalanobis distance matrix\pe
  \end{itemize}
    \end{itemize}

\end{frame}

\begin{frame}
  \frametitle{Distance metrics}
 
    
    The process of finding the optimal $\bm  M$ is called \textbf{metric learning}\pe
  \begin{itemize}

    
  \item When $D$ is large, we typically learn an embedding (mapping): $\bm e = f(\bm x)$ and then compute $d_{\bm M}(\bm e,\bm e')$ instead. \pe
  \item When $f$ is a deep neural network, this process is termed \textbf{deep metric learning}
  \end{itemize}
  
\end{frame}

\begin{frame}
  \frametitle{Methods for estimating $\bm M$}
  \pe
  \begin{itemize}
  \item Large margin nearest neighbors (LMNN) \pe

  \item Neighborhood component analysis (NCA) \pe

  \item Latent coincidence analysis (LCA) \pe

  \item Minimization of classification and ranking losses (with mining techniques and proxy methods): \pe
    \begin{itemize}
    \item Pairwise/contrastive loss\pe
    \item Triplet loss\pe
    \item N-pairs loss
    \end{itemize}
  \end{itemize}
\end{frame}
\section{Density kernels}
\begin{frame}
  \frametitle{Density kernel}
  \pause

  A density kernel $\mc K(x)$ %or $\mc K(\bm x, \bm x')$
  is a weighting function that specifies a mapping or transformation of an input
  $x$: $\mc K : \mb R \to \mb R_{+}$. \pe

  Density kernels have two important properties:\pe

  \begin{itemize}
  \item \textbf{Normalization}: \pe
    \begin{equation}
      \int x \mc K (x) dx = 0
    \end{equation}
    \pe

  \item \textbf{Symmetry}: \pe
    \begin{equation}
      \mc K(-x) = \mc K x
    \end{equation}
  \end{itemize}

  \pause


      Kernels have several uses, e.g.\pause
    \begin{itemize}[<+->]
    \item density function estimation
    \item {\gr local regression} (our focus)
    \item smoothing time series
    \end{itemize}  
  \end{frame}

  
\begin{frame}
  \frametitle{Kernel functions}
        \visible<+->{
        \begin{figure}
          \centering
          \includegraphics[width=.4\textwidth]{kernels}
          \caption{Popular kernel functions}
    \end{figure}
    }
    \pause
  \begin{itemize}
  \item Triangular: $\mc K(x) = (1 - |x|)$
  \item Epanechnikov: $\mc K(x) = \fr34(1 - x^{2})$
  \item Triweight: $\mc K(x) = \fr{35}{32} (1-x^{2})^{3}$
  \item Tricube: $\mc K(x) = \fr{70}{81}(1-|x|^{3})^{3}$
    \item Gaussian: $\mc K(x) = \fr1{\sqrt{2\pi}}e^{-\fr12x^{2}}$
  \end{itemize}
\end{frame}

\begin{frame}
  \frametitle{Radial basis function }
  \pe
  The radial basis function (RBF) kernel is a generalization of a density kernel to an input vector $\bm x$:\pe
  \begin{equation}
    \mc K_{h} (\bm x) \propto \mc K_{h}(||\bm x||)
  \end{equation}
  \pe
  where $h$ is the bandwidth parameter: \pe

  The RBF Gaussian kernel is thus given by: \pe
  \begin{equation}
    \mc K_{h}(\bm x) = \fr{1}{h^{D}(2\pi)^{D/2}}\prod_{d=1}^{D}\exp\lt[ -\fr1{2h^{2}}x^{2}_{d}\rt]
  \end{equation}

  \pe

  \begin{block}{Bandwidth parameter}
    \pe
    Specifies the width of the kernel:\pe
      \begin{equation}
        \mc K_{h} := \fr1h \mc K\lt(\fr xh\rt)
  \end{equation}

  \end{block}
\end{frame}
 
\section{Kernel smoothing}

\begin{frame}
  \frametitle{K-nearest neighbor smoother}
  \pause
  In the simple case of the KNN kernel, we use the neighborhood average:\pause
  \begin{equation}
    \hat f(x_0) = \fr1k\sum_{i\in N_K(x_0)} y_ i
  \end{equation}
  \pause
  where $N_K(x_0)$ is the $K$-nearest neighborhood of $x_0$.
  \pause
  \medskip
  
  \begin{minipage}[]{.43\linewidth}
  \begin{itemize}[<+->]
  \item All points in the neighborhood are equally weighted
  \item The resulting $\hat f(x)$ is not smooth
  \item To achieve smoothness, we weight observations by distance to the target point
  \end{itemize}
\end{minipage}\pause
\begin{minipage}[]{.55\linewidth}
  \visible<+->{
    \begin{figure}[h!]
      \centering
      \includegraphics[width=.9\textwidth,trim={0 16cm 11cm 4.5cm},clip]{ESL6-1}
      \caption{KNN equally-weighted kernel}
    \end{figure}
  }
\end{minipage}

\end{frame}

\begin{frame}
  \frametitle{Nadaraya-Watson smoother}\pause
 \textbf{\rd Nadaraya-Watson} kernel-weighted average implements distance-based weighting:\pause
  \begin{equation}
    \label{eq:20}\rd 
   \mb{E}[y|\bm x, \mc D] = \pe \hat f(x_{0}) \pause = \fr{\sum_{i=1}^{n}K_{\la}(x_{0},x_{i})y_{i}}{\sum_{i=1}^{n}K_{\la}(x_{0},x_{i})}
  \end{equation}
  \pause
  where $K_\la$ can be any kernel function. \pause

  \medskip
  If we use the popular  \textbf{\bl Epanechnikov} (quadratic) kernel, then: \pause
  \begin{equation}\bl
    \label{eq:25}
    K_{\la}(x_{o},x_i) \pause  = D\lt( \fr{|x_i - x_{0}|}{\la}\rt)
  \end{equation}
  \pause where
  \begin{equation}
    \label{eq:26}\bl    
    D(t) = \pause
    \begin{cases}
      \fr34\lt(1 -  t^{2}\rt)  & |t| \le 1 \\ \pause
      0 & \text{otherwise}
    \end{cases}
  \end{equation}
  \pause
  In $D$, the half-width (or \textit{bandwidth}) of the neighborhood is given by $\la$. 
\end{frame}

\begin{frame}
  \frametitle{Nadaraya-Watson smoother (Epanechnikov kernel)}
  \pause
  
  \visible<+->{\begin{figure}[h!]
      \centering
      \includegraphics[width=.47\textwidth,trim={10.5cm 16cm 2cm 4.5cm},clip]{ESL6-1}
      \caption{Nadaraya-Watson estimate of $\hat f(x)$ using the Epanechnikov kernel.  Half-width is fixed at $\la =0.2$}
    \end{figure}
  }
  \pause The half-width can be generalized as any function
  $h_\la$ of the target point $x_{0}$:\pause
  \begin{equation}
    \label{eq:27}
    K_{\la}(x_{0},x_i) \pause  = D\lt( \fr{|x_i - x_{0}|}{h_{\la}(x_{0})}\rt)
  \end{equation}
\end{frame}

\section{Local regression}

\begin{frame}
  \frametitle{Kernels as a localization device}
  \pause
  Rather than estimate a regression function $f(X)$ over the entire $\mathbb{R}^{D}$, we can estimate the response at each training point using a weighted average:
  \begin{itemize}[<+->]
  \item only observations close a target point $x_0$ are used in estimating $f$ at that point
  \item the function defining how the points in the neighborhood of $x_0$ are weighted is called a \textbf{kernel}: \pause $K_\la(x_0,x_i)$ where $\la$ controls the width of the neighborhood
  \item this is a localized/memory-based approach
  \item the resulting $\hat f(X)$ is smooth in $\mathbb{R}^D$
  \end{itemize}
  \pause

    Using kernel functions, we can estimate $\hat f (X)$ in two ways:
    \pause
  \begin{enumerate}[<+->]
  \item \textbf{Nonparametric}: define an averaging function (kernel) to estimate $y_{0}$ for each point $x_{0}$
  \item \textbf{Parametric}: estimate a linear model for each point $x_{0}$
  \end{enumerate}
\end{frame}



\begin{frame}
  \frametitle{Local linear regression}
  \pause
  Using a nonparametric average function produces biased estimates at the boundaries.\pause

  \visible<+->{
    \begin{figure}[h!]
      \centering
      \includegraphics[width=.77\textwidth,decodearray={0 .75 0 .75 0 1},
      trim={1cm 14.8cm 0cm 5.7cm},clip]{ESL6-3}
      \caption{Locally weighted average (Nadaraya-Watson) versus local linear regression. }
    \end{figure}
  }
  To correct this, we can estimate a linear model at each point $x_0$ by solving a weighted least squares:
  \pause
  \begin{equation}
    \min_{\al(x_0),\beta(x_0)}\sum_{i=1}^n K_\la(x_0, x_i)[y_i - \alpha(x_0) - \beta(x_0)x_i]^2
  \end{equation}
\end{frame}

\begin{frame}
  \frametitle{Local linear regression estimates}
  \pause
  Let $b(x)^T = (1,x)$ and:
  \begin{equation}
    \bm B =
    \begin{pmatrix}
      1 & x_1 \\
      1 & x_2 \\
      \vdots & \vdots \\
      1 & x_n 
    \end{pmatrix}
    \quad
    \bm W(x_0) = 
    \begin{pmatrix}
      K_\la(x_0,x_1) & 0 & \cdots & 0 \\
      0             & K_\la(x_0,x_2) & \cdots & 0 \\
      \vdots & \vdots & \ddots & \vdots \\
      0 & 0 & \cdots & K_\la(x_0,x_n)
    \end{pmatrix}
  \end{equation}
  Then the solution to the locally weighted regression problem is:\pause
  \begin{eqnarray}
    \hat f(x_0) &=& \pause {\pl b(x_0)^T(\bm B^T\bm W(x_0)\bm B)^{-1}\bm B^T\bm W(x_0)}\bm y  \\\pause
    \hat f(x_{0})   &=& \pause {\pl \sum_{i=1}^n \ell_i(x_0) }y_i
  \end{eqnarray}
  \pause
  The weights $\ell_i(x_0)$ are called the \textbf{\pl equivalent kernel}
\end{frame}

\begin{frame}
  \frametitle{Effect of equivalent kernel}
  \pause
  The equivalent kernel (local regression) corrects the bias from local average kernel methods to the first order.\pause
  
  
   \begin{figure}[h!]
      \centering
      \visible<+->{\includegraphics[width=.95\textwidth,decodearray={0 .75 0 .75 0 1},
      trim={1cm 15.2cm 0cm 5.7cm},clip]{ESL6-4}}
      \caption{  \visible<+->{Green points show the weights $\ell_i(x_0)$ (rescaled for display purposes) along with those for the
          Nadaraya-Watson (N-W) local average (yellow shaded region; also rescaled).}
          \visible<+->{The correction effect of local
        regression can be observed.}}
  \end{figure}
\end{frame}

\begin{frame}
  \frametitle{Local polynomial regression}
  \pause
  \begin{itemize}[<+->]
  \item Higher-order terms in $\hat f(x)$ are required to reduce bias in curved regions
  \item Local polynomial regression can correct this at the cost of higher variance
  \item Given by:\pause
    \begin{equation}
      \hat f(x_0) = \hat\alpha(x_0) + \sum_{j=1}^d\hat\beta_j(x_0)x_0^j
    \end{equation}
    \pause
    where $\hat f(x_0)$ is the solution to:\pause
    \begin{equation}
      \min_{\alpha(x_0),\beta(x_0),j=1,\ldots,d} \sum_{i=1}^nK_\la (x_0,x_i)\lt[y_i-\alpha(x_0)-\sum_{j=1}^d\beta_j(x_0)x_i^j\rt]^2
    \end{equation}
  \end{itemize}
\end{frame}

\begin{frame}
  \frametitle{Bias correction of local quadratic regression}
  \pause
  \visible<+->{
     \begin{figure}[h!]
      \centering
      \includegraphics[width=1\textwidth,decodearray={0 .75 0 .75 0 1},
      trim={2cm 12.7cm 2cm 7.7cm},clip]{ESL6-5}
  \caption{The local linear regression estimator is biased in curved regions. A higher-order local fit (in this case, quadratic) can eliminate this.}
  \end{figure}}
\end{frame}

\begin{frame}[fragile]
  \frametitle{Local regression algorithms} \pause

  \begin{itemize}
  \item LOESS: locally estimated scatterplot smoothing \pause
    \begin{itemize}
    \item Identical approached earlier developed in 1964 by
\href{https://pubs.acs.org/doi/pdf/10.1021/ac60214a047}{Savitzky and Golay} for smoothing noisy data (known as the
Savitzky-Golay filter)
    \item ``Rediscovered'' by
\href{https://www.engineering.iastate.edu/~shermanp/STAT447/Lectures/Cleveland%20paper.pdf}{William Cleveland in 1979}
    \end{itemize}

  \item LOWESS: locally weighted scatterplot smoothing
    \begin{itemize}
    \item Extension of LOESS by
\href{https://sites.stat.washington.edu/courses/stat527/s14/readings/Cleveland_Delvin_JASA_1988.pdf}{Cleveland and Susan
Devlin (1988)}
    \end{itemize}

  \item LOESS can be considered a generalization of LOWESS, as it fits multivariate data, while LOWESS is for univariate
cases
  \end{itemize}
\end{frame}

\section{Outlook}
\begin{frame}
  \frametitle{Summary}
  \pause
  \begin{itemize}[<+->]

  \item Kernel smoothing methods can be used for flexible functional fitting
  \item Local regression generates a linear/polynomial fit at each target point using a kernel weighted loss function
  \end{itemize}

  \pause
  Reading:
  \begin{itemize}[<+->]
  \item \textbf{PMLI} 16
  \item \textbf{ESL 6.1} One-dimensional Kernel Smoothers (pp.\ 191--199)
  \item \textbf{ISLR 7.6:} Local Regression (pp.\ 280--282)
 % \item \textbf{ISLR 7.7:} Generalized Additive Models (pp.\ 282--287)
 % \item \textbf{ISLR 7.8:} Lab: Non-linear Modeling (pp.\ 287--297)

  \end{itemize}
\end{frame}



% \appendix\addtocounter{part}{-1}
% \begin{frame}
%   \frametitle{Kernel functions}
%   \begin{itemize}
%   \item Triangular: $K(u) = (1 - |u|)$
%   \item Epanechnikov: $K(u) = \fr34(1 - u^{2})$
%   \item Triweight: $K(u) = \fr{35}{32} (1-u^{2})^{3}$
%   \item Tricube: $K(u) = \fr{70}{81}(1-|u|^{3})^{3}$
%     \item Gaussian: $K(u) = \fr1{\sqrt{2\pi}}e^{-\fr12u^{2}}$
%   \end{itemize}
% \end{frame}
\end{document}

%%% Local Variables:
%%% mode: latex
%%% TeX-master: t
%%% End:
