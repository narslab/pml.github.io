%\documentclass[smaller, handout]{beamer}
\def\bmode{2} % Mode 0 for presentation, mode 1 for a handout with notes, mode 2 fo% r handout without notes
\if 0\bmode
\documentclass[smaller]{beamer}
\else \if 1\bmode
\immediate\write18{pdflatex -jobname=\jobname-Handout-Notes\space\jobname}
\documentclass[smaller,handout]{beamer}
\usepackage{handoutWithNotes}
\pgfpagesuselayout{2 on 1 with notes}[letterpaper, landscape, border shrink=4mm]
\else \if 2\bmode
\immediate\write18{pdflatex -jobname=\jobname-Handout\space\jobname}
\documentclass[smaller,handout]{beamer}
\fi
\fi
\fi

%\setbeamertemplate{section in head/foot}{}
%\setbeamertemplate{section in head/foot shaded}{\textcolor{white}{\insertsectionhead}}
%%%%%%%%%%%%%%%%%%%%%%%%%%%%%%%%%%%%%%%%%%%%%%%%%%%%%%%%%%%%%%%%%%%%%%%%%%%%%%%%%%%%%%%%%%%%%
\newcommand{\coursetitle}{CEE 616: Probabilistic Machine Learning}
\newcommand{\longlecturetitle}{M4 Nonparametric Methods:\\ L4b: Gaussian Processes}
\newcommand{\shortlecturetitle}{L4b: Gaussian Processes}
\newcommand{\instructor}{Jimi Oke}
\newcommand{\lecturedate}{Thu, Nov 12, 2025}
%%%%%%%%%%%%%%%%%%%%%%%%%%%%%%%%%%%%%%%%%%%%%%%%%%%%%%%%%%%%%%%%%%%%%%%%%%%%%%%%%%%%%%%%%%%%%

 
 
% \usepackage[T1]{fontenc} 
% \usepackage{lmodern} 
%\usepackage{etex}
 %\newcommand{\num}{6{} }

% \usetheme[
%   outer/progressbar=foot,
%   outer/numbering=fraction,
%   block=fill,
%   inner/subsectionpage=progressbar
% ]{metropolis}
\usetheme{Madrid}
\useoutertheme[subsection=false]{miniframes} % Alternatively: miniframes, infolines, split
\useinnertheme{circles}
% %\useoutertheme{Frankfurt}
% \usecolortheme{beaver}
% %\useoutertheme{crane}
% %\useoutertheme{metropolis}
\usepackage[backend=biber,style=authoryear,maxcitenames=2,maxbibnames=99,safeinputenc,url=false, eprint=false]{biblatex}
%\addbibresource{bib/references.bib}
% \AtEveryCitekey{\iffootnote{{\tiny}\tiny}{\tiny}}

% %\usepackage{pgfpages}
% %\setbeameroption{hide notes} % Only slides
% %\setbeameroption{show only notes} % Only notes
% %\setbeameroption{hide notes} % Only notes
% %\setbeameroption{show notes on second screen=right} % Both

% % \usepackage[sfdefault]{Fira Sans}

% % \setsansfont[BoldFont={Fira Sans}]{Fira Sans Light}
% % \setmonofont{Fira Mono}

% %\usepackage{fira}
% %\setsansfont{Fira}
% %\setmonofont{Fira Mono}
% % To give a presentation with the Skim reader (http://skim-app.sourceforge.net) on OSX so
% % that you see the notes on your laptop and the slides on the projector, do the following:
% % 
% % 1. Generate just the presentation (hide notes) and save to slides.pdf
% % 2. Generate onlt the notes (show only nodes) and save to notes.pdf
% % 3. With Skim open both slides.pdf and notes.pdf
% % 4. Click on slides.pdf to bring it to front.
% % 5. In Skim, under "View -> Presentation Option -> Synhcronized Noted Document"
% %    select notes.pdf.
% % 6. Now as you move around in slides.pdf the notes.pdf file will follow you.
% % 7. Arrange windows so that notes.pdf is in full screen mode on your laptop
% %    and slides.pdf is in presentation mode on the projector.

% % Give a slight yellow tint to the notes page
% \setbeamertemplate{note page}{\pagecolor{yellow!5}\insertnote}\usepackage{palatino}

% %\usetheme{metropolis}
% %\usecolortheme{beaver}
 \usepackage{tipa}
% \usepackage{enumerate}
\definecolor{darkcandyapplered}{HTML}{A40000}
\definecolor{lightcandyapplered}{HTML}{e74c3c}

% %\setbeamercolor{title}{fg=darkcandyapplered}

% \definecolor{UBCblue}{rgb}{0.04706, 0.13725, 0.26667} % UBC Blue (primary)
% \definecolor{UBCgrey}{rgb}{0.3686, 0.5255, 0.6235} % UBC Grey (secondary)

% % \setbeamercolor{palette primary}{bg=darkcandyapplered,fg=white}
% % \setbeamercolor{palette secondary}{bg=darkcandyapplered,fg=white}
% % \setbeamercolor{palette tertiary}{bg=darkcandyapplered,fg=white}
% % \setbeamercolor{palette quaternary}{bg=darkcandyapplered,fg=white}
% % \setbeamercolor{structure}{fg=darkcandyapplered} % itemize, enumerate, etc
% % \setbeamercolor{section in toc}{fg=darkcandyapplered} % TOC sections
% % \setbeamercolor{frametitle}{fg=darkcandyapplered,bg=white} % TOC sections
% % \setbeamercolor{title in head/foot}{bg=white,fg=white} % TOC sections
% % \setbeamercolor{button}{fg=darkcandyapplered} % TOC sections

% % % Override palette coloring with secondary
% % \setbeamercolor{subsection in head/foot}{bg=lightcandyapplered,fg=white}

%\usecolortheme{crane}
% \makeatletter
% \setbeamertemplate{headline}{%
%   \begin{beamercolorbox}[colsep=1.5pt]{upper separation line head}
%   \end{beamercolorbox}
%   \begin{beamercolorbox}{section in head/foot}
%     \vskip1pt\insertsectionnavigationhorizontal{\paperwidth}{}{}\vskip1pt
%   \end{beamercolorbox}%
%   \ifbeamer@theme@subsection%
%     \begin{beamercolorbox}[colsep=1.5pt]{middle separation line head}
%     \end{beamercolorbox}
%     \begin{beamercolorbox}[ht=2.5ex,dp=1.125ex,%
%       leftskip=.3cm,rightskip=.3cm plus1fil]{subsection in head/foot}
%       \usebeamerfont{subsection in head/foot}\insertsubsectionhead
%     \end{beamercolorbox}%
%   \fi%
%   \begin{beamercolorbox}[colsep=1.5pt]{lower separation line head}
%   \end{beamercolorbox}
% }
% \makeatother

% Reduce size of frame box
\setbeamertemplate{frametitle}{%
    \nointerlineskip%
    \begin{beamercolorbox}[wd=\paperwidth,ht=2.0ex,dp=0.6ex]{frametitle}
        \hspace*{1ex}\insertframetitle%
    \end{beamercolorbox}%
}


%\setbeamercolor{frametitle}{bg=darkcandyapplered!80!black!90!white}
%\setbeamertemplate{frametitle}{\bf\insertframetitle}

%\setbeamercolor{footnote mark}{fg=darkcandyapplered}
%\setbeamercolor{footnote}{fg=darkcandyapplered!70}
%\Raggedbottom
%\setbeamerfont{page number in head/foot}{size=\tiny}
%\usepackage[tracking]{microtype}


% %\usepackage[sc,osf]{mathpazo}   % With old-style figures and real smallcaps.
% %\linespread{1.025}              % Palatino leads a little more leading

% % Euler for math and numbers
% %\usepackage[euler-digits,small]{eulervm}
% %\AtBeginDocument{\renewcommand{\hbar}{\hslash}}
\usepackage{graphicx}
\usepackage{multirow}
\usepackage{booktabs}
\usepackage{graphbox}
\usepackage{animate}
\usepackage{media9}
\usepackage{adjustbox}

% %\mode<presentation> { \setbeamercovered{transparent} }

\setbeamertemplate{navigation symbols}{}
\makeatletter
\def\beamerorig@set@color{%
  \pdfliteral{\current@color}%
  \aftergroup\reset@color
}
\def\beamerorig@reset@color{\pdfliteral{\current@color}}
\makeatother


% %=== GRAPHICS PATH ===========
\graphicspath{{./m5-images/}}
% % Marginpar width
% %Marginpar width
% %\setlength{\marginparsep}{.02in}


% %% Captions
% % \usepackage{caption}
% % \captionsetup{
% %   labelsep=quad,
% %   justification=raggedright,
% %   labelfont=sc
% % }

% \setbeamerfont{caption}{size=\footnotesize}
% \setbeamercolor{caption name}{fg=darkcandyapplered}

% %AMS-TeX packages

\usepackage{amssymb}
\usepackage{amsmath}
\usepackage{amsthm}
\usepackage{mathtools} 
\usepackage{bm}
\DeclareMathOperator*{\argmax}{arg\,max}
\DeclareMathOperator*{\argmin}{arg\,min}
% \usepackage{color}

% %https://tex.stackexchange.com/a/31370/2269
\usepackage{cancel}
\renewcommand{\CancelColor}{\color{red}} %change cancel color to red
\makeatletter
\let\my@cancelto\cancelto %copy over the original cancelto command
\newcommand<>{\cancelto}[2]{\alt#3{\my@cancelto{#1}{#2}}{\mathrlap{#2}\phantom{\my@cancelto{#1}{#2}}}}
% redefine the cancelto command, using \phantom to assure that the
% result doesn't wiggle up and down with and without the arrow
\makeatother


% % \usepackage{comment}
\usepackage{enumerate}
\usepackage{hyperref}
% \usepackage{minitoc,array}
% \definecolor{slblue}{rgb}{0,.3,.62}
\hypersetup{
    colorlinks,%
    citecolor=blue,%
    filecolor=blue,%
    linkcolor=blue,
    urlcolor=blue
}

% \usepackage{epstopdf}
% \epstopdfDeclareGraphicsRule{.gif}{png}{.png}{convert gif:#1 png:\OutputFile}
% \AppendGraphicsExtensions{.gif}

% %\usepackage{listings}

% %%% TIKZ
\usepackage{forest}
\usepackage{tikz}
\usepackage{tikz-3dplot}
\usepackage{pgfplots}
\usepackage{pgfplotstable}
% \usepackage{pgfgantt}
\usepackage{neuralnetwork}

\usetikzlibrary{fit,arrows,arrows.meta,shapes,positioning,shapes.geometric}
\usetikzlibrary{decorations.markings}
\usetikzlibrary{shadows,automata}
\usetikzlibrary{patterns}
\usetikzlibrary{trees,mindmap,backgrounds}
%\usetikzlibrary{circuits.ee.IEC}
\usetikzlibrary{decorations.text}
% % For Sagnac Picture
% \usetikzlibrary{%
%     decorations.pathreplacing,%
%     decorations.pathmorphing%
% }
% \tikzset{no shadows/.style={general shadow/.style=}}
% %
% %\usepackage{paralist}

\tikzset{
  font=\Large\sffamily\bfseries,
  red arrow/.style={
    midway,red,sloped,fill, minimum height=3cm, single arrow, single arrow head extend=.5cm, single arrow head indent=.25cm,xscale=0.3,yscale=0.15,
    allow upside down
  },
  black arrow/.style 2 args={-stealth, shorten >=#1, shorten <=#2},
  black arrow/.default={1mm}{1mm},
  tree box/.style={draw, rounded corners, inner sep=1em},
  node box/.style={white, draw=black, text=black, rectangle, rounded corners},
}

% %%% FORMAT PYTHON CODE
% %\usepackage{listings}
% % Default fixed font does not support bold face
% \DeclareFixedFont{\ttb}{T1}{txtt}{bx}{n}{8} % for bold
% \DeclareFixedFont{\ttm}{T1}{txtt}{m}{n}{8}  % for normal

% % Custom colors
% \definecolor{deepblue}{rgb}{0,0,0.5}
% \definecolor{deepred}{rgb}{0.6,0,0}
% \definecolor{deepgreen}{rgb}{0,0.5,0}

% %\usepackage{animate}

% % Python style for highlighting
% % \newcommand\pythonstyle{\lstset{
% % language=Python,
% % basicstyle=\footnotesize\ttm,
% % otherkeywords={self},             % Add keywords here
% % keywordstyle=\footnotesize\ttb\color{deepblue},
% % emph={MyClass,__init__},          % Custom highlighting
% % emphstyle=\footnotesize\ttb\color{deepred},    % Custom highlighting style
% % stringstyle=\color{deepgreen},
% % frame=tb,                         % Any extra options here
%     % showstringspaces=false            % 
% % }}

% % % Python environment
% % \lstnewenvironment{python}[1][]
% % {
% % \pythonstyle
% % \lstset{#1}
% % }
% % {}

% % % Python for external files
% % \newcommand\pythonexternal[2][]{{
% % \pythonstyle
% % \lstinputlisting[#1]{#2}}}

% % Python for inline
% % 
% % \newcommand\pythoninline[1]{{\pythonstyle\lstinline!#1!}}

% %\usepackage{algorithm2e}

\newcommand{\eps}{\epsilon}
\newcommand{\bX}{\mb X}
\newcommand{\by}{\mb y}
\newcommand{\bbe}{\bm\beta}
\newcommand{\beps}{\bm\epsilon}
\newcommand{\bY}{\mb Y}

\newcommand{\osn}{\oldstylenums}
\newcommand{\dg}{^{\circ}}
\newcommand{\lt}{\left}
\newcommand{\rt}{\right}
\newcommand{\pt}{\phantom}
\newcommand{\tf}{\therefore}
\newcommand{\?}{\stackrel{?}{=}}
\newcommand{\fr}{\frac}
\newcommand{\dfr}{\dfrac}
\newcommand{\ul}{\underline}
\newcommand{\tn}{\tabularnewline}
\newcommand{\nl}{\newline}
\newcommand\relph[1]{\mathrel{\phantom{#1}}}
\newcommand{\cm}{\checkmark}
\newcommand{\ol}{\overline}
\newcommand{\rd}{\color{red}}
\newcommand{\bl}{\color{blue}}
\newcommand{\pl}{\color{purple}}
\newcommand{\og}{\color{orange!90!black}}
\newcommand{\gr}{\color{green!40!black}}
\newcommand{\lbl}{\color{CornflowerBlue}}
\newcommand{\dca}{\color{darkcandyapplered}}
\newcommand{\nin}{\noindent}
\newcommand*\circled[1]{\tikz[baseline=(char.base)]{
            \node[shape=circle,draw,thick,inner sep=1pt] (char) {\small #1};}}

\newcommand{\bc}{\begin{compactenum}[\quad--]}
\newcommand{\ec}{\end{compactenum}}

\newcommand{\p}{\partial}
\newcommand{\pd}[2]{\frac{\partial{#1}}{\partial{#2}}}
\newcommand{\dpd}[2]{\dfrac{\partial{#1}}{\partial{#2}}}
\newcommand{\pdd}[2]{\frac{\partial^2{#1}}{\partial{#2}^2}}
\newcommand{\pde}[3]{\frac{\partial^2{#1}}{\partial{#2}\partial{#3}}}
\newcommand{\nmfr}[3]{\Phi\left(\frac{{#1} - {#2}}{#3}\right)}
\newcommand{\Err}{\text{Err}}
\newcommand{\err}{\text{err}}

\DeclarePairedDelimiter\ceil{\lceil}{\rceil}
\DeclarePairedDelimiter\floor{\lfloor}{\rfloor}

%%%% GREEK LETTER SHORTCUTS %%%%%
\newcommand{\la}{\lambda}
\renewcommand{\th}{\theta}
\newcommand{\al}{\alpha}
\newcommand{\G}{\Gamma}
\newcommand{\si}{\sigma}
\newcommand{\Si}{\Sigma}


\pgfmathdeclarefunction{poiss}{1}{%
  \pgfmathparse{(#1^x)*exp(-#1)/(x!)}%
  }

\pgfmathdeclarefunction{gauss}{2}{%
  \pgfmathparse{1/(#2*sqrt(2*pi))*exp(-((x-#1)^2)/(2*#2^2))}%
}

\pgfmathdeclarefunction{expo}{2}{%
  \pgfmathparse{#1*exp(-#1*#2)}%
}

\pgfmathdeclarefunction{expocdf}{2}{%
  \pgfmathparse{1 -exp(-#1*#2)}%
}

\newcommand{\mb}{\mathbb}
\newcommand{\mc}{\mathcal}
\newcommand{\tr}{^{\top}}
\newcommand{\empt}[2]{$#1^{( #2 )}$}
\newcommand{\pe}{\pause}
% \usepackage{pst-plot}

% \usepackage{pstricks-add}
% \usepackage{auto-pst-pdf}   

% \psset{unit = 3}

% \def\target(#1,#2){%
%  {\psset{fillstyle = solid}
%   \rput(#1,#2){%
%     \pscircle[fillcolor = white](0.7,0.7){0.7}
%     \pscircle[fillcolor = blue!60](0.7,0.7){0.5}
%     \pscircle[fillcolor = white](0.7,0.7){0.3}
%     \pscircle[fillcolor = red!80](0.7,0.7){0.1}}}}
% \def\dots[#1](#2,#3){%
%     \psRandom[
%       dotsize = 2pt,
%       randomPoints = 25
%     ](!#2 #1 0.04 sub sub #3 #1 0.04 sub sub)%
%      (!#2 #1 0.04 sub add #3 #1 0.04 sub add)%
%      {\pscircle[linestyle = none](#2,#3){#1}}}


%%%%%%%%%%%%%%%%%%%%%%%%%%%%%%%%%%%%%%%%%%%%%%%%%%%
%%%%%%%%%%%%%%%%%%%%%%%%%%%%%%%%%%%%%%%%%%%%%%%%%%%
\title[\shortlecturetitle]{ {\normalsize \coursetitle}
  \\ \longlecturetitle}
\date[\lecturedate]{\footnotesize \lecturedate}
\author{{\bf \instructor}}
\institute[UMass Amherst]{
%\titlegraphic{\hfill
  \begin{tikzpicture}[baseline=(current bounding box.center)]
    \node[anchor=base] at (-7,0) (its) {\includegraphics[scale=.3]{UMassEngineering_vert}} ;
  \end{tikzpicture}
  % \hfill\includegraphics[height=1.5cm]{logo}
}

%https://tex.stackexchange.com/questions/55806/mindmap-tikzpicture-in-beamer-reveal-step-by-step
  \tikzset{
    invisible/.style={opacity=0},
    visible on/.style={alt={#1{}{invisible}}},
    alt/.code args={<#1>#2#3}{%
      \alt<#1>{\pgfkeysalso{#2}}{\pgfkeysalso{#3}} % \pgfkeysalso doesn't change the path
    },
  }


% https://tex.stackexchange.com/questions/446468/labels-with-arrows-for-an-equation
% https://tex.stackexchange.com/a/402466/121799
\newcommand{\tikzmark}[3][]{
\ifmmode
\tikz[remember picture,baseline=(#2.base)] \node [inner sep=0pt,#1](#2) {$#3$};
\else
\tikz[remember picture,baseline=(#2.base)] \node [inner sep=0pt,#1](#2) {#3};
\fi
}

% \lstset{language=matlab,
%                 basicstyle=\scriptsize\ttfamily,
%                 keywordstyle=\color{blue}\ttfamily,
%                 stringstyle=\color{blue}\ttfamily,
%                 commentstyle=\color{gray}\ttfamily,
%                 morecomment=[l][\color{gray}]{\#}
%               }


%%% Local Variables:
%%% mode: latex
%%% TeX-master: t
%%% End:

              
\begin{document}
\maketitle
\begin{frame}
  \frametitle{Outline}
  \tableofcontents
\end{frame}



 
  
 
%\section{Introduction}
 

\section{Introduction}

\begin{frame}
  \frametitle{Gaussian processes}
  \pe Given a domain $\mc X$, a Gaussian process (GP) defines a joint distribution over functions of the form
  $f: \mc X\in \mb R$. \pe
  
  \begin{block}{Assumptions}
    \pe
    \begin{itemize}
    \item Function values for $M>0$ inputs $\bm f = [f(\bm x_{1}), \ldots , f(\bm x_{M})]$ is jointly Gaussian \pe
      \begin{itemize}
      \item Mean: \pe $\bm \mu = m(\bm x_{1}, \ldots, \bm x_{M})$, \pe where $m$ is a mean function \pe
      \item Covariance: \pe $\bm \Sigma_{ij} = \mc K(\bm x_{i}, \bm x_{j})$, \pe where $\mc K$ is a Mercer kernel
      \end{itemize}
    \end{itemize}
  \end{block}
  \pe
  More formally, a GP can be considered as a finite number of r.v.'s which have a joint Gaussian distribution \pe
  \begin{equation}
    \bm f(\bm x) = \mc {GP} (\bm \mu, \bm \Sigma_{ij})
  \end{equation}
\end{frame}

\begin{frame}
  \frametitle{Gausssian processes and kernels}
  \pe

  \begin{itemize}
  \item The structure of the covariance $\bm \Sigma_{ij}$ is specified by a \textbf{kernel function} (which imposes a prior assumption of the distribution)\pe
  \item Gaussian processes can be used for: \pe
    \begin{itemize}
    \item regression \pe
    \item classification \pe
    \item clustering
    \end{itemize}    
  \end{itemize}
  \pe
  
  \begin{center}
    \includegraphics[width=.3\textwidth]{gp-ex}

    {\tiny Source: \url{http://krasserm.github.io/2018/03/19/gaussian-processes/}}
  \end{center}
  
\end{frame}

% \section{Local regression}%

\section{Mercer kernels}
\begin{frame}
  \frametitle{Mercer kernel}
  \pe
  \begin{itemize}
  \item Nonparametric methods require an approach to map \textbf{prior knowledge} regarding the pairwise
    similarity of input vectors $(\bm x_i,\bm x_j)$
    \pe

  \item This is achieved using a kernel function $\mc K$ \pe
  \item A Mercer kernel is any symmetric function $\mc K: \mc X \times \mc X \to\mb R^+$ such that: \pe
    \begin{equation}
      \sum_{i=1}^N\sum_{j=1}^N \mc K(\bm x_i, \bm x_j) c_ic_j \ge 0
    \end{equation}
    \pe
    for any set of $N$ points $\bm x_i\in \mc X$ and any $c_i \in \mb R$
\end{itemize}

\end{frame}

\begin{frame}
  \frametitle{Gram matrix view}
  \pe
  The Gram (or kernel) matrix is defined as:\pe
  \begin{equation}
    \bm K =
    \begin{pmatrix}
      \mc K(\bm x_1,\bm x_1) & \cdots & \mc K(\bm x_1,\bm x_N) \\
      \vdots & \ddots & \vdots \\
      \mc K(\bm x_N,\bm x_1) & \cdots & \mc K(\bm x_N,\bm x_N)       
    \end{pmatrix}
  \end{equation}
  If $\bm K$ is \textbf{positive definite} for any set of unique inputs $\bm x_i \in \mc X, n = 1:N$, then $\mc K$ is a Mercer kernel.
  \pe
  \begin{itemize}
  \item Thus, a Mercer kernel is also known as a positive definite kernel
  \end{itemize}
\end{frame}

\begin{frame}
  \frametitle{Mercer's theorem}
  \pe
  Any [symmetric] positive definite matrix $\bm K$ can be eigendecomposed as:\pe
  \begin{equation}
    \bm K = \bm U\tr \bm \Lambda\bm U
  \end{equation}
  \pe
  where $\bm \Lambda$ is a diagonal matrix of eigenvalues \pe $\la_i > 0$ and $\bm U$ the matrix of eigenvectors.
  \pe
  \begin{itemize}
  \item Each element $k_{ij}$ can be written as: \pe
      \begin{equation}
        k_{ij} = (\bm \Lambda^{\fr12}\bm u_i)\tr (\bm \Lambda^{\fr12}\bm u_j) \pe \equiv \bm \phi(\bm x_i)\tr\bm \phi(\bm x_j)
      \end{equation}
      \pe where $\bm u_i$ is the $i$-th column/row of $\bm U$.\pe
    \item We can then express each element as an inner product of feature vectors specified by $\bm u_i$: \pe
      \begin{equation}
        k_{ij} = \sum_{m}\phi_m(\bm x_i)\phi_m(\bm x_j)
      \end{equation}
  \end{itemize}
\end{frame}

\begin{frame}
  \frametitle{Popular Mercer kernels}\pe
  \begin{itemize}
  \item \textbf{Squared exponential} (SE)/RBF kernel (below: $\ell =1$)\pe

      
  \begin{center}
    \includegraphics[width=.1\textwidth]{se-matrix}

    {\tiny Source: \url{https://peterroelants.github.io/posts/gaussian-process-kernels/}}
  \end{center}
  \pe

    \item \textbf{Periodic} kernels (below: $\ell = 1, p=1$) \pe
          
  \begin{center}
    \includegraphics[width=.1\textwidth]{periodic-matrix}

    {\tiny Source: \url{https://peterroelants.github.io/posts/gaussian-process-kernels/}}
  \end{center}
  \pe
  
\item \textbf{Automatic relevancy determination} (ARD) kernel \pe

\item Kernels can be composed by addition, multiplication and other operations

  \end{itemize}
\end{frame}
\begin{frame}
  \frametitle{Squared exponential kernel}
  \pe
  The squared exponential (SE) kernel is given by:
  \pe
  \begin{equation}
    \mc K(\bm x,\bm x') = \exp\lt[-\fr{||\bm x - \bm x'||^2}{2\ell^2}\rt]
    \end{equation}
    \pe
    where $\ell$ is the length scale (or \textbf{bandwidth}) parameter \pe
    \begin{itemize}
    \item Note that the SE kernel kernel measures similarity between $\bm x$ and $\bm x'$ using a scaled Euclidean distance\pe
    \item SE is also referred to as Gaussian, RBF or exponentiated quadratic
    \end{itemize}
  \end{frame}

  
\section{Joint MVNs}
\begin{frame}
  \frametitle{Joint Gaussian r.v.'s}
  \pe
  If two random vectors $\bm x_1$ and $\bm x_2$ are jointly Gaussian, then their joint probability is given by:\pe
  \begin{equation}
    p
  \begin{pmatrix}
    \bm x_1\\ \bm x_2
  \end{pmatrix}
  = \mc N (\bm\mu,\bm\Sigma) \pe = 
  \mc N \lt(
  \begin{pmatrix}
    \bm \mu_1 \\\bm \mu_2
  \end{pmatrix}, 
  \begin{pmatrix}
    \bm\Sigma_{11} &   \bm \Sigma_{12}\\
    \bm\Sigma_{21} &   \bm \Sigma_{22}
  \end{pmatrix}
  \rt)
  \end{equation}
\end{frame}

\begin{frame}
  \frametitle{Marginalization}
  \pe
  Provides the distribution on each vector (partial information).\pe

  The marginal distributions are given by: \pe

  \begin{eqnarray}
    p(\bm x_1) &=& \pe \mc N(\bm x_1|\bm mu_1,\bm\Sigma_{11}) \\\pe
    p(\bm x_2) &=& \pe \mc N(\bm x_2|\bm mu_2,\bm\Sigma_{22})
  \end{eqnarray}

  \pe
  Marginal distributions are also Gaussian.
\end{frame}

\begin{frame}
  \frametitle{Conditioning}

  \pe

  The \textbf{posterior} conditional distribution is:\pe
  \begin{equation}
    p(\bm x_1 |\bm x_2) = \mc N(\bm x_1|\bm \mu_{1|2},\bm\Sigma_{1|2})
  \end{equation}
  \pe
  where:
  \begin{eqnarray*}
    \bm \mu_{1|2} &=&\pe \bm \mu_1 + \bm\Sigma_{12}\Sigma_{22}^{-1}(\bm x_2 -\bm\mu_2) \quad \pe \text{(posterior mean)}\\\pe
    \bm \Sigma_{1|2} &=& \pe \bm\Sigma_{11} - \bm\Sigma_{12}\bm\Sigma_{22}^{-1}\bm\Sigma_{21} \quad
                         \pe \text{(posterior covariance)}
  \end{eqnarray*}
\end{frame}

\section{Noise-free}
\begin{frame}
  \frametitle{Noise-free observations}
  \pe
  In standard GP, we assume that observations in the training set are noise-free (exact output based on a known function $f$):
  \pe
  
  \begin{eqnarray}
    \mc D &=& \{(\bm x_{n}, y_{n}): n=1:N\} \\\pe
    y_{n} &=& f (\bm x_{n}) \pe \quad \text{(noise-free observation)}
  \end{eqnarray}
  \pe
  \begin{center}
    \includegraphics[width=.3\textwidth]{gp-noise-free-2}

    {\tiny Source: \url{https://www.aidanscannell.com/post/gaussian-process-regression/}}
  \end{center}
  \pe

  To predict function outputs for new inputs $\bm x^{*}$, we estimate the \textit{posterior conditional distribution}
\end{frame}

\begin{frame}
  \frametitle{Joint distribution}
  The joint distribution $\bl p(\bm f_{X},\bm f_{*}|\bm X,\bm X_{*})$ of function outputs is given by: \pe
  
  \begin{equation}\bl
    \begin{pmatrix}
      \bm f_{X} \\ \bm f_{*}
    \end{pmatrix}
    \pe
    \sim
    \mc N\lt(
    \begin{pmatrix}
      \bm \mu_{X} \\ \bm \mu_{*}
    \end{pmatrix},
    \begin{pmatrix}
      \bm K_{X,X} & \bm K_{X,*} \\
      \bm K_{X,*}\tr & \bm K_{*,*}
    \end{pmatrix}
    \rt)
  \end{equation}
  \pe
  where:
  \begin{itemize}
  \item $\bm f_{X}$: function outputs over training set (interpolator): $[f(\bm x_{1}),\ldots, f(x_{N})]$\pe
  \item $\bm f_{*}$: function outputs over test set: $[f(\bm x^{*}_{1}),\ldots, f(x^{*}_{N_{*}})]$\pe
  \item Test inputs: $\bm X_{*} \in \bm R^{N_{*}\times D}$; \pe training inputs: $\bm X \in \mb R^{N\times D}$\pe
  \item Mean vectors: \pe $\bm \mu_{X} = m(\bm x_{1}, \ldots, \bm x_N)$ and \pe $\bm \mu_{*} = m(\bm x^{*}_{1}, \ldots, \bm x^{*}_{N_{*}})$ \pe
  \item Block covariance matrix: \pe
    \begin{itemize}
    \item $\bm K_{X,X} = \mc K(\bm X, \bm X) \in \mb R^{N\times N}$  \pe
    \item    $\bm K_{X,*} = \mc K(\bm X, \bm X_{*}) \in \mb R^{N\times N_{*}}$  \pe
    \item    $\bm K_{*,*} = \mc K(\bm X_{*}, \bm X_{*}) \in \mb R^{N_{*}\times N_{*}}$
  \end{itemize}

  \end{itemize}
\end{frame}


\begin{frame}
  \frametitle{Posterior conditional distribution}
  \pe

  We sample predictions from the posterior conditional distributon, which is specified as: \pe

  \begin{eqnarray}
    p(\bm f_{*}|\bm X_{*}, \mc D) \pe &=& \mc N(\bm f_{*}|\bm u_{*|X}, \bm \Sigma_{*|X}) \pe \\ \pe
    \bm \mu_{*|X} \pe &=& m(\bm X_{*}) + \bm K_{X,*}\tr K_{X,X}^{-1}(\bm f_{X} - m(\bm X)) \\ \pe
    \bm \Sigma_{*|X} \pe &=& \bm K_{*,*} - \bm K_{X,*}\tr \bm K_{X,X}^{-1}\bm K_{X,*}
  \end{eqnarray}
  
\end{frame}

\section{Noisy}
\begin{frame}
  \frametitle{Noisy observations}
  \pe
  If we have a noisy realization of the underlying function, then:\pe
  \begin{equation}
    y_n = f(\bm x_n) + \epsilon_n
  \end{equation}
  \pe
  where $\epsilon_n \sim \mc N(0,\sigma_y^2)$.
  
  \pe
  The covariance of the observed noisy responses $\bm y$ is thus given by: \pe
  \begin{equation}
    \text{Cov} [\bm y|\bm X] = \pe \bm K_{X,X} + \sigma_y^2\bm{I}_N \pe \equiv \hat{\bm K}_{X,X}
  \end{equation}
  \pe
  In scalar form, this can be written as:\pe

  \begin{equation}
    \text{Cov}[y_i, y_j] = \pe \text{Cov}[f_i, f_j] + \text{Cov}[\epsilon_i,\epsilon_j] \pe =
    \mc{K}(\bm x_i,\bm x_j) + \sigma_y^2\delta_{ij}
  \end{equation}
\end{frame}

\begin{frame}
  \frametitle{Joint density}
  \pe
  The joint distribution of the  observed data $\bm y$ and the noise-free function on test points $\bm f_*$ is: \pe
  \begin{equation}
        \begin{pmatrix}
      \bm f_{X} \\ \bm f_{*}
    \end{pmatrix}
    \pe
    \sim
    \mc N\lt(
    \begin{pmatrix}
      \bm \mu_{X} \\ \bm \mu_{*}
    \end{pmatrix},
    \begin{pmatrix}
      \hat{\bm K}_{X,X} & \bm K_{X,*} \\
      \bm K_{X,*}\tr & \bm K_{*,*}
    \end{pmatrix}
    \rt)
  \end{equation}
  \pe
  Note that this has the same form as the joint distribution in the noise-free case, except: $\bm K_{X,X}$ is replaced by $\hat{\bm K}_{X,X}$, which is the covariance of the training inputs with a constant variance $\sigma_y^2$ added to all the diagonal terms.

    \pe
  
  \begin{center}
    \includegraphics[width=.3\textwidth]{gp-noise}

    {\tiny GPR example with noisy inputs. Source: \url{https://www.aidanscannell.com/post/gaussian-process-regression/}}
  \end{center}
\end{frame}

\begin{frame}
  \frametitle{Posterior predictive density}
  \pe
  The conditional posterior distribution (posterior predictive) is given by:

    \begin{eqnarray}
    p(\bm f_{*}|\bm X_{*}, \mc D) \pe &=& \mc N(\bm f_{*}|\bm u_{*|X}, \bm \Sigma_{*|X}) \pe \\ \pe
    \bm \mu_{*|X} \pe &=& m(\bm X_{*}) + \bm K_{X,*}\tr \hat{\bm K}_{X,X}^{-1}(\bm y - m(\bm X)) \\ \pe
    \bm \Sigma_{*|X} \pe &=& \bm K_{*,*} - \bm K_{X,*}\tr \hat{\bm K}_{X,X}^{-1}\bm K_{X,*}
  \end{eqnarray}
\end{frame}

\section{Kernel learning}
\begin{frame}
  \frametitle{Optimizing kernel parameters}
  \pe
  The value of kernel [hyper]parameters $\bm \th$ affects prediction performance.\pe
  \begin{itemize}
  \item In SE kernels, we want to optimize the length scale $\ell$\pe
  \item For ARD kernels, we want to learn/optimize the characteristic length scale $\ell_d$ and the variance $\sigma^2$\pe
  \item Hyperparameters can be learned via \textbf{maximum marginal likelihood}
  \end{itemize}
\end{frame}

\begin{frame}
  \frametitle{Maximum marginal likelihood}
  \pe
  Assuming the mean function is zero, the prior is given by:\pe
  \begin{equation}
    p(\bm f|\bm X,\bm\th) = \mc N(\bm f|\bm 0,\bm K)
  \end{equation}
  \pe
  and the likelihood of each observation (conditioned on the latent function $\bm f$) can be written as:\pe
  \begin{equation}
    p(\bm y|\bm f,\bm X) = \prod_{n=1}^N\mc N(y_n |f_n,\sigma_y^2)
  \end{equation}
  \pe
  The marginal likelihood is thus given by:\pe
  \begin{equation}
    p(\bm y|\bm X,\bm \th) = \int p(\bm y|\bm f,\bm X) p(\bm f|\bm X,\bm\th) d\bm f
  \end{equation}
  \pe
  To find the optimal kernel parameters $\bm\th$, we maximize the \textbf{\rd log marginal likelihood}: \pe
  \begin{equation}\rd
    \log p(\bm y|\bm X,\bm\th) = \pe
    \log \mc N(\bm y|\bm 0 ,\hat{\bm K}_{X,X}) \pe =
    -fr12\bm y\tr\hat{\bm K}_{X,X}^{-1}\bm y - \fr12\log|\hat{\bm K}_{X,X}| - \fr{N}{2}\log(2\pi)
  \end{equation}
\end{frame}
\section{Outlook}
\begin{frame}
  \frametitle{Reading}
 
  \begin{itemize}[<+->]
\item \textbf{PMLI 17.1-2}
\item \url{https://colab.research.google.com/github/krasserm/bayesian-machine-learning/blob/dev/gaussian-processes/gaussian_processes.ipynb}
\item \url{https://peterroelants.github.io/posts/gaussian-process-kernels/}
\item \url{https://github.com/aidanscannell/probabilistic-modelling/blob/master/notebooks/gaussian-process-regression.ipynb}
  \end{itemize}
\end{frame}



%\appendix\addtocounter{part}{-1}

\end{document}

%%% Local Variables:
%%% mode: latex
%%% TeX-master: t
%%% End:
