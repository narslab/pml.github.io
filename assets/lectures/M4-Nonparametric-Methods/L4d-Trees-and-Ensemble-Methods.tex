%\documentclass[smaller, dvipsnames, handout]{beamer}
\def\bmode{2} % Mode 0 for presentation, mode 1 for a handout with notes, mode 2 fo% r handout without notes
\if 0\bmode
\immediate\write18{pdflatex -jobname=\jobname-Presentation\space\jobname}
\documentclass[usenames,dvipsnames,smaller]{beamer}
\else \if 1\bmode
\immediate\write18{pdflatex -jobname=\jobname-Handout-Notes\space\jobname}
\documentclass[usenames,dvipsnames,smaller,handout]{beamer}
\usepackage{handoutWithNotes}
\pgfpagesuselayout{2 on 1 with notes}[letterpaper, landscape, border shrink=4mm]
\else \if 2\bmode
\immediate\write18{pdflatex -jobname=\jobname-Handout\space\jobname}
\documentclass[usenames,dvipsnames,smaller,handout]{beamer}
\fi
\fi
\fi


%\setbeamertemplate{section in head/foot}{}
%\setbeamertemplate{section in head/foot shaded}{\textcolor{white}{\insertsectionhead}}
%%%%%%%%%%%%%%%%%%%%%%%%%%%%%%%%%%%%%%%%%%%%%%%%%%%%%%%%%%%%%%%%%%%%%%%%%%%%%%%%%%%%%%%%%%%%%
\newcommand{\coursetitle}{CEE 616: Probabilistic Machine Learning}
\newcommand{\longlecturetitle}{M4 Nonparametric Methods:\\ L4D: Trees and Ensemble Methods}
\newcommand{\shortlecturetitle}{L4D: Trees and Ensemble Methods}
\newcommand{\instructor}{Jimi Oke}
\newcommand{\lecturedate}{Tue, Nov 18, 2025}
%%%%%%%%%%%%%%%%%%%%%%%%%%%%%%%%%%%%%%%%%%%%%%%%%%%%%%%%%%%%%%%%%%%%%%%%%%%%%%%%%%%%%%%%%%%%%


 
% \usepackage[T1]{fontenc} 
% \usepackage{lmodern} 
%\usepackage{etex}
 %\newcommand{\num}{6{} }

% \usetheme[
%   outer/progressbar=foot,
%   outer/numbering=fraction,
%   block=fill,
%   inner/subsectionpage=progressbar
% ]{metropolis}
\usetheme{Madrid}
\useoutertheme[subsection=false]{miniframes} % Alternatively: miniframes, infolines, split
\useinnertheme{circles}
% %\useoutertheme{Frankfurt}
% \usecolortheme{beaver}
% %\useoutertheme{crane}
% %\useoutertheme{metropolis}
\usepackage[backend=biber,style=authoryear,maxcitenames=2,maxbibnames=99,safeinputenc,url=false, eprint=false]{biblatex}
%\addbibresource{bib/references.bib}
% \AtEveryCitekey{\iffootnote{{\tiny}\tiny}{\tiny}}

% %\usepackage{pgfpages}
% %\setbeameroption{hide notes} % Only slides
% %\setbeameroption{show only notes} % Only notes
% %\setbeameroption{hide notes} % Only notes
% %\setbeameroption{show notes on second screen=right} % Both

% % \usepackage[sfdefault]{Fira Sans}

% % \setsansfont[BoldFont={Fira Sans}]{Fira Sans Light}
% % \setmonofont{Fira Mono}

% %\usepackage{fira}
% %\setsansfont{Fira}
% %\setmonofont{Fira Mono}
% % To give a presentation with the Skim reader (http://skim-app.sourceforge.net) on OSX so
% % that you see the notes on your laptop and the slides on the projector, do the following:
% % 
% % 1. Generate just the presentation (hide notes) and save to slides.pdf
% % 2. Generate onlt the notes (show only nodes) and save to notes.pdf
% % 3. With Skim open both slides.pdf and notes.pdf
% % 4. Click on slides.pdf to bring it to front.
% % 5. In Skim, under "View -> Presentation Option -> Synhcronized Noted Document"
% %    select notes.pdf.
% % 6. Now as you move around in slides.pdf the notes.pdf file will follow you.
% % 7. Arrange windows so that notes.pdf is in full screen mode on your laptop
% %    and slides.pdf is in presentation mode on the projector.

% % Give a slight yellow tint to the notes page
% \setbeamertemplate{note page}{\pagecolor{yellow!5}\insertnote}\usepackage{palatino}

% %\usetheme{metropolis}
% %\usecolortheme{beaver}
 \usepackage{tipa}
% \usepackage{enumerate}
\definecolor{darkcandyapplered}{HTML}{A40000}
\definecolor{lightcandyapplered}{HTML}{e74c3c}

% %\setbeamercolor{title}{fg=darkcandyapplered}

% \definecolor{UBCblue}{rgb}{0.04706, 0.13725, 0.26667} % UBC Blue (primary)
% \definecolor{UBCgrey}{rgb}{0.3686, 0.5255, 0.6235} % UBC Grey (secondary)

% % \setbeamercolor{palette primary}{bg=darkcandyapplered,fg=white}
% % \setbeamercolor{palette secondary}{bg=darkcandyapplered,fg=white}
% % \setbeamercolor{palette tertiary}{bg=darkcandyapplered,fg=white}
% % \setbeamercolor{palette quaternary}{bg=darkcandyapplered,fg=white}
% % \setbeamercolor{structure}{fg=darkcandyapplered} % itemize, enumerate, etc
% % \setbeamercolor{section in toc}{fg=darkcandyapplered} % TOC sections
% % \setbeamercolor{frametitle}{fg=darkcandyapplered,bg=white} % TOC sections
% % \setbeamercolor{title in head/foot}{bg=white,fg=white} % TOC sections
% % \setbeamercolor{button}{fg=darkcandyapplered} % TOC sections

% % % Override palette coloring with secondary
% % \setbeamercolor{subsection in head/foot}{bg=lightcandyapplered,fg=white}

%\usecolortheme{crane}
% \makeatletter
% \setbeamertemplate{headline}{%
%   \begin{beamercolorbox}[colsep=1.5pt]{upper separation line head}
%   \end{beamercolorbox}
%   \begin{beamercolorbox}{section in head/foot}
%     \vskip1pt\insertsectionnavigationhorizontal{\paperwidth}{}{}\vskip1pt
%   \end{beamercolorbox}%
%   \ifbeamer@theme@subsection%
%     \begin{beamercolorbox}[colsep=1.5pt]{middle separation line head}
%     \end{beamercolorbox}
%     \begin{beamercolorbox}[ht=2.5ex,dp=1.125ex,%
%       leftskip=.3cm,rightskip=.3cm plus1fil]{subsection in head/foot}
%       \usebeamerfont{subsection in head/foot}\insertsubsectionhead
%     \end{beamercolorbox}%
%   \fi%
%   \begin{beamercolorbox}[colsep=1.5pt]{lower separation line head}
%   \end{beamercolorbox}
% }
% \makeatother

% Reduce size of frame box
\setbeamertemplate{frametitle}{%
    \nointerlineskip%
    \begin{beamercolorbox}[wd=\paperwidth,ht=2.0ex,dp=0.6ex]{frametitle}
        \hspace*{1ex}\insertframetitle%
    \end{beamercolorbox}%
}


%\setbeamercolor{frametitle}{bg=darkcandyapplered!80!black!90!white}
%\setbeamertemplate{frametitle}{\bf\insertframetitle}

%\setbeamercolor{footnote mark}{fg=darkcandyapplered}
%\setbeamercolor{footnote}{fg=darkcandyapplered!70}
%\Raggedbottom
%\setbeamerfont{page number in head/foot}{size=\tiny}
%\usepackage[tracking]{microtype}


% %\usepackage[sc,osf]{mathpazo}   % With old-style figures and real smallcaps.
% %\linespread{1.025}              % Palatino leads a little more leading

% % Euler for math and numbers
% %\usepackage[euler-digits,small]{eulervm}
% %\AtBeginDocument{\renewcommand{\hbar}{\hslash}}
\usepackage{graphicx}
\usepackage{multirow}
\usepackage{booktabs}
\usepackage{graphbox}
\usepackage{animate}
\usepackage{media9}
\usepackage{adjustbox}

% %\mode<presentation> { \setbeamercovered{transparent} }

\setbeamertemplate{navigation symbols}{}
\makeatletter
\def\beamerorig@set@color{%
  \pdfliteral{\current@color}%
  \aftergroup\reset@color
}
\def\beamerorig@reset@color{\pdfliteral{\current@color}}
\makeatother


% %=== GRAPHICS PATH ===========
\graphicspath{{./m5-images/}}
% % Marginpar width
% %Marginpar width
% %\setlength{\marginparsep}{.02in}


% %% Captions
% % \usepackage{caption}
% % \captionsetup{
% %   labelsep=quad,
% %   justification=raggedright,
% %   labelfont=sc
% % }

% \setbeamerfont{caption}{size=\footnotesize}
% \setbeamercolor{caption name}{fg=darkcandyapplered}

% %AMS-TeX packages

\usepackage{amssymb}
\usepackage{amsmath}
\usepackage{amsthm}
\usepackage{mathtools} 
\usepackage{bm}
\DeclareMathOperator*{\argmax}{arg\,max}
\DeclareMathOperator*{\argmin}{arg\,min}
% \usepackage{color}

% %https://tex.stackexchange.com/a/31370/2269
\usepackage{cancel}
\renewcommand{\CancelColor}{\color{red}} %change cancel color to red
\makeatletter
\let\my@cancelto\cancelto %copy over the original cancelto command
\newcommand<>{\cancelto}[2]{\alt#3{\my@cancelto{#1}{#2}}{\mathrlap{#2}\phantom{\my@cancelto{#1}{#2}}}}
% redefine the cancelto command, using \phantom to assure that the
% result doesn't wiggle up and down with and without the arrow
\makeatother


% % \usepackage{comment}
\usepackage{enumerate}
\usepackage{hyperref}
% \usepackage{minitoc,array}
% \definecolor{slblue}{rgb}{0,.3,.62}
\hypersetup{
    colorlinks,%
    citecolor=blue,%
    filecolor=blue,%
    linkcolor=blue,
    urlcolor=blue
}

% \usepackage{epstopdf}
% \epstopdfDeclareGraphicsRule{.gif}{png}{.png}{convert gif:#1 png:\OutputFile}
% \AppendGraphicsExtensions{.gif}

% %\usepackage{listings}

% %%% TIKZ
\usepackage{forest}
\usepackage{tikz}
\usepackage{tikz-3dplot}
\usepackage{pgfplots}
\usepackage{pgfplotstable}
% \usepackage{pgfgantt}
\usepackage{neuralnetwork}

\usetikzlibrary{fit,arrows,arrows.meta,shapes,positioning,shapes.geometric}
\usetikzlibrary{decorations.markings}
\usetikzlibrary{shadows,automata}
\usetikzlibrary{patterns}
\usetikzlibrary{trees,mindmap,backgrounds}
%\usetikzlibrary{circuits.ee.IEC}
\usetikzlibrary{decorations.text}
% % For Sagnac Picture
% \usetikzlibrary{%
%     decorations.pathreplacing,%
%     decorations.pathmorphing%
% }
% \tikzset{no shadows/.style={general shadow/.style=}}
% %
% %\usepackage{paralist}

\tikzset{
  font=\Large\sffamily\bfseries,
  red arrow/.style={
    midway,red,sloped,fill, minimum height=3cm, single arrow, single arrow head extend=.5cm, single arrow head indent=.25cm,xscale=0.3,yscale=0.15,
    allow upside down
  },
  black arrow/.style 2 args={-stealth, shorten >=#1, shorten <=#2},
  black arrow/.default={1mm}{1mm},
  tree box/.style={draw, rounded corners, inner sep=1em},
  node box/.style={white, draw=black, text=black, rectangle, rounded corners},
}

% %%% FORMAT PYTHON CODE
% %\usepackage{listings}
% % Default fixed font does not support bold face
% \DeclareFixedFont{\ttb}{T1}{txtt}{bx}{n}{8} % for bold
% \DeclareFixedFont{\ttm}{T1}{txtt}{m}{n}{8}  % for normal

% % Custom colors
% \definecolor{deepblue}{rgb}{0,0,0.5}
% \definecolor{deepred}{rgb}{0.6,0,0}
% \definecolor{deepgreen}{rgb}{0,0.5,0}

% %\usepackage{animate}

% % Python style for highlighting
% % \newcommand\pythonstyle{\lstset{
% % language=Python,
% % basicstyle=\footnotesize\ttm,
% % otherkeywords={self},             % Add keywords here
% % keywordstyle=\footnotesize\ttb\color{deepblue},
% % emph={MyClass,__init__},          % Custom highlighting
% % emphstyle=\footnotesize\ttb\color{deepred},    % Custom highlighting style
% % stringstyle=\color{deepgreen},
% % frame=tb,                         % Any extra options here
%     % showstringspaces=false            % 
% % }}

% % % Python environment
% % \lstnewenvironment{python}[1][]
% % {
% % \pythonstyle
% % \lstset{#1}
% % }
% % {}

% % % Python for external files
% % \newcommand\pythonexternal[2][]{{
% % \pythonstyle
% % \lstinputlisting[#1]{#2}}}

% % Python for inline
% % 
% % \newcommand\pythoninline[1]{{\pythonstyle\lstinline!#1!}}

% %\usepackage{algorithm2e}

\newcommand{\eps}{\epsilon}
\newcommand{\bX}{\mb X}
\newcommand{\by}{\mb y}
\newcommand{\bbe}{\bm\beta}
\newcommand{\beps}{\bm\epsilon}
\newcommand{\bY}{\mb Y}

\newcommand{\osn}{\oldstylenums}
\newcommand{\dg}{^{\circ}}
\newcommand{\lt}{\left}
\newcommand{\rt}{\right}
\newcommand{\pt}{\phantom}
\newcommand{\tf}{\therefore}
\newcommand{\?}{\stackrel{?}{=}}
\newcommand{\fr}{\frac}
\newcommand{\dfr}{\dfrac}
\newcommand{\ul}{\underline}
\newcommand{\tn}{\tabularnewline}
\newcommand{\nl}{\newline}
\newcommand\relph[1]{\mathrel{\phantom{#1}}}
\newcommand{\cm}{\checkmark}
\newcommand{\ol}{\overline}
\newcommand{\rd}{\color{red}}
\newcommand{\bl}{\color{blue}}
\newcommand{\pl}{\color{purple}}
\newcommand{\og}{\color{orange!90!black}}
\newcommand{\gr}{\color{green!40!black}}
\newcommand{\lbl}{\color{CornflowerBlue}}
\newcommand{\dca}{\color{darkcandyapplered}}
\newcommand{\nin}{\noindent}
\newcommand*\circled[1]{\tikz[baseline=(char.base)]{
            \node[shape=circle,draw,thick,inner sep=1pt] (char) {\small #1};}}

\newcommand{\bc}{\begin{compactenum}[\quad--]}
\newcommand{\ec}{\end{compactenum}}

\newcommand{\p}{\partial}
\newcommand{\pd}[2]{\frac{\partial{#1}}{\partial{#2}}}
\newcommand{\dpd}[2]{\dfrac{\partial{#1}}{\partial{#2}}}
\newcommand{\pdd}[2]{\frac{\partial^2{#1}}{\partial{#2}^2}}
\newcommand{\pde}[3]{\frac{\partial^2{#1}}{\partial{#2}\partial{#3}}}
\newcommand{\nmfr}[3]{\Phi\left(\frac{{#1} - {#2}}{#3}\right)}
\newcommand{\Err}{\text{Err}}
\newcommand{\err}{\text{err}}

\DeclarePairedDelimiter\ceil{\lceil}{\rceil}
\DeclarePairedDelimiter\floor{\lfloor}{\rfloor}

%%%% GREEK LETTER SHORTCUTS %%%%%
\newcommand{\la}{\lambda}
\renewcommand{\th}{\theta}
\newcommand{\al}{\alpha}
\newcommand{\G}{\Gamma}
\newcommand{\si}{\sigma}
\newcommand{\Si}{\Sigma}


\pgfmathdeclarefunction{poiss}{1}{%
  \pgfmathparse{(#1^x)*exp(-#1)/(x!)}%
  }

\pgfmathdeclarefunction{gauss}{2}{%
  \pgfmathparse{1/(#2*sqrt(2*pi))*exp(-((x-#1)^2)/(2*#2^2))}%
}

\pgfmathdeclarefunction{expo}{2}{%
  \pgfmathparse{#1*exp(-#1*#2)}%
}

\pgfmathdeclarefunction{expocdf}{2}{%
  \pgfmathparse{1 -exp(-#1*#2)}%
}

\newcommand{\mb}{\mathbb}
\newcommand{\mc}{\mathcal}
\newcommand{\tr}{^{\top}}
\newcommand{\empt}[2]{$#1^{( #2 )}$}
\newcommand{\pe}{\pause}
% \usepackage{pst-plot}

% \usepackage{pstricks-add}
% \usepackage{auto-pst-pdf}   

% \psset{unit = 3}

% \def\target(#1,#2){%
%  {\psset{fillstyle = solid}
%   \rput(#1,#2){%
%     \pscircle[fillcolor = white](0.7,0.7){0.7}
%     \pscircle[fillcolor = blue!60](0.7,0.7){0.5}
%     \pscircle[fillcolor = white](0.7,0.7){0.3}
%     \pscircle[fillcolor = red!80](0.7,0.7){0.1}}}}
% \def\dots[#1](#2,#3){%
%     \psRandom[
%       dotsize = 2pt,
%       randomPoints = 25
%     ](!#2 #1 0.04 sub sub #3 #1 0.04 sub sub)%
%      (!#2 #1 0.04 sub add #3 #1 0.04 sub add)%
%      {\pscircle[linestyle = none](#2,#3){#1}}}


%%%%%%%%%%%%%%%%%%%%%%%%%%%%%%%%%%%%%%%%%%%%%%%%%%%
%%%%%%%%%%%%%%%%%%%%%%%%%%%%%%%%%%%%%%%%%%%%%%%%%%%
\title[\shortlecturetitle]{ {\normalsize \coursetitle}
  \\ \longlecturetitle}
\date[\lecturedate]{\footnotesize \lecturedate}
\author{{\bf \instructor}}
\institute[UMass Amherst]{
%\titlegraphic{\hfill
  \begin{tikzpicture}[baseline=(current bounding box.center)]
    \node[anchor=base] at (-7,0) (its) {\includegraphics[scale=.3]{UMassEngineering_vert}} ;
  \end{tikzpicture}
  % \hfill\includegraphics[height=1.5cm]{logo}
}

%https://tex.stackexchange.com/questions/55806/mindmap-tikzpicture-in-beamer-reveal-step-by-step
  \tikzset{
    invisible/.style={opacity=0},
    visible on/.style={alt={#1{}{invisible}}},
    alt/.code args={<#1>#2#3}{%
      \alt<#1>{\pgfkeysalso{#2}}{\pgfkeysalso{#3}} % \pgfkeysalso doesn't change the path
    },
  }


% https://tex.stackexchange.com/questions/446468/labels-with-arrows-for-an-equation
% https://tex.stackexchange.com/a/402466/121799
\newcommand{\tikzmark}[3][]{
\ifmmode
\tikz[remember picture,baseline=(#2.base)] \node [inner sep=0pt,#1](#2) {$#3$};
\else
\tikz[remember picture,baseline=(#2.base)] \node [inner sep=0pt,#1](#2) {#3};
\fi
}

% \lstset{language=matlab,
%                 basicstyle=\scriptsize\ttfamily,
%                 keywordstyle=\color{blue}\ttfamily,
%                 stringstyle=\color{blue}\ttfamily,
%                 commentstyle=\color{gray}\ttfamily,
%                 morecomment=[l][\color{gray}]{\#}
%               }


%%% Local Variables:
%%% mode: latex
%%% TeX-master: t
%%% End:

%\setbeamercolor{local structure}{fg=white}

\begin{document}
\maketitle
\begin{frame}
  \frametitle{Outline}
  \tableofcontents
\end{frame}


\section{Introduction}

\begin{frame}
  \frametitle{Decision trees}\pause

  \begin{itemize}[<+->]
  \item Consider a regression problem  with continuous response $y$ and inputs $X_1$ and $X_2$.

  \item  Now, imagine we want to model response $y$ as a constant across different regions of the input space.

  \item   One option can be:

  \begin{figure}[h!]
    \centering
    \includegraphics<5->[width=.5\textwidth, trim={0cm 17.5cm 10cm 3cm}, clip]{ESL9-2}
    %\caption{Partitioning input space for predicting a region-wise constant response}
  \end{figure}

\item Can you describe the region using inequalities?
\item How would you estimate response $y$ in each region?
\end{itemize}
\end{frame}

\begin{frame}
  \frametitle{Recursive binary partitioning}\pause
  In order to ensure simplicity and interpretability, we consider splitting a dataset using the \textbf{recursive binary partitioning} (RBP) algorithm
  \pause

    \begin{figure}[h!]
    \centering
    \includegraphics<3->[width=.5\textwidth, trim={10cm 17.5cm 0cm 3cm}, clip]{ESL9-2}
    %\caption{Recursive binary partitioning of input space}
  \end{figure}
  \pause
  \vspace{-2ex}
  Steps:
  \begin{enumerate}[<+->]
  \item Split the input space at $X_1 = t_1$; the response is modeled as the mean of $y$ in each region
  \item Split the region $X_1 \le t_1$ at $X_2 = t_2$ and the region $X_1 > t_1$ at $X_1 = t_3$
  \item Split the region $X_1 > t_3$ at $X_2 = t_4$
  \end{enumerate}
  \pause
  Thus we obtain 5 regions. In each case, the \textit{decision} is to choose the \textbf{variable} and the \textbf{split-point} that provide the best fit.
\end{frame}

\begin{frame}
  \frametitle{Decision trees (cont.)}
  \pause

  \begin{figure}[h!]
    \centering
    \includegraphics<2->[width=.9\textwidth, trim={0cm 8.5cm 0cm 11cm}, clip]{ESL9-2}
    % \caption{Recursive binary partitioning of input space}
  \end{figure}
  \pause

  \begin{itemize}
  \item The partitioning steps can be displayed as a \textbf{decision tree}
  \item The split points are the \textbf{internal nodes}
  \item Lines connecting the nodes are called \textit{branches}
  \item The final regions are \textbf{terminal nodes} or \textbf{leaves} ($R_1$, $R_2$, $R_3$, $R_4$ and $R_5$ in the figure)
  \end{itemize}
  
\end{frame}


\begin{frame}
  \frametitle{Ensemble learning}
  \pause
      Ensemble methods employ various techniques to combine predictions from multiple models. \pause
          This approach can be termed ``learning by committee''

  % Now, imagine the men represent \textbf{various models} and the elephant were the \textbf{response} being predicted. \\

  % \medskip
  
  % \pause

  % \begin{itemize}[<+->]
  % \item How would we take advantage of the different perspectives offered by each model?
  % \end{itemize}
  % \pause

  %\begin{block}{Learning by committee}
  \begin{minipage}{.5\linewidth}

 
    \begin{center}
      \includegraphics[width=.8\textwidth]{combining-classifiers}

      {\tiny Source: \url{http://www.scholarpedia.org/article/File:Combining_classifiers2.jpg}}
    \end{center}

 
  \end{minipage}\pe
  \begin{minipage}{.45\linewidth}
    
    We will consider three approaches with a focus on decision trees: \pause

    \begin{enumerate}
    \item \textbf{Bagging:} average/majority vote over bootstrap sample \pause
    \item \textbf{Random Forests:} average/majority vote over collection of de-correlated trees \pause
    \item \textbf{Boosting:} weighted aggregation of weak learners
    \end{enumerate}
  \end{minipage}
  \end{frame}

 
  
\section{Regression trees}

\begin{frame}
  \frametitle{Regression tree}
  Given $D$ features and $N$ observations, how do we predict the response for an observation $\bm x_0$ using a regression tree?

  \bigskip
  
  \pause
  The goal is to partition the input space into $M$ regions using the \textbf{recursive binary partitioning} (RBP) approach.

  \bigskip
  
  \begin{itemize}[<+->]
  \item This is a greedy, top-down approach
  \item We find the RSS-minimizing split (variable-cutpoint pair) for each subsequent region
  \item Tree-growing is terminated using a reasonable stopping condition
  \end{itemize}

\end{frame}

\begin{frame}
  \frametitle{Recursive binary partitioning decision}

    \pause
    For a given $j$th variable and split point $s$, we have a pair of half-planes:\pause
    \begin{eqnarray}
      R_1(j,s) &=& \{\bm x :\bm x_j < t\}\\\pause
      R_2(j,s) &=& \{\bm x : \bm x_j \ge t\}
    \end{eqnarray}
    \pause
    Find splitting variable $j^*$ and point $s^*$ that solve:
    \begin{equation}\bl
      j^*, s^* = \pause
      \arg\min_{j,t}\lt[ \sum_{x_i \in R_1(j,t)} (y_i - \hat y_{R_1})^2 + \sum_{x_i \in R_2(j,t)} (y_i - \hat y_{R_2})^2  \rt]
    \end{equation}
    \pause
    where \pause
    \begin{eqnarray}
      \hat y_{R_1} &=& \text{ave} (y_i : \bm x_i \in R_1 (j,t)) \\\pe
      \hat y_{R_2} &=& \text{ave} (y_i : \bm  x_i \in R_2 (j,t)) 
    \end{eqnarray}
\end{frame}

\begin{frame}
  \frametitle{Overfitting and pruning}
  \pause
  \begin{itemize}[<+->]
  \item Earlier, we mentioned  tree partitioning ends when the appropriate terminating condition is met
      \begin{itemize}[<+->]
      \item Traditional thresholding can be myopic, as a potentially better solutions could be discarded.
      \end{itemize}
    
  \item Also, we want to avoid excessively large trees, as these are prone to overfitting
  \item How then can we guarantee the best result?
    % \item Key question: how do we avoid overfitting by keeping tree size to an acceptable minimum?
  \end{itemize}
  \pause
  \begin{alertblock}{Strategy}
    \pause
    \begin{itemize}[<+->]
    \item Grow a large tree $T_0$ to a minimum specified node size
    \item Use \textit{cost-complexity pruning} to prune $T_0$ to find an acceptable subtree $T \subset T_0$
    \end{itemize}
  \end{alertblock}
\end{frame}

\begin{frame}
  \frametitle{Cost-complexity pruning}
  \pause
  Let $m$ be the index of terminal nodes (recall that each node represents a region). Then let: \pause
  \begin{eqnarray}
    N_m      &=& \sum_{i=1}^{N} \mb I(\bm x_i \in R_m) \\\pe
    \hat y_m &=& \fr1{N_m}\sum_{\bm x_i\in R_m} y_i \\ \pe
    Q_m(T)   &=& \fr1{N_m}\sum_{\bm x_i\in R_m} (y_i - \hat y_m)^2 \quad \text{(a measure of impurity)}
  \end{eqnarray}
  \pause
  The \textbf{\pl cost-complexity criterion} is then given by:
  \begin{equation}\pl
    C_\alpha(T) = \sum_{m = 1}^{|T|} N_m Q_m(T) + \alpha |T| 
  \end{equation}
  \pause
  where $|T|$ is the number of terminal nodes in $T$ and $\alpha \ge 0$ is a tuning parameter.
\end{frame}

\begin{frame}
  \frametitle{Cost-complexity pruning (cont.)}
  \pause
  Using the cost-complexity criterion, we find subtree $T_\alpha \subseteq T_0$ that minimizes $C_\alpha(T)$. \\ \pause

  \medskip
  
  \begin{itemize}[<+->]
  \item $\alpha |T|$ is a complexity penalty term
  \item If $\alpha = 0$, then $T_\alpha = T_0$
  \item Larger $\alpha$ values result in smaller trees $T_\alpha$
  \item We find $T_\alpha$ using a procedure known as \textit{\bl weakest link pruning}\footnote{Consult ESL p.\ 308 for further details on this algorithm.}
  \end{itemize}

  \medskip
  \pe

  Your final question should be, how do we find the optimal $\hat\alpha$?
  \pause
  \begin{alertblock}{Answer}
    Cross-validation \pause ($k$-fold)
  \end{alertblock}
\end{frame}
\section{Classification trees}
\begin{frame}
  \frametitle{Classification trees}
  \begin{itemize}[<+->]
  \item In regression trees, the predicted response for an observation in a certain region (terminal node) is given by the \textit{mean response} of observations at that node
  \item In classfication trees, the \textit{most commonly occurring class} is the predicted response
  \item To make the binary splits in classification trees, we use a suitable impurity measure in place of the RSS
  \end{itemize}
  \pause
  The proportion of class $c$ observations in node $m$ is given by:\pause
  \begin{equation}\bl
    \hat p_{mc} = \fr1{N_m}\sum_{\bm x_i\in R_m} \mb I(y_i = c)
  \end{equation}
  We then classify observations in node $m$ to the class with maximum representation:\pause
  \begin{equation}
    c(m) = \arg\max_k\hat p_{mc}
  \end{equation}
\end{frame}

\begin{frame}
  \frametitle{Node impurity measures}
  \pause
  The recursive binary partitioning splits the observations into regions $R_m$ to minimize \textbf{node impurity} at $m$, measured by:
  \pe
  \begin{itemize}[<+->]
  \item \textbf{Misclassification error}
    \begin{equation}
      E = \fr1{N_m}\sum_{i\in R_m}\mb I (y_i \ne c(m)) = 1 - \hat p_{mc(m)}
    \end{equation}
    \pause
    where $ck(m) = \arg\max_k\hat p_{mc}$

  \item  \textbf{Gini index}\pause
    \begin{equation}
      G = \sum_{c\ne c'}\hat p_{mc}\hat p_{mc'} = \pause \sum_{c=1}^C\hat p_{mc}(1 - \hat p_{mc})
    \end{equation}

  \item \textbf{Cross-entropy (deviance)} \pause
    \begin{equation}
      D = - \sum_{c=1}^C\hat p_{mc}\log\hat p_{mc}
    \end{equation}
  \end{itemize}
  
\end{frame}

\begin{frame}
  \frametitle{Node impurity measures (cont.)}
  \pause

  \begin{itemize}[<+->]
  \item The Gini index and the cross-entropy are more sensitive to changes in node probabilities
  \item They are also continuously differentiable can be readily optimized  
  \end{itemize}
  
  \pause
  \begin{figure}[h!]
    \centering
    \includegraphics<3->[width=.9\textwidth, trim={1cm 12cm 1cm 8cm}, clip]{ESL9-3}
    \caption{Various impurity measures plotted against $p$ in the two-class case}
  \end{figure}

  \pause
  Typically, the misclassification error is used in pruning, while the others are used in growing the tree
\end{frame}

 
 
\section{Bagging}

\begin{frame}
  \frametitle{BAGGING: Bootstrap AGGregatING}
  \pause

  Consider a training data set \pause $\mc D = \{(\bm x_1,y_1), (\bm x_2, y_2), \ldots, (\bm x_n,y_n)\}$. \pause
  Let a fitted decision tree be given by $\hat f(\bm x)$. \pause

  \begin{itemize}[<+->]
  \item Predictions have low bias but high variance (trees are unstable)
  \item Obtaining  predictions from trees fitted to $B$ bootstrap samples reduces  variance (\textbf{bagging})
  \end{itemize}

  \pause

  
  \begin{adjustbox}{width=.6\textwidth, center}
    \begin{forest}
  for tree={l sep=3em, s sep=3em, anchor=center, inner sep=0.7em, fill=blue!50, circle, where level=2{no edge}{}}
  [
  Training Data, node box
  [sample and feature bagging, node box, alias=bagging, above=4em
  [,red!70,alias=a1[[,alias=a2][]][,red!70,edge label={node[above=1ex,red arrow]{}}[[][]][,red!70,edge label={node[above=1ex,red arrow]{}}[,red!70,edge label={node[below=1ex,red arrow]{}}][,alias=a3]]]]
  [,red!70,alias=b1[,red!70,edge label={node[below=1ex,red arrow]{}}[[,alias=b2][]][,red!70,edge label={node[above=1ex,red arrow]{}}]][[][[][,alias=b3]]]]
  [~~$\dots$~,scale=2,no edge,fill=none,yshift=-4em]
  [,red!70,alias=c1[[,alias=c2][]][,red!70,edge label={node[above=1ex,red arrow]{}}[,red!70,edge label={node[above=1ex,red arrow]{}}[,alias=c3][,red!70,edge label={node[above=1ex,red arrow]{}}]][,alias=c4]]]]
  ]
  \pause
  \node[tree box, fit=(a1)(a2)(a3)](t1){};\pause
  \node[tree box, fit=(b1)(b2)(b3)](t2){};\pause
  \node[tree box, fit=(c1)(c2)(c3)(c4)](tn){};\pause
  \node[below right=0.5em, inner sep=0pt] at (t1.north west) {Tree 1};\pause
  \node[below right=0.5em, inner sep=0pt] at (t2.north west) {Tree 2};\pause
  \node[below right=0.5em, inner sep=0pt] at (tn.north west) {Tree $B$};\pause
  \draw[black arrow={5mm}{4mm}] (bagging) -- (t1.north);
  \draw[black arrow] (bagging) -- (t2.north);
  \draw[black arrow={5mm}{4mm}] (bagging) -- (tn.north);
  % \visible<+->{
  \pause
  \path (t1.south west)--(tn.south east) node[midway,below=4em, node box] (mean) {mean in regression or majority vote in classification};
 \draw[black arrow={5mm}{5mm}] (t1.south) -- (mean);
  \draw[black arrow] (t2.south) -- (mean);
  \draw[black arrow={5mm}{5mm}] (tn.south) -- (mean);
  % \visible<+->{
  \pause
  \node[below=3em of mean, node box] (pred) {prediction};
  \draw[black arrow] (mean) -- (pred);
\end{forest}
\end{adjustbox}


\end{frame}

 
\begin{frame}
  \frametitle{Bagging algorithm}
     \pause

    \begin{enumerate}[<+->]
    \item Sample with replacement from $\mc D$ to generate $B$ bootstrap samples, $\{\mc D^{*b}: b= 1 1, 2, \ldots, B\}$

    \item Fit a decision tree to each $\mc D^{*b}$ to obtain prediction $\hat f^{*b}(\bm x)$

    \item  Obtain the \textbf{\bl bagging estimate}: \pause
      \begin{enumerate}[\bf a]
      \item Regression: Average the predictions      \pe
      \begin{equation}\bl
        \hat f_{\text{bag}}(\bm x) =\pause \fr 1 B \sum_{b=1}^B \hat f^{*b}(\bm x)
      \end{equation}
      \pe
         \item Classification: majority vote \pe
       \begin{equation}\bl
        \hat f_{\text{bag}}(\bm x) = \argmax_{c} \sum_{b=1}^{B} \mb I( \hat f^{*b}(\bm x) = c)
      \end{equation}
     \end{enumerate}

    \end{enumerate}
 \end{frame}
\begin{frame}
  \frametitle{Example 1: Bagging trees on simulated dataset}
  \pause
  Given a dataset of $n=30$ observations with $p=5$ predictors \pause
  \begin{itemize}[<+->]
  \item Pairwise correlation for each predictor is 0.95
  \item 200 bootstrap samples are created and unpruned decision trees are fitted
  \item High correlations leads to high variance in individual decision tree estimates %(6 are shown below)
%  \item 
  \end{itemize}

  \begin{figure}\centering
    \visible<6->{\includegraphics[width=.55\textwidth, trim={1cm 13cm 2cm 3.5cm}, clip]{ESL8-9}}
    \visible<7->{\includegraphics[width=.4\textwidth, trim={2cm 12cm 2cm 4cm}, clip]{ESL8-10}}
    
    \caption{\footnotesize (L): Fitted trees on 6 bootstramp samples. (R): Test errors.}
\end{figure}

\end{frame}

% \begin{frame}
%   \frametitle{Example 1: Bagging trees on simulated dataset (cont.)}
%   \pause

%   Bagging reduces the variance in prediction.
%   \pause
  
%   \begin{figure}[h!]
%     \centering
%     \visible<3->{\includegraphics[width=.7\textwidth, trim={2cm 12cm 2cm 4cm}, clip]{ESL8-10}}
%     \caption{Errors from classification by majority (consensus) vote are shown in orange, while those from averaging probabilities are shown in green.}
%   \end{figure}
% \end{frame}

\begin{frame}
  \frametitle{Out-of-bag error}
  \pause

  Bootstrap samples only contain 63\% of the unique observations on average. Why?\pe

  \begin{itemize}[<+->]
  \item Observations left out of  bootstrap sample are \textbf{out-of-bag} (OOB) instances\pe
  \item Errors can be computed on each OOB instance per bootstrap sample \pe
  \item These are then averaged to obtain a validation error estimate  for use in hyperparameter selection
  \end{itemize}
  \pause
  
  \begin{figure}[h!]
      \centering
      \visible<6->{\includegraphics[width=.35\textwidth, trim={1cm .7cm 1cm 1cm}, clip]{8-8}}
      \caption{\footnotesize Comparing test (validation set approach) and OOB (similar to cross-validation) errors
        on bagging and random forest estimates on the \texttt{Heart} dataset (survival of heart transplant patients).}
  \end{figure}
\end{frame}
  
\begin{frame}
  \frametitle{Feature importance}
  \pause

  Bagging provides stability (reduced variance) at the expense of simplicity (interpretability). \pause

  \begin{itemize}[<+->]
  \item To aid inference, we compute \textbf{feature importances} of bagged tree predictors: \pe
    \begin{equation}
      fi_{d}(T) = \sum_{m=1}^{M-1}G_{m}\mb I(t_{m} = d)
    \end{equation} \pe
    where
    \begin{itemize}
    \item $fi_{d}(T)$ is the importance of feature $d$ in tree $T$ \pe
    \item $G_{m}$ is the gain in accuracy (reduction in loss) at node $m$\pe
    \item $j_{m} =d$ if node $m$ uses feature $d$ as its splitting variable $t$
    \end{itemize}\pe
  \item These are then averaged over all $B$ trees (estimators) \pe
  \item {Regression trees:} total amount of RSS decreases due to splits 
  \item {Classification trees:}  Gini index used  
  \end{itemize}
\end{frame}


\section{Random Forests}

\begin{frame}
  \frametitle{Random forests}
  \pause

  Invented by \href{https://link.springer.com/content/pdf/10.1023/A:1010933404324.pdf}{\bl Leo Breiman in 2001},
  \textbf{random forests} build on bagging by fitting \textit{decorrelated} trees on the bootstrap samples.\pe

  \begin{itemize}[<+->]
 
  \item   Implemented by splitting the observations by considering only a \textit{random} subset $S \subset D$ of predictors.
    
  \item Helps to further reduce variance when predicted values are highly correlated across samples (due to correlated predictors).

  \item In bagging, $S=D$.

  \item Recommended values of $S$: \pause   $S = \floor*{\fr D 3}$ (regression); $S = \sqrt{D}$ (classification)
  \end{itemize}
  \pe
  \begin{block}{Algorithm}
    \pe
      \begin{enumerate}[<+->]
  \item For $b = 1$ to $B$: \pause
    \begin{enumerate}[(a)]
    \item Draw a bootstrap sample $\mc D^*$ of size $N$ from training dataset\pause
    \item Grow a random-forest tree $T_b$ to fit the sample by RBP as follows until minimum node size is reached:\pause
      \begin{enumerate}[i.]
      \item Randomly select subset $S_{m}$ of the $D$ predictors at node $m$\pause
      \item Pick the best predictor/split-point $(j,t)$ from $S_{m}$ \pause
      \item Split node $m$ into two daughter nodes\pause
      \end{enumerate}
    \end{enumerate}
  \item Output the ensemble of trees $\{T_b\}_1^B$
  \end{enumerate}
  \end{block}
\end{frame}

% \begin{frame}
%   \frametitle{Random forests algorithm}
%   \pause
%   \begin{enumerate}[<+->]
%   \item For $b = 1$ to $B$: \pause
%     \begin{enumerate}[(a)]
%     \item Draw a bootstrap sample $\mc D^*$ of size $n$ from training dataset\pause
%     \item Grow a random-forest tree $T_b$ to fit the sample by recursive binary partitioning as follows until minimum node size is reached:\pause
%       \begin{enumerate}[i.]
%       \item Randomly select subset $S_{m}$ of the $D$ predictors at node $m$\pause
%       \item Pick the best predictor/split-point $(j,t)$ from $S_{m}$ \pause
%       \item Split node $m$ into two daughter nodes\pause
%       \end{enumerate}
%     \end{enumerate}
%   \item Output the ensemble of trees $\{T_b\}_1^B$
%   \end{enumerate}

%   \pe

% %   \pause
  
% %   \begin{block}{To make a prediction at point $x$:}
% %   \pause
% %   \begin{itemize}[<+->]
% %   \item {\it Regression:} \pause $\hat f_{rf}^B(x) = \fr1B\sum_{b=1}^B T_b(x)$
% %   \item {\it Classification:} \pause Let $\hat C_b(x)$ be the class prediction of the $b$th random-forest tree. \pause

% %     Then $\hat C_{rf}^B(x) = \text{majority vote} \{\hat C_b(x)\}_1^B$
% %   \end{itemize}
% % \end{block}

% \end{frame}

 
\begin{frame}
  \frametitle{Feature importance in random forests}
  \begin{minipage}[t]{.4\linewidth}
  \begin{itemize}[<+->]
  \item Since the feature subset in each step of partitioning is random in all the $B$ trees, the standard variable importance approach is not effective in random forests

  \item Instead, OOB samples are used to determine the variable importance
  \item This is done by perturbing each variable $j$ in the OOB samples and then averaging the decrease in accuracy across the trees
  \item Illustrated on the \texttt{spam} dataset:
  \end{itemize}
\end{minipage}
\pause\quad
\begin{minipage}[t]{.55\linewidth}
    \begin{figure}[h!]
      \centering
      \visible<6->{\includegraphics[width=.75\textwidth, trim={4.5cm 8cm 4cm 4.4cm}, clip]{ESL15-5}}
      \caption{\footnotesize Variable importance computed on a scale of 0-100
        from (Left) gradient boosted model (via Gini index) and (Right) random forest model (via OOB).
      Rankings are comparable, but relative values are not.}
  \end{figure}
\end{minipage}

\end{frame}


\section{Boosting}

\begin{frame}
  \frametitle{Boosting}\pause
  Generally, boosting refers to the approach of sequentially weighting the predictions of weak learners to obtain a
  final prediction
  \pause

 It can be considered as a method of fitting an {\bl additive model}:\pause
    \begin{equation}\bl
      f(\bm x;\bm \th) = \sum_{m=1}^M\beta_m b_{m}(\bm x; \bm\th_m)
    \end{equation}
    \pause
    where:
    \begin{itemize}
    \item $\beta_m$ are the expansion coefficients (weights) \pe
    \item $b_{m}(\bm x;\bm\th_{m})$ are elementary (simple) basis  functions of $\bm x$ characterized by a set of parameters $\bm\th_{m}$, e.g.
      \pe
      \begin{itemize}
      \item Shallow decision trees: $\bm\th_{m}$ parameterizes the split variable/split-point $(j, t)$ at the internal
        nodes and the predictions at the terminal nodes\pause
      \item Single-hidden-layer neural networks: $b(\bm x; \bm\th_{m}) = \bm \sigma(\bm w^T\bm x)$, where $\bm\sigma(t) = 1/(1+e^{-t})$ 
      \end{itemize}      
  \end{itemize}
\end{frame}


\begin{frame}
  \frametitle{Boosting (cont.)}
  \pause

  \begin{itemize}[<+->]
 
  \item To fit such additive models, we minimize the sum of the loss functions across the basis expansions. \pause

    \begin{equation}
      \min_{\{\beta_{m},\bm\th_{m}\}^{M}_{1}}\sum_{i=1}^{N}\ell\lt(y_{i},\sum_{m=1}^{M}\beta_{m}b(\bm x_{i};\bm\th_{m})\rt)
    \end{equation}
\pause

    \item Choices for $\ell$ include: squared error, likelihood-based, exponential (Adaboost)

           \item Fitting can be efficiently done via the \textit{forward stagewise additive modeling} algorithm\footnote{ESL
        10.1--10.5}
   \end{itemize}
\end{frame}

\begin{frame}
  \frametitle{Forward stagewise additive modeling}
  \pause
  \begin{enumerate}
  \item Initialize $f_{0}(\bm x) = 0$\pause

  \item For $m = 1$ to $M$:\pause
    \begin{enumerate}[(a)]
    \item Find:
      \begin{equation*}
        (\beta_{m},\bm\th_{m}) = \arg\min_{\beta,\bm\th}\sum_{i=1}^{M}\ell(y_{i},f_{m-1}(\bm x_{i}) + \beta b(\bm x_{i};\bm\th))
      \end{equation*}
      \pause
    \item Set $f_{m}(\bm x) = f_{m-1}(\bm x) + \beta_{m}b(\bm x;\bm\th_{m})$
    \end{enumerate}
  \end{enumerate}
  \pause
  \bigskip
  
  If $\ell$ is specified as squared error (typical for regression), then:\pause
  \begin{eqnarray}
    \ell(y,f(\bm x)) &=& (y - f(\bm x))^{2} \\\pause
    \ell(y_{i}, f_{m-1}(\bm x_{i}) + \beta b(\bm x_{i};\bm\th)) &=& \pause ({\bl y_{i} - f_{m-1}(\bm x_{i})} - \beta b(\bm x_{i};\bm\th))^{2}\\\pause
    &=& ({\bl r_{im}} - \beta b(\bm x_{i},\bm\th))^{2}
  \end{eqnarray}
  \pause
  where $\bl r_{im}$ is the \textbf{residual} of model $m$ on the $i$th observation.  
\end{frame}

\begin{frame}
  \frametitle{Boosting trees}
  \pause
  Boosting can be used as a ``slow learning'' approach for decision trees
\pause
  \begin{itemize}[<+->] 
  \item First we generate $B$ bootstrap samples
  \item Starting with an initial tree, we fit a subsequent tree to the residuals from the first model
  \item We then update the tree and its residuals using a fraction $\la$ of the predictions of the new tree
  \item The estimation and residual update step is repeated until the $B$th sample is used
  \item The boosted model is then given by the weighted sum of the fitted tree estimates
  \end{itemize}
\end{frame}

\begin{frame}
  \frametitle{Boosted regression tree algorithm}
  \pause
  \begin{enumerate}[<+->]
  \item Set $\hat f(\bm x) = 0$ and $r_i = y_i$ for all $i$ in training dataset
  \item For $b=1,2, \ldots, B$, repeat:\pause
    \begin{enumerate}[(a)]
    \item Fit a tree $\hat f^b$ with $d$ splits ($d+1$ terminal nodes) to training data $(\bm x,r)$\pause
    \item Update $\hat f$ by adding in a shrunken version of new tree:\pause
      \begin{equation}
        \hat f(\bm x) \leftarrow \hat f(\bm x) + \la \hat f^b(\bm x)
      \end{equation}\pause
    \item Update residuals:\pause
      \begin{equation}
        r_i \leftarrow r_i - \la \hat f^b(\bm x_i)
      \end{equation}
      \pause
    \end{enumerate}
  \item Output boosted model:\pause
    \begin{equation}
      \hat f(\bm x) = \sum_{b=1}^B \la \hat f^b(\bm x)
    \end{equation}
  \end{enumerate}
\end{frame}


\begin{frame}
  \frametitle{Notes on boosting trees}
  \pause
  \begin{itemize}[<+->]
  \item There are three tuning parameters to consider:\pause
    \begin{itemize}[<+->]
    \item Number of trees $B$. If $B$ is too large, model is prone to overfitting. We choose $B$ by cross-validation
      \medskip
      
    \item Shrinkage/learning parameter $\la > 0$ (typically 0.01 or 0.001)
      \medskip
      
    \item Number of splits $d$ in each tree (interaction depth); $d=1$ is a typical choice (in this case,
      the trees are referred to as ``stumps'')
      
    \end{itemize}
    \bigskip

    
  \item The sequential fitting of $\hat f(\bm x)$ is analogous to \textit{gradient descent} (since we are minimizing a loss function in each split)
    \begin{itemize}[<+->]
    \item Thus, $r_i$ can be updated via the gradient of the loss function
      \medskip
      
    \item Hence, the procedure is referred to as \textbf{\rd gradient boosting}

    \item The popular modeling package \texttt{XGBoost} (Extreme Gradient Boosting) is based on this idea but introduces regularization and second-order gradients
    %   \medskip

    % \item Implemented in R via the \texttt{gbm} package
    %   \medskip
      
    % \item Scikit-learn has a nice ensemble methods
    %   \href{https://scikit-learn.org/stable/modules/ensemble.html}{\bl\underline{documentation}}

    %   \medskip
      
    % \item A nice introduction to gradient boosting is available \href{http://www.ccs.neu.edu/home/vip/teach/MLcourse/4_boosting/slides/gradient_boosting.pdf}{\bl\underline{here}}
    \end{itemize}
  \end{itemize}
\end{frame}

\begin{frame}
  \frametitle{Boosting perfomance}
  \pause
  Boosting typically outperforms random forests and depth-1 trees perform better than other choices for $d$.
  \pause

  \visible<3->{\includegraphics[width=.7\textwidth]{8-11}}
\end{frame}

\section{Summary}
\begin{frame}
  \frametitle{Further issues and topics}
  \pause
  \begin{itemize}[<+->]

  \item  The methods described in this lecture are generally referred to as the CART implementation (classification and regression tree).
  \item Other approaches exist, e.g. PRIM (Patient Rule Induction Method) and ID3 (and its successors)
\end{itemize}

\pause

\begin{alertblock}{Potential issues with trees and mitigations}
  \begin{itemize}[<+->]
  \item \textbf{Multilevel categorical variables:} the greater the number of levels, the more prone to the tree is to overfitting
  \item \textbf{Missing data:} dropping incomplete observations can reduce accuracy; imputation methods can be used
  \item \textbf{Tree instability:} errors or changes to data can drastically affect results\\ \pause
    \textit{Bagging} can be used to address this (next lecture)
  \item \textbf{Lack of smoothness:} can be addressed using MARS (Multivariate Adaptive Regression Splines)
  \end{itemize}
\end{alertblock}
  
\end{frame}





\begin{frame}
  \frametitle{Reading}\pause
  
  \begin{itemize}[<+->]
  \item \textbf{PMLI} 18
\item \textbf{ESL} Section 9.2
\item ISLR 8.3
\item Bagging: ESL 8.7
\item Random Forests: ESL 15
\item Boosting: ESL 10 (Very dense chapter. You may want to focus on 10.1--10.3, 10.7, 10.10, 10.13, 10.14.)
\end{itemize}
\pause
\bigskip

Further reading/study:\pause
\begin{itemize}[<+->]
\item \textbf{PRIM}: ESL 9.3 and Friedman's paper at \url{http://statweb.stanford.edu/~jhf/ftp/prim.pdf}. \\
  
  PRIM/CART can be used in scenario discovery. For a brief overview, see Jan Kwakkel's blog \href{https://waterprogramming.wordpress.com/2015/08/05/scenario-discovery-in-python/}{\bl\bf post}.

\item \textbf{MARS}: ESL 9.4. This article provides a readily digestible overview: \url{http://uc-r.github.io/mars}.
\end{itemize}
 
 
\end{frame}

 
%\appendix
\end{document}

%%% Local Variables:
%%% mode: latex
%%% TeX-master: t
%%% End:
