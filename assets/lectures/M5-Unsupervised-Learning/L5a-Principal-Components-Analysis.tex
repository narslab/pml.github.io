%\documentclass[smaller, handout, dvipsnames]{beamer}
\def\bmode{0} % Mode 0 for presentation, mode 1 for a handout with notes, mode 2 fo% r handout without notes
\if 0\bmode
\immediate\write18{pdflatex -jobname=\jobname-Presentation\space\jobname}
\documentclass[usenames,dvipsnames,smaller]{beamer}
\else \if 1\bmode
\immediate\write18{pdflatex -jobname=\jobname-Handout-Notes\space\jobname}
\documentclass[usenames,dvipsnames,smaller,handout]{beamer}
\usepackage{handoutWithNotes}
\pgfpagesuselayout{2 on 1 with notes}[letterpaper, landscape, border shrink=4mm]
\else \if 2\bmode
\immediate\write18{pdflatex -jobname=\jobname-Handout\space\jobname}
\documentclass[usenames,dvipsnames,smaller,handout]{beamer}
\fi
\fi
\fi


%\setbeamertemplate{section in head/foot}{}
%\setbeamertemplate{section in head/foot shaded}{\textcolor{white}{\insertsectionhead}}
%%%%%%%%%%%%%%%%%%%%%%%%%%%%%%%%%%%%%%%%%%%%%%%%%%%%%%%%%%%%%%%%%%%%%%%%%%%%%%%%%%%%%%%%%%%%%
\newcommand{\coursetitle}{CEE 616: Probabilistic Machine Learning}
\newcommand{\longlecturetitle}{M5 Unsupervised Learning:\\ L5A:  Principal Components Analysis}
\newcommand{\shortlecturetitle}{L5A: PCA}
\newcommand{\instructor}{Jimi Oke}
\newcommand{\lecturedate}{Thu, Nov 20, 2025}
%%%%%%%%%%%%%%%%%%%%%%%%%%%%%%%%%%%%%%%%%%%%%%%%%%%%%%%%%%%%%%%%%%%%%%%%%%%%%%%%%%%%%%%%%%%%%


 
% \usepackage[T1]{fontenc} 
% \usepackage{lmodern} 
%\usepackage{etex}
 %\newcommand{\num}{6{} }

% \usetheme[
%   outer/progressbar=foot,
%   outer/numbering=fraction,
%   block=fill,
%   inner/subsectionpage=progressbar
% ]{metropolis}
\usetheme{Madrid}
\useoutertheme[subsection=false]{miniframes} % Alternatively: miniframes, infolines, split
\useinnertheme{circles}
% %\useoutertheme{Frankfurt}
% \usecolortheme{beaver}
% %\useoutertheme{crane}
% %\useoutertheme{metropolis}
\usepackage[backend=biber,style=authoryear,maxcitenames=2,maxbibnames=99,safeinputenc,url=false, eprint=false]{biblatex}
%\addbibresource{bib/references.bib}
% \AtEveryCitekey{\iffootnote{{\tiny}\tiny}{\tiny}}

% %\usepackage{pgfpages}
% %\setbeameroption{hide notes} % Only slides
% %\setbeameroption{show only notes} % Only notes
% %\setbeameroption{hide notes} % Only notes
% %\setbeameroption{show notes on second screen=right} % Both

% % \usepackage[sfdefault]{Fira Sans}

% % \setsansfont[BoldFont={Fira Sans}]{Fira Sans Light}
% % \setmonofont{Fira Mono}

% %\usepackage{fira}
% %\setsansfont{Fira}
% %\setmonofont{Fira Mono}
% % To give a presentation with the Skim reader (http://skim-app.sourceforge.net) on OSX so
% % that you see the notes on your laptop and the slides on the projector, do the following:
% % 
% % 1. Generate just the presentation (hide notes) and save to slides.pdf
% % 2. Generate onlt the notes (show only nodes) and save to notes.pdf
% % 3. With Skim open both slides.pdf and notes.pdf
% % 4. Click on slides.pdf to bring it to front.
% % 5. In Skim, under "View -> Presentation Option -> Synhcronized Noted Document"
% %    select notes.pdf.
% % 6. Now as you move around in slides.pdf the notes.pdf file will follow you.
% % 7. Arrange windows so that notes.pdf is in full screen mode on your laptop
% %    and slides.pdf is in presentation mode on the projector.

% % Give a slight yellow tint to the notes page
% \setbeamertemplate{note page}{\pagecolor{yellow!5}\insertnote}\usepackage{palatino}

% %\usetheme{metropolis}
% %\usecolortheme{beaver}
 \usepackage{tipa}
% \usepackage{enumerate}
\definecolor{darkcandyapplered}{HTML}{A40000}
\definecolor{lightcandyapplered}{HTML}{e74c3c}

% %\setbeamercolor{title}{fg=darkcandyapplered}

% \definecolor{UBCblue}{rgb}{0.04706, 0.13725, 0.26667} % UBC Blue (primary)
% \definecolor{UBCgrey}{rgb}{0.3686, 0.5255, 0.6235} % UBC Grey (secondary)

% % \setbeamercolor{palette primary}{bg=darkcandyapplered,fg=white}
% % \setbeamercolor{palette secondary}{bg=darkcandyapplered,fg=white}
% % \setbeamercolor{palette tertiary}{bg=darkcandyapplered,fg=white}
% % \setbeamercolor{palette quaternary}{bg=darkcandyapplered,fg=white}
% % \setbeamercolor{structure}{fg=darkcandyapplered} % itemize, enumerate, etc
% % \setbeamercolor{section in toc}{fg=darkcandyapplered} % TOC sections
% % \setbeamercolor{frametitle}{fg=darkcandyapplered,bg=white} % TOC sections
% % \setbeamercolor{title in head/foot}{bg=white,fg=white} % TOC sections
% % \setbeamercolor{button}{fg=darkcandyapplered} % TOC sections

% % % Override palette coloring with secondary
% % \setbeamercolor{subsection in head/foot}{bg=lightcandyapplered,fg=white}

%\usecolortheme{crane}
% \makeatletter
% \setbeamertemplate{headline}{%
%   \begin{beamercolorbox}[colsep=1.5pt]{upper separation line head}
%   \end{beamercolorbox}
%   \begin{beamercolorbox}{section in head/foot}
%     \vskip1pt\insertsectionnavigationhorizontal{\paperwidth}{}{}\vskip1pt
%   \end{beamercolorbox}%
%   \ifbeamer@theme@subsection%
%     \begin{beamercolorbox}[colsep=1.5pt]{middle separation line head}
%     \end{beamercolorbox}
%     \begin{beamercolorbox}[ht=2.5ex,dp=1.125ex,%
%       leftskip=.3cm,rightskip=.3cm plus1fil]{subsection in head/foot}
%       \usebeamerfont{subsection in head/foot}\insertsubsectionhead
%     \end{beamercolorbox}%
%   \fi%
%   \begin{beamercolorbox}[colsep=1.5pt]{lower separation line head}
%   \end{beamercolorbox}
% }
% \makeatother

% Reduce size of frame box
\setbeamertemplate{frametitle}{%
    \nointerlineskip%
    \begin{beamercolorbox}[wd=\paperwidth,ht=2.0ex,dp=0.6ex]{frametitle}
        \hspace*{1ex}\insertframetitle%
    \end{beamercolorbox}%
}


%\setbeamercolor{frametitle}{bg=darkcandyapplered!80!black!90!white}
%\setbeamertemplate{frametitle}{\bf\insertframetitle}

%\setbeamercolor{footnote mark}{fg=darkcandyapplered}
%\setbeamercolor{footnote}{fg=darkcandyapplered!70}
%\Raggedbottom
%\setbeamerfont{page number in head/foot}{size=\tiny}
%\usepackage[tracking]{microtype}


% %\usepackage[sc,osf]{mathpazo}   % With old-style figures and real smallcaps.
% %\linespread{1.025}              % Palatino leads a little more leading

% % Euler for math and numbers
% %\usepackage[euler-digits,small]{eulervm}
% %\AtBeginDocument{\renewcommand{\hbar}{\hslash}}
\usepackage{graphicx}
\usepackage{multirow}
\usepackage{booktabs}
\usepackage{graphbox}
\usepackage{animate}
\usepackage{media9}
\usepackage{adjustbox}

% %\mode<presentation> { \setbeamercovered{transparent} }

\setbeamertemplate{navigation symbols}{}
\makeatletter
\def\beamerorig@set@color{%
  \pdfliteral{\current@color}%
  \aftergroup\reset@color
}
\def\beamerorig@reset@color{\pdfliteral{\current@color}}
\makeatother


% %=== GRAPHICS PATH ===========
\graphicspath{{./m5-images/}}
% % Marginpar width
% %Marginpar width
% %\setlength{\marginparsep}{.02in}


% %% Captions
% % \usepackage{caption}
% % \captionsetup{
% %   labelsep=quad,
% %   justification=raggedright,
% %   labelfont=sc
% % }

% \setbeamerfont{caption}{size=\footnotesize}
% \setbeamercolor{caption name}{fg=darkcandyapplered}

% %AMS-TeX packages

\usepackage{amssymb}
\usepackage{amsmath}
\usepackage{amsthm}
\usepackage{mathtools} 
\usepackage{bm}
\DeclareMathOperator*{\argmax}{arg\,max}
\DeclareMathOperator*{\argmin}{arg\,min}
% \usepackage{color}

% %https://tex.stackexchange.com/a/31370/2269
\usepackage{cancel}
\renewcommand{\CancelColor}{\color{red}} %change cancel color to red
\makeatletter
\let\my@cancelto\cancelto %copy over the original cancelto command
\newcommand<>{\cancelto}[2]{\alt#3{\my@cancelto{#1}{#2}}{\mathrlap{#2}\phantom{\my@cancelto{#1}{#2}}}}
% redefine the cancelto command, using \phantom to assure that the
% result doesn't wiggle up and down with and without the arrow
\makeatother


% % \usepackage{comment}
\usepackage{enumerate}
\usepackage{hyperref}
% \usepackage{minitoc,array}
% \definecolor{slblue}{rgb}{0,.3,.62}
\hypersetup{
    colorlinks,%
    citecolor=blue,%
    filecolor=blue,%
    linkcolor=blue,
    urlcolor=blue
}

% \usepackage{epstopdf}
% \epstopdfDeclareGraphicsRule{.gif}{png}{.png}{convert gif:#1 png:\OutputFile}
% \AppendGraphicsExtensions{.gif}

% %\usepackage{listings}

% %%% TIKZ
\usepackage{forest}
\usepackage{tikz}
\usepackage{tikz-3dplot}
\usepackage{pgfplots}
\usepackage{pgfplotstable}
% \usepackage{pgfgantt}
\usepackage{neuralnetwork}

\usetikzlibrary{fit,arrows,arrows.meta,shapes,positioning,shapes.geometric}
\usetikzlibrary{decorations.markings}
\usetikzlibrary{shadows,automata}
\usetikzlibrary{patterns}
\usetikzlibrary{trees,mindmap,backgrounds}
%\usetikzlibrary{circuits.ee.IEC}
\usetikzlibrary{decorations.text}
% % For Sagnac Picture
% \usetikzlibrary{%
%     decorations.pathreplacing,%
%     decorations.pathmorphing%
% }
% \tikzset{no shadows/.style={general shadow/.style=}}
% %
% %\usepackage{paralist}

\tikzset{
  font=\Large\sffamily\bfseries,
  red arrow/.style={
    midway,red,sloped,fill, minimum height=3cm, single arrow, single arrow head extend=.5cm, single arrow head indent=.25cm,xscale=0.3,yscale=0.15,
    allow upside down
  },
  black arrow/.style 2 args={-stealth, shorten >=#1, shorten <=#2},
  black arrow/.default={1mm}{1mm},
  tree box/.style={draw, rounded corners, inner sep=1em},
  node box/.style={white, draw=black, text=black, rectangle, rounded corners},
}

% %%% FORMAT PYTHON CODE
% %\usepackage{listings}
% % Default fixed font does not support bold face
% \DeclareFixedFont{\ttb}{T1}{txtt}{bx}{n}{8} % for bold
% \DeclareFixedFont{\ttm}{T1}{txtt}{m}{n}{8}  % for normal

% % Custom colors
% \definecolor{deepblue}{rgb}{0,0,0.5}
% \definecolor{deepred}{rgb}{0.6,0,0}
% \definecolor{deepgreen}{rgb}{0,0.5,0}

% %\usepackage{animate}

% % Python style for highlighting
% % \newcommand\pythonstyle{\lstset{
% % language=Python,
% % basicstyle=\footnotesize\ttm,
% % otherkeywords={self},             % Add keywords here
% % keywordstyle=\footnotesize\ttb\color{deepblue},
% % emph={MyClass,__init__},          % Custom highlighting
% % emphstyle=\footnotesize\ttb\color{deepred},    % Custom highlighting style
% % stringstyle=\color{deepgreen},
% % frame=tb,                         % Any extra options here
%     % showstringspaces=false            % 
% % }}

% % % Python environment
% % \lstnewenvironment{python}[1][]
% % {
% % \pythonstyle
% % \lstset{#1}
% % }
% % {}

% % % Python for external files
% % \newcommand\pythonexternal[2][]{{
% % \pythonstyle
% % \lstinputlisting[#1]{#2}}}

% % Python for inline
% % 
% % \newcommand\pythoninline[1]{{\pythonstyle\lstinline!#1!}}

% %\usepackage{algorithm2e}

\newcommand{\eps}{\epsilon}
\newcommand{\bX}{\mb X}
\newcommand{\by}{\mb y}
\newcommand{\bbe}{\bm\beta}
\newcommand{\beps}{\bm\epsilon}
\newcommand{\bY}{\mb Y}

\newcommand{\osn}{\oldstylenums}
\newcommand{\dg}{^{\circ}}
\newcommand{\lt}{\left}
\newcommand{\rt}{\right}
\newcommand{\pt}{\phantom}
\newcommand{\tf}{\therefore}
\newcommand{\?}{\stackrel{?}{=}}
\newcommand{\fr}{\frac}
\newcommand{\dfr}{\dfrac}
\newcommand{\ul}{\underline}
\newcommand{\tn}{\tabularnewline}
\newcommand{\nl}{\newline}
\newcommand\relph[1]{\mathrel{\phantom{#1}}}
\newcommand{\cm}{\checkmark}
\newcommand{\ol}{\overline}
\newcommand{\rd}{\color{red}}
\newcommand{\bl}{\color{blue}}
\newcommand{\pl}{\color{purple}}
\newcommand{\og}{\color{orange!90!black}}
\newcommand{\gr}{\color{green!40!black}}
\newcommand{\lbl}{\color{CornflowerBlue}}
\newcommand{\dca}{\color{darkcandyapplered}}
\newcommand{\nin}{\noindent}
\newcommand*\circled[1]{\tikz[baseline=(char.base)]{
            \node[shape=circle,draw,thick,inner sep=1pt] (char) {\small #1};}}

\newcommand{\bc}{\begin{compactenum}[\quad--]}
\newcommand{\ec}{\end{compactenum}}

\newcommand{\p}{\partial}
\newcommand{\pd}[2]{\frac{\partial{#1}}{\partial{#2}}}
\newcommand{\dpd}[2]{\dfrac{\partial{#1}}{\partial{#2}}}
\newcommand{\pdd}[2]{\frac{\partial^2{#1}}{\partial{#2}^2}}
\newcommand{\pde}[3]{\frac{\partial^2{#1}}{\partial{#2}\partial{#3}}}
\newcommand{\nmfr}[3]{\Phi\left(\frac{{#1} - {#2}}{#3}\right)}
\newcommand{\Err}{\text{Err}}
\newcommand{\err}{\text{err}}

\DeclarePairedDelimiter\ceil{\lceil}{\rceil}
\DeclarePairedDelimiter\floor{\lfloor}{\rfloor}

%%%% GREEK LETTER SHORTCUTS %%%%%
\newcommand{\la}{\lambda}
\renewcommand{\th}{\theta}
\newcommand{\al}{\alpha}
\newcommand{\G}{\Gamma}
\newcommand{\si}{\sigma}
\newcommand{\Si}{\Sigma}


\pgfmathdeclarefunction{poiss}{1}{%
  \pgfmathparse{(#1^x)*exp(-#1)/(x!)}%
  }

\pgfmathdeclarefunction{gauss}{2}{%
  \pgfmathparse{1/(#2*sqrt(2*pi))*exp(-((x-#1)^2)/(2*#2^2))}%
}

\pgfmathdeclarefunction{expo}{2}{%
  \pgfmathparse{#1*exp(-#1*#2)}%
}

\pgfmathdeclarefunction{expocdf}{2}{%
  \pgfmathparse{1 -exp(-#1*#2)}%
}

\newcommand{\mb}{\mathbb}
\newcommand{\mc}{\mathcal}
\newcommand{\tr}{^{\top}}
\newcommand{\empt}[2]{$#1^{( #2 )}$}
\newcommand{\pe}{\pause}
% \usepackage{pst-plot}

% \usepackage{pstricks-add}
% \usepackage{auto-pst-pdf}   

% \psset{unit = 3}

% \def\target(#1,#2){%
%  {\psset{fillstyle = solid}
%   \rput(#1,#2){%
%     \pscircle[fillcolor = white](0.7,0.7){0.7}
%     \pscircle[fillcolor = blue!60](0.7,0.7){0.5}
%     \pscircle[fillcolor = white](0.7,0.7){0.3}
%     \pscircle[fillcolor = red!80](0.7,0.7){0.1}}}}
% \def\dots[#1](#2,#3){%
%     \psRandom[
%       dotsize = 2pt,
%       randomPoints = 25
%     ](!#2 #1 0.04 sub sub #3 #1 0.04 sub sub)%
%      (!#2 #1 0.04 sub add #3 #1 0.04 sub add)%
%      {\pscircle[linestyle = none](#2,#3){#1}}}


%%%%%%%%%%%%%%%%%%%%%%%%%%%%%%%%%%%%%%%%%%%%%%%%%%%
%%%%%%%%%%%%%%%%%%%%%%%%%%%%%%%%%%%%%%%%%%%%%%%%%%%
\title[\shortlecturetitle]{ {\normalsize \coursetitle}
  \\ \longlecturetitle}
\date[\lecturedate]{\footnotesize \lecturedate}
\author{{\bf \instructor}}
\institute[UMass Amherst]{
%\titlegraphic{\hfill
  \begin{tikzpicture}[baseline=(current bounding box.center)]
    \node[anchor=base] at (-7,0) (its) {\includegraphics[scale=.3]{UMassEngineering_vert}} ;
  \end{tikzpicture}
  % \hfill\includegraphics[height=1.5cm]{logo}
}

%https://tex.stackexchange.com/questions/55806/mindmap-tikzpicture-in-beamer-reveal-step-by-step
  \tikzset{
    invisible/.style={opacity=0},
    visible on/.style={alt={#1{}{invisible}}},
    alt/.code args={<#1>#2#3}{%
      \alt<#1>{\pgfkeysalso{#2}}{\pgfkeysalso{#3}} % \pgfkeysalso doesn't change the path
    },
  }


% https://tex.stackexchange.com/questions/446468/labels-with-arrows-for-an-equation
% https://tex.stackexchange.com/a/402466/121799
\newcommand{\tikzmark}[3][]{
\ifmmode
\tikz[remember picture,baseline=(#2.base)] \node [inner sep=0pt,#1](#2) {$#3$};
\else
\tikz[remember picture,baseline=(#2.base)] \node [inner sep=0pt,#1](#2) {#3};
\fi
}

% \lstset{language=matlab,
%                 basicstyle=\scriptsize\ttfamily,
%                 keywordstyle=\color{blue}\ttfamily,
%                 stringstyle=\color{blue}\ttfamily,
%                 commentstyle=\color{gray}\ttfamily,
%                 morecomment=[l][\color{gray}]{\#}
%               }


%%% Local Variables:
%%% mode: latex
%%% TeX-master: t
%%% End:

%\setbeamercolor{local structure}{fg=white}

\begin{document}
\maketitle
\begin{frame}
  \frametitle{Outline}
  \tableofcontents
\end{frame}


 

\section{Introduction}

\begin{frame}
  \frametitle{Unsupervised vs. supervised learning}

  \begin{center}
    \includegraphics<2->[width=.45\textwidth,trim={0 0 22cm 0}, clip]{supunsup}\quad
    \includegraphics<3->[width=.45\textwidth,trim={22cm 0 0 0}, clip]{supunsup}
  \end{center}

  \pause

  \begin{itemize}[<+->]
  \item Supervised learning: given response $y$ and $p$ features measured on the same observations, predict $y$ on the $x_j$
  \item Unsupervised learning: only $p$ features; no given response; what then can we learn about the data?
  \end{itemize}
\end{frame}


\begin{frame}
  \frametitle{Learning tools}
  \pause
  \begin{block}{Supervised}
  Goal: predict or infer a response (regression or classification)
  \begin{itemize}[<+->]
  \item multiple linear regression
  \item logistic regression
  \item linear/quadratic discriminant analysis
  \item decision trees
  \item support vector machines
  \end{itemize}
\end{block}
\pause

\begin{alertblock}{Unsupervised}
  Goal: exploration (e.g.\ grouping, pattern discovery, dimensional analysis)
  \begin{itemize}[<+->]
  \item dimensionality reduction
  \item clustering
  \end{itemize}
\end{alertblock}
\end{frame}

\begin{frame}
  \frametitle{Dimensionality reductoin}
  \pe
  Dimensionality reduction seeks to learn a suitable mapping from a high-dimensional feature space $\bm x \in\mb R^D$ to a low-dimensional \textbf{latent space} $\bm z\in \mb R^L$. \pe
  \begin{itemize}
  \item \textbf{Parametric approach:} estimate $\bm z = f(\bm x;\bm\th)$\pe
  \item \textbf{Nonparametric approach:} compute embedding $\bm z_n$ for each input $\bm x_n$\pe
  \item Uses:\pe
    \begin{itemize}
    \item data pre-processing
    \item model simplification
    \end{itemize}
    \pe
  \item Algorithms: \pe
    \begin{itemize}
    \item principal components analysis (PCA)\pe
    \item factor analysis (FA)\pe
    \item autoencoders
    \end{itemize}
  \end{itemize}
\end{frame}

\begin{frame}{Principal components analysis (PCA)}
  \pause
  PCA is a dimensionality reduction technique that seeks an $L$-dimensional basis that best captures the variance in a $D$-dimensional dataset
  \pause
    \begin{itemize}[<+->]
    \item The direction with the largest projected variance is the \textit{first principal component}
    \item The orthogonal direction capturing the second largest projected variance is the \textit{second principal component}
    \item The direction that maximizes the variance is that which also minimizes the mean squared error
    \end{itemize}
  \end{frame}
  

    
\section{Background}
% \begin{frame}
%   \frametitle{Eigenvectors and eigenvalues (review)}
%   A vector $\bm v$ is an eigenvector of a matrix $A$ if its transformation by $A$ scales rather than changes the direction of $\bm v$:\pause
%   \begin{equation}
%     \label{eq:5}
%     \bm A \bm v = \pause \la \bm v
%   \end{equation}\pause
%   where $\la$ is the \textbf{eigenvalue} (a scalar).
% \pause
%   \begin{tikzpicture}[]\small
%     \begin{axis}[no markers, domain=-40:0, %samples=100,
%       axis x line=center,
%       axis y line=center,
%       %xlabel=$z$, ylabel=$f_X(x)$,,
%       height=7cm, width=13cm,
%       % xtick={-6,0,6},
%       % xticklabels={$z_{\alpha/2}$,$0$,$z_{(1-\alpha/2)}$},
%       ymax=12,ymin=0,xmax=-0.5, xmin=-37,
%       % ytick=\empty,
%       % x label style={anchor=west},
%       % y label style={anchor=south},
%       %
%       enlargelimits=true, clip=false, %axis on top,
%       grid style={line width=.1pt, draw=gray},
%       % yticklabel style={
%       % /pgf/number format/fixed,
%       % /pgf/number format/fixed zerofill,
%       % /pgf/number format/precision=2
%       % },        
%       grid = both
%       ]
%       \visible<+->{
%         \draw[->, thick, blue] (axis cs: 0,0) -- (axis cs:  -0.95, 0.31) node[left] {$\bm v$};
%       \node[blue] (a) at (axis cs: -35, 7.5) {\small$\bm v =
%         \begin{pmatrix}
%           -0.95 \\ 0.31
%         \end{pmatrix}
%         $};        
%     }
%       \visible<+->{
%       \node[red,below=of a] (b) {\small$\bm A =
%         \begin{pmatrix}
%           111 & 222 \\
%           222 & 720
%         \end{pmatrix}
%         $} ;}
%     \visible<+->{
%       \draw[->, thick, dashed, purple] (axis cs:  0,0) -- (axis cs: -36.8, 11.98) node[above] {$\bm{Av}$};
%       \node[purple, right=of b] (c) {\small$\bm {Av} = \la \bm v =
%         \begin{pmatrix}
%           -36.80 \\ 11.98
%         \end{pmatrix}
%         $};
%     }
%   \end{axis}
%   \end{tikzpicture}
% \end{frame}


\begin{frame}
  \frametitle{Interpreting principal components}
  \pause
  The first principal component of a design/feature matrix $\bm X$ can be considered as the ``best-fit'' (closest) line to all the datapoints.

  \pause

  \visible<3->{  \begin{figure}[h!]
      \centering
      \includegraphics[width=.7\textwidth, trim = {2cm 10cm 2cm 7cm}, clip]{ESL14-20}
      \caption{First principal component (PC) of a dataset.
        The PC minimizes the total squared distance from each point to its orthogonal projection onto the line}
    \end{figure}}
\end{frame}

\begin{frame}
  \frametitle{Interpreting principal components (cont.)}
  \pause

  The first two principal components of a dataset span the [2D] plane closest to the data.

  \visible<3->{  \begin{figure}[h!]
      \centering
      \includegraphics[align=c,width=.27\textwidth, trim = {4.5cm 10cm 4.5cm 8cm}, clip]{ESL14-15}\quad\pause
      \includegraphics[align=c,width=.25\textwidth, trim = {2cm 11.8cm 11cm 7cm}, clip]{ESL14-21}\quad\pause
      \includegraphics[align=c,width=.3\textwidth, trim = {10.7cm 12cm 2cm 7cm}, clip]{ESL14-21}
      \caption{(L) Simulated dataset near surface of half-sphere. (C) Best 2-dimensional representationof data.
        (R) Projected points on the plane ($\bm U_2\bm \Gamma_2$) }
    \end{figure}}
  
\end{frame}

\begin{frame}
  \frametitle{Sample covariance matrix (review)}
  \pause
  The sample covariance matrix is given by:\pause
  \begin{equation}
    \label{eq:11}
    \widehat{\bm \Sigma} = E[(\bm X - \hat{\bm\mu})(\bm X - \hat{\bm\mu})^{T}] = \pause
    \begin{pmatrix}
      \hat\si_{1}^{2 } &       \hat\si_{12}& \cdots  &       \hat\si_{1D} \\[2mm] \pause
      \hat\si_{21} &       \hat\si_{2}^{2 }& \cdots  &       \hat\si_{2D} \\[2mm]\pause
      \cdots & \cdots & \cdots & \cdots \\[2mm]\pause
      \hat\si_{D1} &       \hat\si_{D2}& \cdots  &       \hat\si_{D}^{2 } 
    \end{pmatrix}
  \end{equation}
  \pause If $\bm X$ is mean centered, then we can write:\pause
  \begin{equation}
    \label{eq:12}
    \widehat{\bm \Sigma} = \fr1N(\bm X^{T}\bm X) = \fr1n
    \begin{pmatrix}
      \bm x_1^T \bm x_1 & \bm x_1^T \bm x_2  &  \cdots & \bm x_1^T \bm x_D  \\[2mm]\pause
      \bm x_2^T \bm x_1 & \bm x_2^T \bm x_2  &  \cdots & \bm x_2^T \bm x_D  \\[2mm]\pause
      \vdots   & \vdots    & \ddots & \vdots \\[2mm]\pause
      \bm x_D^T \bm x_1 & \bm x_D^T \bm x_2  &  \cdots & \bm x_D^T \bm x_D  
    \end{pmatrix}
  \end{equation}\pause
  The sample covariance matrix is given as the pairwise inner/dot products of the
  centered atrribute/feature vectors, normalized by the sample size $N$.
\end{frame}


\section{Max variance approach}
%   \begin{frame}

%   \frametitle{Orthonormal basis expansion}
%   Given an input data matrix (design matrix) $\bm X$:
%   \begin{itemize}[<+->]
%   \item Each point $x_1, \ldots, x_D$ is a column vector in $D$-dimensional vector space spanned by the $p$ standard
%     basis vectors $\bm e_1$, $\bm e_2, \ldots, \bm e_D$, where $\bm e_j$ corresponds to the $j$th attribute $\bm x_j$
%   \item The standard basis is an \textbf{\bl orthonormal} ({\bl pairwise orthogonal}) basis for the data space:
%     $\bm e_i^T\bm e_j = 0$ \pause and $||\bm e_i|| = 1$
%   \item Thus, given any other set of $D$ orthonormal vectors $\bm v_1, \bm v_2, \ldots, \bm v_D$ with
%     $\bm v_j^T\bm v_k = 0$ and $||\bm v_j|| = \bm v_j^T\bm v_j = 1$, we can write each $X_j$ as the linear combination
%     \pause
    
%     \begin{equation}
%       x_j = a_{j1}\bm v_1 + a_{j2}\bm v_2 + \cdots + a_{jp}\bm v_p = \pause \bm V\bm a_j
%     \end{equation}
%     \pause
%     \begin{itemize}[<+->]
%     \item where $\bm V$ is a $D\times D$ \textbf{orthogonal} matrix whose $j$th column comprises the $j$th basis vector $\bm v_j$
%     \item and the vector $\bm a_j$ represents the coordinates of $X_j$ in the new basis
%   \end{itemize}

%   \end{itemize}
% \end{frame}

% \begin{frame}
%   \frametitle{Transforming a dataset to a new basis}
%   \pause
%   Since $\bm V$ is orthogonal:
%   \begin{itemize}[<+->]
%   \item its columns are orthonormal \textit{basis vectors} of unit length:\pause
%     \begin{equation}
%       \bm v_j^T\bm v_k =
%       \begin{cases}
%         1 & \text{ if } j = k \\
%         0 & \text{ if } j \ne k
%       \end{cases}
%     \end{equation}
%   \item its inverse equals its transpose:\pause
%     \begin{equation}
%       \bm V^{-1} = \bm V^T
%     \end{equation}
%     \pause and
%     \begin{equation}
%       \bm V^T\bm V = \bm I_{D\times D}
%     \end{equation}
%     \pause
%     Finally, we can compute the coordinates of $X_j$ in the new basis as follows:\pause
%     \begin{align}
%       \bm V^T X_j &= \bm V^T\bm{Va}_j\\
%       \bm a_j &= \bm V^TX_j
%     \end{align}
% \end{itemize}
% \end{frame}


% \begin{frame}
%   \frametitle{Finding an optimal basis}
%   \pause
%   Potentially infinite choices exist for the set of orthonomal basis vectors

%   \pause

%   \begin{alertblock}{Questions and considerations}
%     \begin{itemize}[<+->]
%     \item Is there an optimal basis?
%     \item In a high dimensional dataset (large $D$), can we find a reduced dimensionality subspace (size $L$) that preserves essential characteristics of the data?
%     \item In this case, we might want to find an optimal representation of $\bm X$ with $L \ll D$
%     \end{itemize}
%   \end{alertblock}
%   \pause
%   Assuming the basis vectors were sorted in decreasing order of importance, we can truncate the linear expansion of $X_j$ in the new basis:\pause
%   \begin{equation}
%     \widetilde X_j = a_{j1}\bm v_1 + a_{j2}\bm v_2 + \cdots + a_{jM}\bm v_L = \sum_{m=1}^M a_{jm}\bm{v}_m
%   \end{equation}
% \end{frame}

\begin{frame}
  \frametitle{Projection of $\bm X$ onto first $L$ basis vectors}
  \pause

  The expression $\widetilde {\bm x}_j = \sum_{k=1}^L a_{jk}\bm v_{k}$ is a projection of
  $\bm x_j$ onto the first $L$ basis vectors.

  \bigskip
  \pause

  We derive a compact representation as follows:

  \begin{eqnarray*}
    \widetilde {\bm x_j}   &=& \bm V_L\bm a_{L} \\\pause
    \bm a_L &=& \bm V_L^T{\bm x_j} \\\pause
    \implies \quad \pause \widetilde {\bm x_j} &=& \bm V_L \bm V_L^T {\bm x_j} = \bm P_L {\bm x_j} 
  \end{eqnarray*}
  \pause
  where $\bl \bm P_L = \bm V_L\bm V_L^T$ is the orthogonal \textbf{\bl projection matrix} for the subspace spanned by the first $L$ basis vectors. \pause
  \begin{itemize}
  \item   We can compute the error vector as the projection of ${\bm x_j}$ onto the subspace spanned by the remaining basis vectors:\pause
  \begin{equation}
    \bm \epsilon_j = \sum_{k=L+1}^D a_{jk}\bm v_k = \pause {\bm x_j} - \widetilde {\bm x_j}
  \end{equation}
  \end{itemize}
\end{frame}

 
\begin{frame}
  \frametitle{Direction of max variance}
  \pause
  We seek the unit vector $\bm v$ that maximizes the projected variance of the points. \\
  \medskip
  \pause
  If $\bm X$ is centered and $\bm\Sigma$ its covariance matrix, then
  the {\gr projection of $X_j$ on $\bm v$} is:\pause
  \begin{equation}\gr
    X_j = \lt( \fr{\bm v^T \bm X_j}{\bm v^T\bm v}\rt) \bm v = (\bm v^T \bm X_j)\bm v = a_j\bm v
  \end{equation}
  Across all points, the \textbf{\rd projected variance} along $\bm v$ is:\pause
  \begin{equation}\rd
    \sigma_{\bm v}^2= \fr1n\sum_{j=1}^n (a_j - \mu_{\bm v})^2 = \fr1n\sum_{j}\bm v^T (X_j X_j^T)\bm v
    = \bm v^T\bm{\Sigma v}
  \end{equation}
  The optimal basis that maximizes the projected variance $\sigma_{\bm v}^2$ subject to $\bm v^T\bm v =1$ is:
  \begin{equation}
    \max_{\bm v}J(\bm v) =\pause  \bm v^T \bm{\Sigma v} - \lambda(\bm v^T \bm v - 1)
  \end{equation}
\end{frame}

\begin{frame}
  \frametitle{Direction of max variance (cont.)}
  \pause
  Taking the derivative of $J(\bm v)$ w.r.t. $\bm v$ and setting to zero, we obtain:\pause
  \begin{align*}
    \pd{(\bm v^T\bm\Sigma\bm v - \al(\bm v^T\bm v - 1))}{\bm v} &= \bm 0 \\
    \implies 2\bm\Sigma \bm v - 2\lambda \bm v &= \bm 0 \\\pause
    \bm{\Sigma\bm v} &= \bl \lambda \bm v
  \end{align*}
  \pause
  Thus $\lambda$ is an eigenvalue of $\bm\Sigma$ and $\bm v$ the eigenvector.\\ \pause
  \medskip
  
  Recall that the projected variance is given by $\sigma_{\bm v}^2 = \bm v^T{\bl \bm{\Sigma v}}$. Thus:
  \begin{equation}
    \sigma_{\bm v}^2 = \bm v^T {\bl \lambda \bm v} = \lambda
  \end{equation}
  \pause
  To maximize $\sigma_{\bm v}^2$ we set $\lambda$ to the largest eigenvalue $\la_1$ of $\bm \Sigma$; $\bm v_1$ indicates the direction of max variance (first principal component).
\end{frame}

\begin{frame}
  \frametitle{Iris dataset: first principal component}
  \begin{figure}[h!]
    \centering
    \includegraphics<2->[width=.4\textwidth,trim={0 .8cm 6cm 0},clip]{DMA7-1}\quad
    \includegraphics<3->[width=.34\textwidth]{DMA7-2}
    \caption{(Left) Iris dataset showing original basis: sepal length ($X_1$), sepal width ($X_2$) and petal length ($X_3$). (Right) First principal component $\bm u_1$ superimposed }
  \end{figure}
\end{frame}

\begin{frame}
  \frametitle{Two dimensions}
  \pause
  If we solve a similar optimization problem for two basis vectors $\bm u_1$ and $\bm u_2$, we obtain the first and second principal components whose total projected variance is $\la_1 + \la_2$.
  \pause

    \begin{figure}[h!]
    \centering
    \includegraphics<2->[width=.6\textwidth,trim={0 .3cm 0 0},clip]{DMA7-3}
    \caption{(Left) Optimal two-dimensional basis for Iris data. (Right) Non-optimal basis}
  \end{figure}
\end{frame}




\section{SVD approach}

\begin{frame}
  \frametitle{Singular value decomposition (SVD)}
  Recall the singular value decomposition of $\bm X$:

  \begin{equation}
    {\bl \bm X} = \bm {{\rd U}{\gr S} {\pl V}}^T 
  \end{equation}

  \pause

  where:
  \begin{itemize}[<+->]
  \item   $\bm X$ is an $N\times D$ data matrix, whose entries have been centered ($x_{nj} \leftarrow x_{nj} - \ol{x}_j$)
  \item $\bm U$ is an $N\times D$ orthogonal\footnote{i.e.\ $\bm U^T\bm U = \bm I$ and $\bm U^T = \bm U^{-1}$} matrix. The columns of $\bm U$ are called \textit{\pl left singular vectors}
  \item  $\bm S$ is a $D\times D$ diagonal matrix (whose elements are called \textit{\gr singular values})
  \item  $\bm V$ is an $D\times D$ orthogonal\footnote{i.e.\ $\bm V^T\bm V= \bm I$  and  $\bm V^T = \bm V^{-1}$} matrix. The columns of $\bm V$ are called \textit{\pl right singular vectors}
  \item The columns of $\bm{US}$ are called the \textbf{principal components} of $\bm X$.
\end{itemize}

\end{frame}

\begin{frame}
  \frametitle{SVD (cont.)}

  \begin{eqnarray}\pause
    \bl
  \overbrace{
    \begin{pmatrix}
      x_{11}   & \cdots & x_{1D} \\
      \vdots  & \ddots & \vdots \\
      x_{N1}   & \cdots & x_{ND} 
    \end{pmatrix}
  }^{\bm X}
  &=&\pause
{\rd  \overbrace{\begin{pmatrix}
      u_{11}  & \cdots & u_{1D} \\
       \vdots  & \ddots & \vdots \\
      u_{N1}   & \cdots & u_{ND}     
    \end{pmatrix}}^{\bm U: \text{ eigenvectors of } \bm{XX}^T}} \pause % & &
  {\gr \overbrace{
    \begin{pmatrix}
      \sqrt{\la_1}              & \cdots & 0 \\
       0                       & \ddots & 0 \\
       0                       & \cdots & \sqrt{\la_D}
    \end{pmatrix}
  }^{\substack{\bm S: \text{ $\sqrt{\text{eigenvalues}}$ of $\bm{XX}^T$ }  \\ \text{also singular values of } \bm X }}} \notag \\ \pause & &\qquad\qquad\qquad\qquad
{\pl \overbrace{\begin{pmatrix}
      v_{11}   & \cdots & v_{1D} \\
       \vdots  & \ddots & \vdots \\
      v_{D1}   & \cdots & v_{DD}     
    \end{pmatrix}}^{\bm V: \text{ eigenvectors of } \bm{X}^T\bm X}  
}  \end{eqnarray}

  \pause

  \begin{itemize}[<+->]
  \item The columns $\rd \bm u_1, \ldots, \bm u_D$ are the left singular vectors of $\bm X$
  \item  The columns $\pl \bm v_1, \ldots, \bm v_D$ are the right singular vectors of $\bm X$
  \item The elements $\gr \sqrt{\la_1} \ge \ldots \ge  \sqrt{\la_D} = 0$ are the singular values of $\bm X$
  \item $\la_1\ge  \ldots \ge \la_D =0$ are the eigenvalues of $\bm {XX}^T$ and also of $\bm X^T \bm X$
  \end{itemize}
\end{frame}



\begin{frame}
  \frametitle{PCA via SVD}
  \pause

  \begin{itemize}[<+->]
  % \item  The goal of PCA is to find a low-dimensional representation of a dataset that retains a reasonable amount of information (i.e.\ variance)

  \item   In the SVD framework, this means we find the best number $L$ of principal components $\bm u_ks_k$, where $k = 1, \ldots, L, L + 1, \ldots,  D$.\footnote{Note that $s_k = \sqrt{\la_k}$ in our notation.}

  \item The transformed (reduced)  dataset is given by:
    \begin{equation}
      {\bl \bm Z} = \pause \bm U_L \bm S_{L} = \pause \bm X \bm V_L
    \end{equation}
    where $\bl \bm Z\in\mb R^{N\times L}$ is the \textbf{\bl score matrix} and $\bm U_L$, $\bm S_L$ and $\bm V_L$ are the $L$-truncated matrix components of the SVD of $\bm X$\pe
    \begin{itemize}
    \item $\bm V_L$ is also referred to as the \textbf{weight matrix} $\bm W$\pe
    \end{itemize}
  \item The data matrix $\bm X$ can be approximately recovered from the transformation by:\pause
    \begin{equation}
      \widetilde{\bm X} = \pause \bm{Z} {\rd \bm V^T_L}
    \end{equation}
    where $\rd \bm V_L^T\in\mb R^{L\times D}$ ({\rd \bf loadings matrix}) is the transpose of the first $L$ columns of $\bm V$\pe
  \item Thus, PCA is considered the $L$-truncated SVD approximation of $\bm X$: \pe
    \begin{equation}
      \widetilde{\bm X}  = \bm U_L\bm S_L \bm V_L\tr
    \end{equation}
\end{itemize}
\end{frame}

\begin{frame}
  \frametitle{PCA via SVD (cont.)}
  \pause
  Since $\bm Z = \bm {XV}_L$, we can also recover $\bm X$ by:\pause
  \begin{equation}
    \widetilde{\bm X} = \pause \bm{ZV}^T_L = \bm {X} {\gr \bm V_L\bm V^T_L}
  \end{equation}
  \pause
  The matrix ${\gr \bm V_L\bm V^T_L}$ is called the \textbf{\gr projection matrix}.
  \pause

  
\visible<5->{    \begin{figure}[h!]
      \centering
      \includegraphics[width=.9\textwidth]{pcex}
      \caption{1D projection of dataset onto first PC and reconstruction}
    \end{figure}}
  
  \begin{itemize}[<+->] 
  \item When $L=D$, then ${\gr \bm V_L\bm V^T_L} = \bm I_D$ ($D\times D$ identity matrix) \pause and $\bm X$ is recovered exactly
  \item A great illustration can be found \href{https://stats.stackexchange.com/a/140579/67841}{\bf\bl here}.
\end{itemize}

\end{frame}

\begin{frame}
  \frametitle{Proportion of variance explained}
  \pause

  The total variance present in the dataset (mean-centered) is given by:\pause
  \begin{equation}
    \sum_{j=1}^D \mb V(\bm x_j) = \sum_{j=1}^D\la_j
  \end{equation}
  \pause
  That is, the eigenvalues of the covariance matrix $\bm X^T\bm X$ sum up to the total variance. \pause
  Since  $\bm X^T\bm X$ is positive semidefinite, its eigenvalues are non-negative:\pause
  \begin{equation}
    \la_1 \ge \la_2 \cdots \la_L \ge \la_{L+1}\cdots\ge \la_D \ge 0
  \end{equation} \pause  
  The total projected variance in the $L$-dimensional subspace is given by:\pause
  \begin{equation}
    \mb V( \widetilde{\bm X} ) = \sum_{j=1}^L\la_j
  \end{equation}
  \pause
  The {\bf \bl proportion of variance explained} by the $j$th PC is then given by:\pause
  \begin{equation}\bl
    PVE = \fr{\la_j}{\sum_{j=1}^D\la_j}
  \end{equation}\pause
  \begin{itemize}
  \item $\sqrt{\la_j}$ are the diagonal (non-zero) elements of the singular value matrix $\bm S$
  \end{itemize}
\end{frame}

\begin{frame}
  \frametitle{Selecting the ``best''  $L$-dimensional approximation}
  \pause
  For a given number of dimensions $L$, the {\pl cumulative PVE} is given by: \pause
  \begin{equation}\pl
    CVPE_L =\pause \fr{\sum_{j=1}^L\la_j}{\sum_{j'=1}^D\la_{j'}}
  \end{equation}
  \pause
  
  We choose $M$ (number of PCs) such that the CVPE  is greater than a reasonably large threshold:\pause
    \begin{eqnarray}\pl 
      L^{*} &=& \min L \\\pause
      \text{s.t.} && CVPE_L \ge \tau
    \end{eqnarray}
    \pause
    where $\tau$ is the desired threshold (e.g. 0.9)\pause
    
    \begin{itemize}
    \item This can also be visualized using a \textbf{scree plot}
  \end{itemize}
 % \begin{alertblock}{Selecting $M$}
 %   In order to choose $M$, we can compute the proportion of total variance explained by the first $M$ principal components:\pause
    % \begin{equation}
    %   f(M) = \fr{\sum_{m=1}^M \la_m}{\sum_{m=1}^p \la_m} = \fr{\sum_{m=1}^M \la_m}{\sum_{j=1}^p \si_j^2}
    %   = \fr{\sum_{m=1}^M \la_m}{\mb V(\bm X)}
    % \end{equation}
    \pause

 % \end{alertblock}
\end{frame}

\section{PCR and PLS}
\begin{frame}
  \frametitle{Dimensionality reduction for regression}
  \pause

  \begin{itemize}
  \item Previous methods to control variance:
    \begin{itemize}
    \item Subset selection
    \item Coefficient shrinkage
    \end{itemize}
  \item All used original predictors in dataset $\bm x_{1}, \bm x_{2}, \ldots, \bm x_{D}$.
  \item We can also improve a fit by training a model on a transformation of the input space: $\bm z_{1}, \bm z_{2}, \ldots, \bm z_{L}$:
    \begin{equation}
      \label{eq:9}
      \bm z_{j} = \sum_{j=1}^L\bm X v_{j}, \pause\quad L < D
    \end{equation}
    \pause
    \begin{itemize}
    \item if $L << D$, variance of coefficients can be signficantly reduced\pause
    \item The estimation problem is thus reduced from estimating $D+1$ coefficients to $L+1$ coefficients
    \end{itemize}\pause
  \item Consider \textbf{principal components analysis (PCA)} as an approach for regression\pause
    \begin{itemize}
    \item In selecting the number of principal components as regressors, we can use cross-validation to choose the
      $L$ which gives the lowest error estimate.
  \end{itemize}
\end{itemize}
\end{frame}


\begin{frame}
  \frametitle{Principal components regression (PCR)}
  Let the columns $\bm z_{k}$ be the linear combinations (principal components) of the original inputs $X_{j}$ (or $\bm x_{j}$). \\ \pause
  \medskip
  
  In PCR, we regress the response $\bm y$ onto the subspace spanned by $\bm z_{k} = \bm X v_k$, where
  $L \le D$ and $\bm z_{k}$ are the principal components of $\bm X$:
  \begin{equation}
    \label{eq:8}
    \hat{\bm{y}}^{pcr}_{(L)} = \ol y\bm 1 + \sum_{j=1}^{L}\hat\theta_{j}\bm{z}_{j}
  \end{equation}
  \pause
  The coordinates of $\widetilde{\bm x}_{j}$ in the new $L$-dimensional basis are then given by:\\ \pause
  \begin{equation}
    \label{eq:20}
    \bm z_{j} = \bm V^{T}_{L} \bm x_{j}
  \end{equation}
  and the estimates are:
  \pause
  \begin{equation}
    \label{eq:21}
    \hat\theta_{j} = \fr{\bm z_{j}^{T}\bm y}{\bm z_{j}^{T}\bm z_{j}}
  \end{equation}
  \pause
  We can then express the solution in terms of PCR coefficients of $\bm x_{j}$:\pause
  \begin{equation}
    \label{eq:22}
    \hat{\bm{y}}^{pcr}_{(L)} = \ol y\bm 1 + \bm X \hat{\bm w}^{pcr} 
%    \ol y\bm 1 +  \sum_{m=1}^{M}\hat\theta_{m}
  \end{equation}
\end{frame}


\begin{frame}
  \frametitle{Partial least squares regression}
  \pause
  This is a supervised and iterative form of PCR in which the construction of $\bm z_{j}$ is informed by the correlation of each $\bm x_{j}$ with $\bm y$.
  \pause
  \begin{figure}[h!]
    \centering
    \includegraphics<3->[width=.6\textwidth]{6-21}
    \caption{An example showing the first PLS direction (solid line) and first PCR direction (dotted line)}
  \end{figure}
\end{frame}



\section{Summary}
\begin{frame}
  \frametitle{Summary of PCA steps}
  \pause
  \begin{itemize}[<+->]
  \item Perform singular value decomposition of $N\times D$ data matrix $\bm X$:
      \begin{equation}
    {\bl \bm X} = \bm {{\rd U}{\gr S} {\pl V}}^T 
  \end{equation}
\item Determine the number of principal components $M$ to extract/retain using the cumulative proportion of variance explained:\pause
    \begin{equation}
    CVPE_L =\pause \fr{\sum_{j=1}^L\la_j}{\sum_{j'=1}^D\la_{j'}}
  \end{equation}\pause
\item Get loadings matrix $\bm V_{L}^{T}$ by truncating $\bm V^{T}$
 
\item Find score matrix $\bm Z = \bm X \bm V_{L} \pe = \bm X\bm W$ \pe (transformed data into reduced subspace). \pe
   Use $ \bm Z$ for regression, clustering, etc
 \item Approximation of original data matrix can be obtained via:
   \begin{equation}
    \widetilde{\bm X} = \pause \bm{Z} {\rd \bm V^T_L} \pe = \bm Z\bm W\tr
  \end{equation}
   \end{itemize}
\end{frame}

\begin{frame}[fragile]
  \frametitle{Outlook}
  \pause

  \begin{block}{Key points}\pause
      \begin{itemize}[<+->]
      \item PCA is a technique used for reducing the dimensionality of a dataset and exploring underlying patterns in the
        variables
      \item PCA identifies a low-dimensional subspace that captures the largest fraction of the input data variance
      \item Standardizing (mean centering and scaling by standard deviation) is desired to ensure that variances are not
        dominated by features on a larger scale \end{itemize}
\end{block}
\pause

\begin{exampleblock}{Reading}
  \begin{itemize}[<+->]
  \item  \textbf{PMLI} 20.1
  \item \textbf{ESL} 14.5 (note that in the book $\bm D$ corresponds to the $\bm S$ used in this lecture)
  \item \textbf{PMLCE} 10.2 %Problem Set 3, Problem 2 preamble 
  % \item PCA tutorial: \url{https://www.cs.princeton.edu/picasso/mats/PCA-Tutorial-Intuition_jp.pdf}
  \end{itemize}
\end{exampleblock}
\end{frame}



\appendix\addtocounter{part}{-1}

\section{Appendix: PCs and ridge regression}
% \begin{frame}
%   \frametitle{Singular value decomposition approach}
%   We motivate principal components through the singular value decomposition (SVD) of $\bm X^T\bm X$:
%   \begin{equation}
%     \label{eq:0}
%     \bm S = \fr1n \bm X^{T}\bm X
%   \end{equation}
%   \pause
%   where $\bm S$ is the sample covariance matrix and $\bm X$ is the mean centered $p\times p$ input matrix.
% \end{frame}
\begin{frame}
  \frametitle{Ridge estimates}
  Recall the ridge regression estimate:\pause
  \begin{equation}
    \label{eq:1}
    \hat{\bm w}^{R} = (\bm X^{T}\bm X + \la \bm I)^{-1} \bm X^{T}\bm y
  \end{equation}
  The \textbf{\bl singular value decomposition} of $\bm X$ can yield important insights into the nature of the solution:
  \begin{equation}\bl
    \bm X = \bm{UDV}^{T}\quad  \label{eq:2}    
  \end{equation}
  where $\bm U_{N\times D}$   and $\bm V_{D \times D}$ are orthogonal matrices. Recall that an orthogonal matrix is one whose columns/rows are orthogonal unit vectors (i.e. all rows and columns have only one non-zero element: $\pm 1$); $\bm U^{T}\bm U =  \bm I$
  \begin{align*}
    \intertext{$\bm D$ is a $D\times D$ diagonal matrix; $d_{j}\ge 0$}
  \pause \hat{\bm y} =  \bm X\hat{\bm w}^{OLS} &= \bm X(\bm X ^{T}\bm X)^{-1}\bm X^{T}\bm y \\
                         &= \bm{USDV}^{T} (\bm{VDU}^{T} \bm{USV}^{T} )^{-1} \bm{VSU}^{T}\bm{y}\\
                         &=\bm{U}(\bm S^{2})^{-1}\bm{S}^{2}\bm U^{T}\bm y\\
                         &= \bm{UU}^{T}\bm y
  \end{align*}
\end{frame}

\begin{frame}
  \frametitle{Ridge estimate decomposition}
  We can then write the ridge solutions as:
  \begin{align*}
    \bm X \hat{\bm w}^{R} &= \bm X(\bm X^{T}\bm X + \la\bm I)^{-1}\bm X^{T}\bm y\\
    &= \bm{U S} (\bm{S}^{2} + \la\bm{I})^{-1}\bm{S U}^{T}\bm{y}\\
    &= \sum_{j=1}^{p}\bm{u}_{j}\fr{d_{j}^{2}}{d_{j}^{2}+\la}\bm{u}_{j}^{T}\bm{y}
  \end{align*}
  where $\bm{u}_{j}$ are the columns of $\bm U$.

  \bigskip
  Thus, we see that ridge regression shrinks the coordinates of $\bm y$ in the basis $\bm U$ by $\fr{d_{j}^{2}}{d_{j}^{2}+ \la}$.
  \begin{itemize}
  \item As $d_{j}$ decreases, the term $\fr{d_{j}^{2}}{d_{j}^{2}+ \la}$ increases.
  \item Thus, more shrinkage is applied to the coordinates whose basis vectors correspond to smaller $d_{j}$.
  \end{itemize}
\end{frame}

\begin{frame}
  \frametitle{Principal components}
  Keeping in mind that $\bm X$ is a centered matrix, then the sample covariance matrix is given by:
  \begin{equation}
    \label{eq:3}
    \bm S = \fr{\bm X^{T}\bm X}{N}
  \end{equation}
  Substituting $\bm X$ with its SVD we obtain:
  \begin{equation}
    \label{eq:4}
    \bm X^{T}\bm X = (\bm{UDV}^{T})^{T}\bm{UDV}^{T} = \bm{VDU}^{T}\bm{UDV}^{T} = \bm V\bm D^{2}\bm V^{T}
  \end{equation}
  \begin{itemize}
  \item The columns $\bm v_{j}$ of $\bm V$ are the \textbf{eigenvectors} of $\bm X$ (or \textbf{principal components}).
  \item The expression $\bm V\bm D^{2}\bm V^{T}$ is called the \textbf{eigendecomposition} of $\bm S$.
\end{itemize}

\end{frame}

\begin{frame}
  \frametitle{First principal component}
  Given the eigen decomposition:
  \begin{equation}
    \label{eq:6}
    \bm{X}^{T}\bm{X} = \bm{V S}^{2}\bm V^{T}
  \end{equation}
  The first principal component\footnote{Also known as Karhunen-Loeve direction} of $\bm X$ satisfies the property:
  \begin{equation}
    \label{eq:7}
    \mb V(\bm z_{1})  = \mb V(\bm X \bm v_{1}) = \fr{s_{1}^{2}}{N}  = \fr{\lambda_{1}}{N}
  \end{equation}
  % Generally, the principal components $\bm z_{j}$ have a maximum variance of $\fr {d_{j}^{2}}n$.
  \pause
  \begin{itemize}[<+->]
  \item The variable $\bm z_{1}$ is the \textbf{first principal component} of $\bm X$:\pause
    \begin{equation}
      \label{eq:10}
      \bm z_{1} = \bm X v_{1} \pause = \bm u_{1}s_{1}
    \end{equation}
  where the vector $\bm u_{1}$ is the normalized first principal component.
  \item The last principal component has minimum variance.
  \item  Since this corresponds to the lowest $s_{k}$, this corresponds to the direction shrunk the most by the ridge regression
  \end{itemize}
\end{frame}

\begin{frame}
  \frametitle{Principal components --- 2 dimensions}
  \pause
  
  Ridge regression projects $\bm y$ onto the principal components, shrinking the coefficient of the low-variance component more than the high-variance component.

  \begin{figure}[h!]
    \centering
    \includegraphics<3->[width=.6\textwidth,trim=2cm 12cm 2cm 4cm, clip]{ESL3-9}
    \caption{Principal components of a two-dimensional input dataset. The largest principal component (PC) maximizes the variance of the projected data. The smallest PC minimizes that variance.}
  \end{figure}
  
 
\end{frame}

% \begin{frame}
%   \frametitle{Further reading}
%   \pause

%   \begin{itemize}[<+->]
%   \item For a derivation of the ordinary least squares (OLS) estimates and relevant statistics,
%     see ESL pp.\ 11--12, 43--50
%   \item A good introduction to weighted least squares (WLS) and two-stage estimation: \url{https://ms.mcmaster.ca/canty/teaching/stat3a03/Lectures7.pdf}
%   \end{itemize}
% \end{frame}

%  \begin{frame}
%   \frametitle{Iteratively reweighted least squares}\pause
%   The update for the logistic regression coefficient is given by:
%   \begin{eqnarray}
%     \label{eq:34}
%     \begin{split}
%       \bm w_{k+1} 
%         &=&&&(\bm X^T \bm W \bm X)^{-1} \bm X^T \bm W {\gr \bm z}
%     \end{split}
%   \end{eqnarray}
%   \pause
%   where the adjusted response $\bm z$ is given as:
%   \begin{equation}
%     \label{eq:35}
%     \gr\bm z = \bm X\bm w_k + \bm W^{-1} (\bm y - \bm p)
%   \end{equation}
%   \begin{alertblock}{OLS, WLS and IRLS}\pause
%     \begin{itemize}
%     \item  Recall the OLS estimate: $\hat{\bm w} = (\bm X^T\bm X)^{-1}\bm X^T\bm z $ \pause if $\bm z$ is the response.\pause

%     \item The weighted least squares (WLS) is given by: $\hat{\bm w} = (\bm X^T\bm W \bm  X)^{-1}\bm X^T \bm W z $. \pause

%     \item In the logistic regression case, $\bm W$ and $\bm z$ change in each step, \pause hence it is called
%       \textit{iteratively reweighted least squares (IRLS)}.
%   \end{itemize}
%   \end{alertblock}

% \end{frame}
\end{document}

%%% Local Variables:
%%% mode: latex
%%% TeX-master: t
%%% End:
