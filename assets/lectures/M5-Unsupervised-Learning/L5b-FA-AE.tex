%\documentclass[smaller,handout, dvipsnames]{beamer}
\def\bmode{0} % Mode 0 for presentation, mode 1 for a handout with notes, mode 2 fo% r handout without notes
\if 0\bmode
\immediate\write18{pdflatex -jobname=\jobname-Presentation\space\jobname}
\documentclass[usenames,dvipsnames,smaller]{beamer}
\else \if 1\bmode
\immediate\write18{pdflatex -jobname=\jobname-Handout-Notes\space\jobname}
\documentclass[usenames,dvipsnames,smaller,handout]{beamer}
\usepackage{handoutWithNotes}
\pgfpagesuselayout{2 on 1 with notes}[letterpaper, landscape, border shrink=4mm]
\else \if 2\bmode
\immediate\write18{pdflatex -jobname=\jobname-Handout\space\jobname}
\documentclass[usenames,dvipsnames,smaller,handout]{beamer}
\fi
\fi
\fi


%\setbeamertemplate{section in head/foot}{}
%\setbeamertemplate{section in head/foot shaded}{\textcolor{white}{\insertsectionhead}}
%%%%%%%%%%%%%%%%%%%%%%%%%%%%%%%%%%%%%%%%%%%%%%%%%%%%%%%%%%%%%%%%%%%%%%%%%%%%%%%%%%%%%%%%%%%%%
\newcommand{\coursetitle}{CEE 616: Probabilistic Machine Learning}
\newcommand{\longlecturetitle}{M5 Unsupervised Learning:\\ 5B:  Factor Analysis and Autoencoders}
\newcommand{\shortlecturetitle}{5B: Factor Analysis and Autoencoders}
\newcommand{\instructor}{Jimi Oke}
\newcommand{\lecturedate}{Dec 4, 2025}
%%%%%%%%%%%%%%%%%%%%%%%%%%%%%%%%%%%%%%%%%%%%%%%%%%%%%%%%%%%%%%%%%%%%%%%%%%%%%%%%%%%%%%%%%%%%%


 
% \usepackage[T1]{fontenc} 
% \usepackage{lmodern} 
%\usepackage{etex}
 %\newcommand{\num}{6{} }

% \usetheme[
%   outer/progressbar=foot,
%   outer/numbering=fraction,
%   block=fill,
%   inner/subsectionpage=progressbar
% ]{metropolis}
\usetheme{Madrid}
\useoutertheme[subsection=false]{miniframes} % Alternatively: miniframes, infolines, split
\useinnertheme{circles}
% %\useoutertheme{Frankfurt}
% \usecolortheme{beaver}
% %\useoutertheme{crane}
% %\useoutertheme{metropolis}
\usepackage[backend=biber,style=authoryear,maxcitenames=2,maxbibnames=99,safeinputenc,url=false, eprint=false]{biblatex}
%\addbibresource{bib/references.bib}
% \AtEveryCitekey{\iffootnote{{\tiny}\tiny}{\tiny}}

% %\usepackage{pgfpages}
% %\setbeameroption{hide notes} % Only slides
% %\setbeameroption{show only notes} % Only notes
% %\setbeameroption{hide notes} % Only notes
% %\setbeameroption{show notes on second screen=right} % Both

% % \usepackage[sfdefault]{Fira Sans}

% % \setsansfont[BoldFont={Fira Sans}]{Fira Sans Light}
% % \setmonofont{Fira Mono}

% %\usepackage{fira}
% %\setsansfont{Fira}
% %\setmonofont{Fira Mono}
% % To give a presentation with the Skim reader (http://skim-app.sourceforge.net) on OSX so
% % that you see the notes on your laptop and the slides on the projector, do the following:
% % 
% % 1. Generate just the presentation (hide notes) and save to slides.pdf
% % 2. Generate onlt the notes (show only nodes) and save to notes.pdf
% % 3. With Skim open both slides.pdf and notes.pdf
% % 4. Click on slides.pdf to bring it to front.
% % 5. In Skim, under "View -> Presentation Option -> Synhcronized Noted Document"
% %    select notes.pdf.
% % 6. Now as you move around in slides.pdf the notes.pdf file will follow you.
% % 7. Arrange windows so that notes.pdf is in full screen mode on your laptop
% %    and slides.pdf is in presentation mode on the projector.

% % Give a slight yellow tint to the notes page
% \setbeamertemplate{note page}{\pagecolor{yellow!5}\insertnote}\usepackage{palatino}

% %\usetheme{metropolis}
% %\usecolortheme{beaver}
 \usepackage{tipa}
% \usepackage{enumerate}
\definecolor{darkcandyapplered}{HTML}{A40000}
\definecolor{lightcandyapplered}{HTML}{e74c3c}

% %\setbeamercolor{title}{fg=darkcandyapplered}

% \definecolor{UBCblue}{rgb}{0.04706, 0.13725, 0.26667} % UBC Blue (primary)
% \definecolor{UBCgrey}{rgb}{0.3686, 0.5255, 0.6235} % UBC Grey (secondary)

% % \setbeamercolor{palette primary}{bg=darkcandyapplered,fg=white}
% % \setbeamercolor{palette secondary}{bg=darkcandyapplered,fg=white}
% % \setbeamercolor{palette tertiary}{bg=darkcandyapplered,fg=white}
% % \setbeamercolor{palette quaternary}{bg=darkcandyapplered,fg=white}
% % \setbeamercolor{structure}{fg=darkcandyapplered} % itemize, enumerate, etc
% % \setbeamercolor{section in toc}{fg=darkcandyapplered} % TOC sections
% % \setbeamercolor{frametitle}{fg=darkcandyapplered,bg=white} % TOC sections
% % \setbeamercolor{title in head/foot}{bg=white,fg=white} % TOC sections
% % \setbeamercolor{button}{fg=darkcandyapplered} % TOC sections

% % % Override palette coloring with secondary
% % \setbeamercolor{subsection in head/foot}{bg=lightcandyapplered,fg=white}

%\usecolortheme{crane}
% \makeatletter
% \setbeamertemplate{headline}{%
%   \begin{beamercolorbox}[colsep=1.5pt]{upper separation line head}
%   \end{beamercolorbox}
%   \begin{beamercolorbox}{section in head/foot}
%     \vskip1pt\insertsectionnavigationhorizontal{\paperwidth}{}{}\vskip1pt
%   \end{beamercolorbox}%
%   \ifbeamer@theme@subsection%
%     \begin{beamercolorbox}[colsep=1.5pt]{middle separation line head}
%     \end{beamercolorbox}
%     \begin{beamercolorbox}[ht=2.5ex,dp=1.125ex,%
%       leftskip=.3cm,rightskip=.3cm plus1fil]{subsection in head/foot}
%       \usebeamerfont{subsection in head/foot}\insertsubsectionhead
%     \end{beamercolorbox}%
%   \fi%
%   \begin{beamercolorbox}[colsep=1.5pt]{lower separation line head}
%   \end{beamercolorbox}
% }
% \makeatother

% Reduce size of frame box
\setbeamertemplate{frametitle}{%
    \nointerlineskip%
    \begin{beamercolorbox}[wd=\paperwidth,ht=2.0ex,dp=0.6ex]{frametitle}
        \hspace*{1ex}\insertframetitle%
    \end{beamercolorbox}%
}


%\setbeamercolor{frametitle}{bg=darkcandyapplered!80!black!90!white}
%\setbeamertemplate{frametitle}{\bf\insertframetitle}

%\setbeamercolor{footnote mark}{fg=darkcandyapplered}
%\setbeamercolor{footnote}{fg=darkcandyapplered!70}
%\Raggedbottom
%\setbeamerfont{page number in head/foot}{size=\tiny}
%\usepackage[tracking]{microtype}


% %\usepackage[sc,osf]{mathpazo}   % With old-style figures and real smallcaps.
% %\linespread{1.025}              % Palatino leads a little more leading

% % Euler for math and numbers
% %\usepackage[euler-digits,small]{eulervm}
% %\AtBeginDocument{\renewcommand{\hbar}{\hslash}}
\usepackage{graphicx}
\usepackage{multirow}
\usepackage{booktabs}
\usepackage{graphbox}
\usepackage{animate}
\usepackage{media9}
\usepackage{adjustbox}

% %\mode<presentation> { \setbeamercovered{transparent} }

\setbeamertemplate{navigation symbols}{}
\makeatletter
\def\beamerorig@set@color{%
  \pdfliteral{\current@color}%
  \aftergroup\reset@color
}
\def\beamerorig@reset@color{\pdfliteral{\current@color}}
\makeatother


% %=== GRAPHICS PATH ===========
\graphicspath{{./m5-images/}}
% % Marginpar width
% %Marginpar width
% %\setlength{\marginparsep}{.02in}


% %% Captions
% % \usepackage{caption}
% % \captionsetup{
% %   labelsep=quad,
% %   justification=raggedright,
% %   labelfont=sc
% % }

% \setbeamerfont{caption}{size=\footnotesize}
% \setbeamercolor{caption name}{fg=darkcandyapplered}

% %AMS-TeX packages

\usepackage{amssymb}
\usepackage{amsmath}
\usepackage{amsthm}
\usepackage{mathtools} 
\usepackage{bm}
\DeclareMathOperator*{\argmax}{arg\,max}
\DeclareMathOperator*{\argmin}{arg\,min}
% \usepackage{color}

% %https://tex.stackexchange.com/a/31370/2269
\usepackage{cancel}
\renewcommand{\CancelColor}{\color{red}} %change cancel color to red
\makeatletter
\let\my@cancelto\cancelto %copy over the original cancelto command
\newcommand<>{\cancelto}[2]{\alt#3{\my@cancelto{#1}{#2}}{\mathrlap{#2}\phantom{\my@cancelto{#1}{#2}}}}
% redefine the cancelto command, using \phantom to assure that the
% result doesn't wiggle up and down with and without the arrow
\makeatother


% % \usepackage{comment}
\usepackage{enumerate}
\usepackage{hyperref}
% \usepackage{minitoc,array}
% \definecolor{slblue}{rgb}{0,.3,.62}
\hypersetup{
    colorlinks,%
    citecolor=blue,%
    filecolor=blue,%
    linkcolor=blue,
    urlcolor=blue
}

% \usepackage{epstopdf}
% \epstopdfDeclareGraphicsRule{.gif}{png}{.png}{convert gif:#1 png:\OutputFile}
% \AppendGraphicsExtensions{.gif}

% %\usepackage{listings}

% %%% TIKZ
\usepackage{forest}
\usepackage{tikz}
\usepackage{tikz-3dplot}
\usepackage{pgfplots}
\usepackage{pgfplotstable}
% \usepackage{pgfgantt}
\usepackage{neuralnetwork}

\usetikzlibrary{fit,arrows,arrows.meta,shapes,positioning,shapes.geometric}
\usetikzlibrary{decorations.markings}
\usetikzlibrary{shadows,automata}
\usetikzlibrary{patterns}
\usetikzlibrary{trees,mindmap,backgrounds}
%\usetikzlibrary{circuits.ee.IEC}
\usetikzlibrary{decorations.text}
% % For Sagnac Picture
% \usetikzlibrary{%
%     decorations.pathreplacing,%
%     decorations.pathmorphing%
% }
% \tikzset{no shadows/.style={general shadow/.style=}}
% %
% %\usepackage{paralist}

\tikzset{
  font=\Large\sffamily\bfseries,
  red arrow/.style={
    midway,red,sloped,fill, minimum height=3cm, single arrow, single arrow head extend=.5cm, single arrow head indent=.25cm,xscale=0.3,yscale=0.15,
    allow upside down
  },
  black arrow/.style 2 args={-stealth, shorten >=#1, shorten <=#2},
  black arrow/.default={1mm}{1mm},
  tree box/.style={draw, rounded corners, inner sep=1em},
  node box/.style={white, draw=black, text=black, rectangle, rounded corners},
}

% %%% FORMAT PYTHON CODE
% %\usepackage{listings}
% % Default fixed font does not support bold face
% \DeclareFixedFont{\ttb}{T1}{txtt}{bx}{n}{8} % for bold
% \DeclareFixedFont{\ttm}{T1}{txtt}{m}{n}{8}  % for normal

% % Custom colors
% \definecolor{deepblue}{rgb}{0,0,0.5}
% \definecolor{deepred}{rgb}{0.6,0,0}
% \definecolor{deepgreen}{rgb}{0,0.5,0}

% %\usepackage{animate}

% % Python style for highlighting
% % \newcommand\pythonstyle{\lstset{
% % language=Python,
% % basicstyle=\footnotesize\ttm,
% % otherkeywords={self},             % Add keywords here
% % keywordstyle=\footnotesize\ttb\color{deepblue},
% % emph={MyClass,__init__},          % Custom highlighting
% % emphstyle=\footnotesize\ttb\color{deepred},    % Custom highlighting style
% % stringstyle=\color{deepgreen},
% % frame=tb,                         % Any extra options here
%     % showstringspaces=false            % 
% % }}

% % % Python environment
% % \lstnewenvironment{python}[1][]
% % {
% % \pythonstyle
% % \lstset{#1}
% % }
% % {}

% % % Python for external files
% % \newcommand\pythonexternal[2][]{{
% % \pythonstyle
% % \lstinputlisting[#1]{#2}}}

% % Python for inline
% % 
% % \newcommand\pythoninline[1]{{\pythonstyle\lstinline!#1!}}

% %\usepackage{algorithm2e}

\newcommand{\eps}{\epsilon}
\newcommand{\bX}{\mb X}
\newcommand{\by}{\mb y}
\newcommand{\bbe}{\bm\beta}
\newcommand{\beps}{\bm\epsilon}
\newcommand{\bY}{\mb Y}

\newcommand{\osn}{\oldstylenums}
\newcommand{\dg}{^{\circ}}
\newcommand{\lt}{\left}
\newcommand{\rt}{\right}
\newcommand{\pt}{\phantom}
\newcommand{\tf}{\therefore}
\newcommand{\?}{\stackrel{?}{=}}
\newcommand{\fr}{\frac}
\newcommand{\dfr}{\dfrac}
\newcommand{\ul}{\underline}
\newcommand{\tn}{\tabularnewline}
\newcommand{\nl}{\newline}
\newcommand\relph[1]{\mathrel{\phantom{#1}}}
\newcommand{\cm}{\checkmark}
\newcommand{\ol}{\overline}
\newcommand{\rd}{\color{red}}
\newcommand{\bl}{\color{blue}}
\newcommand{\pl}{\color{purple}}
\newcommand{\og}{\color{orange!90!black}}
\newcommand{\gr}{\color{green!40!black}}
\newcommand{\lbl}{\color{CornflowerBlue}}
\newcommand{\dca}{\color{darkcandyapplered}}
\newcommand{\nin}{\noindent}
\newcommand*\circled[1]{\tikz[baseline=(char.base)]{
            \node[shape=circle,draw,thick,inner sep=1pt] (char) {\small #1};}}

\newcommand{\bc}{\begin{compactenum}[\quad--]}
\newcommand{\ec}{\end{compactenum}}

\newcommand{\p}{\partial}
\newcommand{\pd}[2]{\frac{\partial{#1}}{\partial{#2}}}
\newcommand{\dpd}[2]{\dfrac{\partial{#1}}{\partial{#2}}}
\newcommand{\pdd}[2]{\frac{\partial^2{#1}}{\partial{#2}^2}}
\newcommand{\pde}[3]{\frac{\partial^2{#1}}{\partial{#2}\partial{#3}}}
\newcommand{\nmfr}[3]{\Phi\left(\frac{{#1} - {#2}}{#3}\right)}
\newcommand{\Err}{\text{Err}}
\newcommand{\err}{\text{err}}

\DeclarePairedDelimiter\ceil{\lceil}{\rceil}
\DeclarePairedDelimiter\floor{\lfloor}{\rfloor}

%%%% GREEK LETTER SHORTCUTS %%%%%
\newcommand{\la}{\lambda}
\renewcommand{\th}{\theta}
\newcommand{\al}{\alpha}
\newcommand{\G}{\Gamma}
\newcommand{\si}{\sigma}
\newcommand{\Si}{\Sigma}


\pgfmathdeclarefunction{poiss}{1}{%
  \pgfmathparse{(#1^x)*exp(-#1)/(x!)}%
  }

\pgfmathdeclarefunction{gauss}{2}{%
  \pgfmathparse{1/(#2*sqrt(2*pi))*exp(-((x-#1)^2)/(2*#2^2))}%
}

\pgfmathdeclarefunction{expo}{2}{%
  \pgfmathparse{#1*exp(-#1*#2)}%
}

\pgfmathdeclarefunction{expocdf}{2}{%
  \pgfmathparse{1 -exp(-#1*#2)}%
}

\newcommand{\mb}{\mathbb}
\newcommand{\mc}{\mathcal}
\newcommand{\tr}{^{\top}}
\newcommand{\empt}[2]{$#1^{( #2 )}$}
\newcommand{\pe}{\pause}
% \usepackage{pst-plot}

% \usepackage{pstricks-add}
% \usepackage{auto-pst-pdf}   

% \psset{unit = 3}

% \def\target(#1,#2){%
%  {\psset{fillstyle = solid}
%   \rput(#1,#2){%
%     \pscircle[fillcolor = white](0.7,0.7){0.7}
%     \pscircle[fillcolor = blue!60](0.7,0.7){0.5}
%     \pscircle[fillcolor = white](0.7,0.7){0.3}
%     \pscircle[fillcolor = red!80](0.7,0.7){0.1}}}}
% \def\dots[#1](#2,#3){%
%     \psRandom[
%       dotsize = 2pt,
%       randomPoints = 25
%     ](!#2 #1 0.04 sub sub #3 #1 0.04 sub sub)%
%      (!#2 #1 0.04 sub add #3 #1 0.04 sub add)%
%      {\pscircle[linestyle = none](#2,#3){#1}}}


%%%%%%%%%%%%%%%%%%%%%%%%%%%%%%%%%%%%%%%%%%%%%%%%%%%
%%%%%%%%%%%%%%%%%%%%%%%%%%%%%%%%%%%%%%%%%%%%%%%%%%%
\title[\shortlecturetitle]{ {\normalsize \coursetitle}
  \\ \longlecturetitle}
\date[\lecturedate]{\footnotesize \lecturedate}
\author{{\bf \instructor}}
\institute[UMass Amherst]{
%\titlegraphic{\hfill
  \begin{tikzpicture}[baseline=(current bounding box.center)]
    \node[anchor=base] at (-7,0) (its) {\includegraphics[scale=.3]{UMassEngineering_vert}} ;
  \end{tikzpicture}
  % \hfill\includegraphics[height=1.5cm]{logo}
}

%https://tex.stackexchange.com/questions/55806/mindmap-tikzpicture-in-beamer-reveal-step-by-step
  \tikzset{
    invisible/.style={opacity=0},
    visible on/.style={alt={#1{}{invisible}}},
    alt/.code args={<#1>#2#3}{%
      \alt<#1>{\pgfkeysalso{#2}}{\pgfkeysalso{#3}} % \pgfkeysalso doesn't change the path
    },
  }


% https://tex.stackexchange.com/questions/446468/labels-with-arrows-for-an-equation
% https://tex.stackexchange.com/a/402466/121799
\newcommand{\tikzmark}[3][]{
\ifmmode
\tikz[remember picture,baseline=(#2.base)] \node [inner sep=0pt,#1](#2) {$#3$};
\else
\tikz[remember picture,baseline=(#2.base)] \node [inner sep=0pt,#1](#2) {#3};
\fi
}

% \lstset{language=matlab,
%                 basicstyle=\scriptsize\ttfamily,
%                 keywordstyle=\color{blue}\ttfamily,
%                 stringstyle=\color{blue}\ttfamily,
%                 commentstyle=\color{gray}\ttfamily,
%                 morecomment=[l][\color{gray}]{\#}
%               }


%%% Local Variables:
%%% mode: latex
%%% TeX-master: t
%%% End:

%\setbeamercolor{local structure}{fg=white}

\begin{document}
\maketitle
\begin{frame}
  \frametitle{Outline}
  \tableofcontents
\end{frame}


 

\section{Factor analysis}

 
\begin{frame}
  \frametitle{Factor analysis model}
  \pe
    \begin{itemize}
  \item Basic idea: there are latent (hidden) \textbf{common factors} $\bm z$ underlying some multivariate observations $\bm x_{n} \in \mb R^{D}$

  \end{itemize}
  \pe
  Factor analysis (FA) is a latent variable generative model specified as:\pe

  \begin{eqnarray}
    p(\bm z) &=& \mc{N}(\bm z|\bm \mu_{0},\bm\Sigma_{0}) \\\pe
    p(\bm x|\bm z,\bm\th) &=& \mc N(\bm x|\bm W\bm z + \bm\mu, \bm \Psi)
  \end{eqnarray}\pe
  where:
  \begin{itemize}
  \item $\bm z$: latent vector \pe
  \item $\bm W$: factor loading matrix, $D\times L$\pe
  \item $\bm \Psi$: diagonal covariance matrix, $D\times D$\pe
  \end{itemize}
\end{frame}

\begin{frame}
  \frametitle{Induced marginal distribution}
  \pe

  \begin{eqnarray}
    p(\bm x|\bm \th) &=&\pe
                         \int \mc{N}(\bm x|\bm W\bm z  + \bm\mu, \bm\Psi)\mc{N}(\bm z|\bm\mu_{0},\bm\Sigma_{0})d\bm z \\\pe
                     &=& \mc{N}(\bm x|\underbrace{\bm W{\rd\bm\mu_{0}} + \bm \mu}_{\text{mean}} ,
                         \underbrace{\bm\Psi + \bm W{\bl\bm\Sigma_{0}}\bm W\tr}_{\text{variance}})
  \end{eqnarray}

  \pe
  Simplifications:
  \begin{itemize}
  \item $\rd\bm\mu_{0} \to \bm 0$ \pe
  \item $\bl\bm\Sigma_{0} \to \bm I$
  \end{itemize}
  \pe
  The simplified marginal distribution then becomes:\pe
  \begin{equation}
    p(\bm x|\bm\th) = \mc{N}(\bm x|\bm \mu, \bm{WW}\tr+\bm\Psi)
  \end{equation}
\end{frame}

\begin{frame}
  \frametitle{Generative model simplified}

  \begin{eqnarray}
    \text{[Prior]}\quad   p(\bm z) &=& \mc{N}(\bm z|\bm 0, \bm I) \\\pe
    \text{[Likelihood]}\quad   p(\bm x|\bm z) \pe &=& \mc{N}(\bm x|\bm {Wz} + \bm\mu, \bm\Psi) \\\pe
    \text{[Evidence/marginal]}\quad   p(\bm x) \pe &=& \mc{N}(\bm x|\bm \mu, \bm{WW}\tr + \bm\Psi)
  \end{eqnarray}
\end{frame}

\begin{frame}
  \frametitle{What FA does}
  \pe

  It approximates the covariance matrix of the visible/observed vector $\bm x$ using a low-rank decomposition:\pe

  \begin{equation}
    \bm C = \text{Cov}[\bm x] = \underbrace{\bm{WW}\tr + \bm\Psi}_{\text{low-rank decomp}}
  \end{equation}
  \pe
  \begin{itemize}
  \item $\bm{WW}\tr$ is $D\times D$ (recall: $\bm W \in \mathbb{R}^{D\times L}$) \pe
  \item $\bm\Psi$ is $D\times D$ (restricted to be diagonal) \pe
  \item For each variable $x_{d}$, the \textbf{marginal variance} (each diagonal term in $\bm C$) is given by: \pe
    \begin{equation}
      \mb{V}[x_{d}] = \sum_{k=1}^{L}\underbrace{w_{dk}^{2}}_{\text{common}} + \underbrace{\psi_{d}}_{\text{unique}}
    \end{equation}
  \end{itemize}
\end{frame}

\begin{frame}
  \frametitle{Parameters to be estimated}
  \pe
  The unknown parameters in FA are:
  \pe
  \begin{itemize}
  \item $\bm W$: factor loading matrix \pe
  \item $\bm \Psi$: covariance matrix
  \end{itemize}

  \pe
  These can be estimated via:
  \begin{itemize}
  \item Maximum likelihood estimation (MLE) \pe
  \item Expectation-maximization (EM) algorithm \pe
  \item Bayesian methods (e.g., variational inference, MCMC)
  \end{itemize}
\pe

  Once estimated, the \textbf{posterior} of latent embeddings is given by:\pe
  \begin{equation}
    p(\bm z|\bm x) = \mc{N}(\bm z|\bm W\tr \bm C^{-1}(\bm x - \bm\mu),\bm I - \bm{W}\tr\bm{C}^{-1}\bm{W})
  \end{equation}
  \pe
  \begin{itemize}
  \item Posterior has closed-form solution under Gaussian distribution
  \end{itemize}
\end{frame}

\begin{frame}
  \frametitle{FA model summary}
  \pe
    \begin{eqnarray}
    \text{[Prior]}\quad   p(\bm z) &=& \mc{N}(\bm z|\bm 0, \bm I) \\\pe
    \text{[Likelihood]}\quad   p(\bm x|\bm z) \pe &=& \mc{N}(\bm x|\bm {Wz} + \bm\mu, \bm\Psi) \\\pe
    \text{[Evidence/marginal]}\quad   p(\bm x) \pe &=& \mc{N}(\bm x|\bm \mu, \bm C) \\\pe
     \text{[Posterior]}\quad p(\bm z|\bm x) &=& \mc{N}(\bm z|\bm W\tr \bm C^{-1}(\bm x - \bm\mu),\bm I - \bm{W}\tr\bm{C}^{-1}\bm{W})  
    \end{eqnarray}
    \pe

    \begin{itemize}
    \item $\bm z$: latent vector, length $L$ (assumed to be zero-mean, unit variance) \pe
    \item $\bm x$: observed vector \pe
    \item $\bm W$: $D\times L$ factor loading matrix\pe
    \item $\bm \Psi$: $D\times D$ diagonal covariance matrix or \textbf{matrix of unique variances} \pe
    \item $\bm C = \bm{WW}\tr + \bm\Psi$
    \end{itemize}
\end{frame}
  
\section{FA Estimation}
\begin{frame}
  \frametitle{Unidentifiability of FA parameters}

  \pe

  The parameters $\bm W$ and $\bm\Psi$ are unidentifiable. \pe This can be addressed via: \pe

  \begin{itemize}
  \item Constraining $\bm W$ to have orthonormal columns [PCA] \pe
  \item Constraining $\bm W$ to be lower triangular \pe
  \item Informative rotation: $\tilde{\bm W} = \bm{WR}$, where $\bm R$ is the rotation matrix \pe
    \begin{itemize}
      \item Commonly used rotations: Varimax, Promax, Oblimin, Geomin, Thurstone, Equamax
    \end{itemize}\pe
  \item Sparsity-promoting priors on $\bm W$ \pe
  \item Non-Gaussian priors for latent factors
  \end{itemize}
\end{frame}


% \begin{frame}
%   \frametitle{Rotation}
%   \pe
%   We rotate factors for identifiability and interpretability of factors:
  
%   \begin{itemize}
%   \item Varimax
%   \item Promax
%   \item Oblimin
%   \item Geomin
%   \item Thurstone
%   \item Equamax
%   \end{itemize}
% \end{frame}

\begin{frame}
  \frametitle{PCA as a special case of FA}\pause

  Principal components analysis (PCA) is a special case of FA with:\pe
  \begin{equation}
    \bm\Psi = \sigma^2 \bm I
  \end{equation}
  \pe
  where $\sigma^2$ is the isotropic noise variance. \pe
  In PCA, the covariance of the observed vector is given by:\pe
  \begin{equation}
    \bm C = \bm{WW}\tr + \sigma^2 \bm I
  \end{equation}  

  

\end{frame}

\section{Autoencoders}
\begin{frame}
  \frametitle{Autoencoders as nonlinear PCA/FA}
  \pe
  In PCA/FA, we learn a \textbf{linear mapping} from a high-dimensional observed space $\bm x \in\mb R^D$ to a low-dimensional latent space $\bm z \in\mb R^L$ and vice-versa.
  \pe

  \begin{itemize}
  \item \textbf{Encoder} $f_e$: mapping from $\bm x\to \bm z$ \pe
  \item \textbf{Decoder} $f_d$: mapping from $\bm z \to \bm x$
  \end{itemize}
  \pe
  In PCA, for example, $f_e$ is given by:\pe
  \begin{equation}
    \bm z = \bm W\tr \bm x \equiv f_e(\bm x)
  \end{equation}
  \pe
  And $f_d$ is given by: \pe
  \begin{equation}
    \hat{\bm x} = \bm W \bm z \equiv f_d(\bm z)
  \end{equation}
  To introduce flexibility, we can specify $f_e$ and $f_e$ are nonlinear/more complex functions. This is best accomplished via neural network, resulting in an \textbf{autoencoder} (AE).
\end{frame}

\begin{frame}
  \frametitle{Reconstruction loss}
  \pe
  The reconstruction function is the approximation of the observation from the decoder: \pe
  \begin{equation}
    \hat{\bm x} \equiv r(\bm x) = f_d(f_e(\bm x))
  \end{equation}
  \pe
  An autoencoder is thus trained to minimize the reconstruction loss \pe
  \begin{equation}
    \mc{L}(\bm\th) = ||r(\bm x) - \bm x||_2^2
  \end{equation}
  or equivalently, the negative log-likelihood:\pe
  \begin{equation}
    \mc{L}(\bm\th) = -\log p(\bm x|r(\bm x))
  \end{equation}
\end{frame}

\begin{frame}[fragile]
  \frametitle{Basic autoencoder (AE) architecture}
  \pe

  Autoencoder with 2 single-layer MLPS: input layer, hidden layer (latent representation) and output layer (reconstruction)

  \bigskip
  
  \newcommand{\xin}[2]{$x_#2$}
  \newcommand{\xout}[2]{$\hat x_#2$}
  \newcommand{\z}[2]{$z_#2$}

  \begin{minipage}{.45\linewidth}
\begin{adjustbox}{height= .6\textheight}\centering
  \begin{neuralnetwork}
    
  \tikzstyle{input neuron}=[neuron, fill=orange!70];
  \tikzstyle{output neuron}=[neuron, fill=blue!60!black, text=white];

  \inputlayer[count=7, bias=false, text=\xin]

  % \hiddenlayer[count=5, bias=false]
  % \linklayers

  \hiddenlayer[count=3, bias=false,text  = \z]
  \linklayers

  % \hiddenlayer[count=5, bias=false, text = \z]
  % \linklayers

  \outputlayer[count=7,  text=\xout]
  \linklayers
\end{neuralnetwork}
\end{adjustbox}
\end{minipage}\pe
\begin{minipage}{.45\linewidth}
  \begin{itemize}
  \item Hidden layer (size $L$) is a low-dimensional \textbf{bottleneck} between input and reconstruction \pe
  \item $L\ll D$: undercomplete representation \pe
  \item $L \gg D$: overcomplete representation (regularize to prevent identity learning)
  \end{itemize}
\end{minipage}
\end{frame}


\section{AE variants}
\begin{frame}
  \frametitle{Denoising autoencoders}\pe
  In denoising autoencoders (DAEs), the input is corrupted ($\tilde{\bm x}$) by:
  \begin{itemize}
  \item Gaussian noise: $p_c(\tilde{\bm x}|\bm x) = \mc N(\tilde{\bm x}|\bm x, \sigma^2\bm I)$ \pe

  \item Bernoulli dropout: randomly setting a proportion of input nodes to zero
    
  \end{itemize}



  
    \begin{center}
      \includegraphics[width=.7\textwidth]{dae}

      {\tiny Schematic of a DAE. \\ Source: \url{https://lilianweng.github.io/posts/2018-08-12-vae/}}
    \end{center}

    
  The model is then trained to
  minimize the loss between the reconstructed input $r(\tilde{\bm x})$ and its uncorrupted version $\bm x$
  
\end{frame}

\begin{frame}
  \frametitle{Uses of DAE}
  \pe
  \begin{itemize}
  \item DAEs are used for denoising images \pe
    
  
    \begin{center}
      \includegraphics[width=.5\textwidth]{dae-recon}

      {\tiny Original, corrupted and reconstructed images from MNIST dataset. \\ Source: \url{http://www.opendeep.org/v0.0.5/docs/tutorial-your-first-model}}
    \end{center}

  \item They can also learn vector fields of input data
  \pe
  \item Popular for their simplicity
  \end{itemize}
\end{frame}



\begin{frame}
  \frametitle{Sparse autoencoder (SAE)}

  \pe
  \textbf{Sparse autoencoder} (SAE): sparsity penalty on latent activations \pe

    \begin{equation}
      \Omega(\bm z) = \la ||\bm z||_1
    \end{equation}
    \begin{itemize}
    \item \href{https://arxiv.org/pdf/1312.5663.pdf}{$k$-Sparse autoencoder}: use only $k$ largest activations in training
    \end{itemize}

   
      \begin{center}
      \includegraphics[width=.5\textwidth]{k-sparse-autoencoder}

      {\tiny Filters of the k -sparse autoencoder for different sparsity levels k, learnt from MNIST with 1000 hidden units. \\ Source: \url{https://arxiv.org/pdf/1312.5663.pdf}}
    \end{center}

     Useful for interpretability
  \end{frame}

\begin{frame}
  \frametitle{Other AEs}
  \pe

  \begin{itemize}
  \item \href{http://www.icml-2011.org/papers/455_icmlpaper.pdf}{\textbf{Contractive autoencoder}} (CAE): regularizes via penalty on reconstruction loss \pe

    \begin{equation}
      \Omega(\bm z,\bm x) = \la \Bigg|\Bigg|\fr{\partial f_e(\bm x)}{\partial \bm x} \Bigg|\Bigg|_F^2
    \end{equation}
    \pe

  \item Variational autoencoder (VAE): probablistic version of AE/generative model
  \end{itemize}

\end{frame}

\section{Outlook}
\begin{frame}
  \frametitle{Reading}

  \begin{itemize}
  \item \textbf{PMLI} 20.3
  \item \textbf{DL} 20
  \end{itemize}
\end{frame}
  
\end{document}

%%% Local Variables:
%%% mode: latex
%%% TeX-master: t
%%% End:
