\documentclass[11pt,twoside]{article}
\usepackage{etex}

\raggedbottom

%geometry (sets margin) and other useful packages
\usepackage{geometry}
\geometry{top=1in, left=1in,right=1in,bottom=1in}
 \usepackage{graphicx,booktabs,calc}
 
\usepackage{listings}


% Marginpar width
%Marginpar width
\newcommand{\pts}[1]{\marginpar{ \small\hspace{0pt} \textit{[#1]} } } 
\setlength{\marginparwidth}{.5in}
%\reversemarginpar
%\setlength{\marginparsep}{.02in}

 
%\usepackage{cmbright}lstinputlisting
%\usepackage[T1]{pbsi}


\usepackage{chngcntr,mathtools}
\counterwithin{figure}{section}
\numberwithin{equation}{section}

%\usepackage{listings}

%AMS-TeX packages
\usepackage{amssymb,amsmath,amsthm} 
\usepackage{bm}
\usepackage[mathscr]{eucal}
\usepackage{colortbl}
\usepackage{color}


\usepackage{subfigure,hyperref,enumerate,polynom,polynomial}
\usepackage{multirow,minitoc,fancybox,array,multicol}

\definecolor{slblue}{rgb}{0,.3,.62}
\hypersetup{
    colorlinks,%
    citecolor=blue,%
    filecolor=blue,%
    linkcolor=blue,
    urlcolor=slblue
}

%%%TIKZ
\usepackage{tikz}

\usepackage{pgfplots}
\pgfplotsset{compat=newest}

\usetikzlibrary{arrows,shapes,positioning}
\usetikzlibrary{decorations.markings}
\usetikzlibrary{shadows}
\usetikzlibrary{patterns}
%\usetikzlibrary{circuits.ee.IEC}
\usetikzlibrary{decorations.text}
% For Sagnac Picture
\usetikzlibrary{%
    decorations.pathreplacing,%
    decorations.pathmorphing%
}

\tikzstyle arrowstyle=[black,scale=2]
\tikzstyle directed=[postaction={decorate,decoration={markings,
    mark=at position .65 with {\arrow[arrowstyle]{stealth}}}}]
\tikzstyle reverse directed=[postaction={decorate,decoration={markings,
    mark=at position .65 with {\arrowreversed[arrowstyle]{stealth};}}}]
\tikzstyle dir=[postaction={decorate,decoration={markings,
    mark=at position .98 with {\arrow[arrowstyle]{latex}}}}]
\tikzstyle rev dir=[postaction={decorate,decoration={markings,
    mark=at position .98 with {\arrowreversed[arrowstyle]{latex};}}}]

\usepackage{ctable}

%
%Redefining sections as problems
%
\makeatletter
\newenvironment{exercise}{\@startsection 
	{section}
	{1}
	{-.2em}
	{-3.5ex plus -1ex minus -.2ex}
    	{1.3ex plus .2ex}
    	{\pagebreak[3]%forces pagebreak when space is small; use \eject for better results
	\large\bf\noindent{Exercise 1.\hspace{-1.5ex} }
	}
	}
	%{\vspace{1ex}\begin{center} \rule{0.3\linewidth}{.3pt}\end{center}}
	%\begin{center}\large\bf \ldots\ldots\ldots\end{center}}
\makeatother

%
%Fancy-header package to modify header/page numbering 
%
\usepackage{fancyhdr}
\pagestyle{fancy}
%\addtolength{\headwidth}{\marginparsep} %these change header-rule width
%\addtolength{\headwidth}{\marginparwidth}
%\fancyheadoffset{30pt}
%\fancyfootoffset{30pt}
\fancyhead[LO,RE]{\small Oke}
\fancyhead[RO,LE]{\small Page \thepage} 
\fancyfoot[RO,LE]{\small PS 3} 
\fancyfoot[LO,RE]{\small \scshape CEE 260/MIE 273} 
\cfoot{} 
\renewcommand{\headrulewidth}{0.1pt} 
\renewcommand{\footrulewidth}{0.1pt}
%\setlength\voffset{-0.25in}
%\setlength\textheight{648pt}


\usepackage{paralist}

\newcommand{\osn}{\oldstylenums}
\newcommand{\lt}{\left}
\newcommand{\rt}{\right}
\newcommand{\pt}{\phantom}
\newcommand{\tf}{\therefore}
\newcommand{\?}{\stackrel{?}{=}}
\newcommand{\fr}{\frac}
\newcommand{\dfr}{\dfrac}
\newcommand{\ul}{\underline}
\newcommand{\tn}{\tabularnewline}
\newcommand{\nl}{\newline}
\newcommand\relph[1]{\mathrel{\phantom{#1}}}
\newcommand{\cm}{\checkmark}
\newcommand{\ol}{\overline}
\newcommand{\rd}{\color{red}}
\newcommand{\bl}{\color{blue}}
\newcommand{\pl}{\color{purple}}
\newcommand{\og}{\color{orange!90!black}}
\newcommand{\gr}{\color{green!40!black}}
\newcommand{\nin}{\noindent}
\newcommand{\la}{\lambda}
\renewcommand{\th}{\theta}
\newcommand*\circled[1]{\tikz[baseline=(char.base)]{
            \node[shape=circle,draw,thick,inner sep=1pt] (char) {\small #1};}}

\newcommand{\bc}{\begin{compactenum}[\quad--]}
\newcommand{\ec}{\end{compactenum}}

\newcommand{\n}{\\[2mm]}
%% GREEK LETTERS
\newcommand{\al}{\alpha}
\newcommand{\gam}{\gamma}
\newcommand{\eps}{\epsilon}
\newcommand{\sig}{\sigma}

\newcommand{\p}{\partial}
\newcommand{\pd}[2]{\frac{\partial{#1}}{\partial{#2}}}
\newcommand{\dpd}[2]{\dfrac{\partial{#1}}{\partial{#2}}}
\newcommand{\pdd}[2]{\frac{\partial^2{#1}}{\partial{#2}^2}}
\newcommand{\mr}{\mathbb{R}}
\newcommand{\xs}{x^{*}}
\newenvironment{solution}
{\medskip\par\quad\quad\begin{minipage}[c]{.8\textwidth}\gr}{\medskip\end{minipage}}

 
%%%%%%%%%%%%%%%%%%%%%%%%%%%%%%%%%%%%%%%%%%%%%%%%%%%
%%%%%%%%%%%%%%%%%%%%%%%%%%%%%%%%%%%%%%%%%%%%%%%%%%%

\begin{document}

\lstset{language=C++,
                basicstyle=\tiny\ttfamily,
                keywordstyle=\color{blue}\ttfamily,
                stringstyle=\color{red}\ttfamily,
                commentstyle=\color{gray}\ttfamily,
                morecomment=[l][\color{gray}]{\#}
}


\thispagestyle{empty}


\nin{\LARGE Problem Set 3 {\gr }}\hfill{\bf Prof. Oke}

\medskip\hrule\medskip

\nin {\small CEE 260/MIE 273: Probability \& Statistics in Civil Engineering
\hfill\textit{ 09.16.2025}}

\bigskip

\nin{\it Due September 23, 2025 at 1:00 PM as PDF and .ipynb/.m files uploaded on Canvas.
  If it helps and if possible, you can write your responses directly on this document and upload it instead.
    \textbf{Show as much work as possible in order to get FULL credit.}
  There are 4 problems with a total of 31 points available.
}\\


\section*{Problem 1 \textit{(8 points)}}

Respond ``T'' ({\it True})  or  ``F'' (\textit{False}) to the following statements. Use the boxes provided. Each response is worth 1 point.

\begin{enumerate}[\bf (i)]
\item \hfill
  \begin{minipage}{.1\linewidth}
    \framebox(40,40){ \gr  }
  \end{minipage}\quad
  \begin{minipage}{.85\linewidth}
    Given three events $D$, $A$ and $G$. If $P(\ol{D}|AG) = 1$, then $P(D)$ is an impossible event.
 
  \end{minipage}
%  \underline{\hspace{10ex}} %F 

    \smallskip
  
\item \hfill
  \begin{minipage}{.1\linewidth}
    \framebox(40,40){\gr  }
  \end{minipage}\quad
  \begin{minipage}{.85\linewidth}
   Two events $A$ and $B$ can both be mutually exclusive and yet collectively exhaustive.
  \end{minipage}

  \smallskip
  
\item \hfill
  \begin{minipage}{.1\linewidth}
    \framebox(40,40){\gr  }
  \end{minipage}\quad
  \begin{minipage}{.85\linewidth}
    If $P(A|B) = P(A)$ for a set of events $A$ and $B$, then both events are  dependent.
  \end{minipage}

  \smallskip
  
\item \hfill
  \begin{minipage}{.1\linewidth}
    \framebox(40,40){\gr  }
  \end{minipage}\quad
  \begin{minipage}{.85\linewidth}
    Events $E$ and $F$ are mutually exclusive and collectively exhaustive. If $P(E) = 0.3$, then $P(F) = 0.7$.
   \end{minipage}

  \smallskip

  \item \hfill
  \begin{minipage}{.1\linewidth}
    \framebox(40,40){\gr  }
  \end{minipage}\quad
  \begin{minipage}{.85\linewidth}
    Events $E$ and $F$ are  collectively exhaustive but \textit{not} mutually exclusive.
    For a given event $A$, its probability can be obtained by $P(A) = P(AE) + P(AF)$.
    % {\gr This only holds if $E$ and $F$ are mutually exclusive. But since they are not, then $P(A) = P(AE\cup AF) =
    %   P(AE) + P(AF) - P(AE\cap AF)$ (addition rule).}
  \end{minipage}
  
  \smallskip
  \item \hfill
  \begin{minipage}{.1\linewidth}
    \framebox(40,40){\gr  }
  \end{minipage}\quad
  \begin{minipage}{.85\linewidth}
    The number of ways 6 books can be arranged on a bookshelf is 720. If two of the books are identical, then the total number of distinct arrangements is 360. %{\gr The permutations are $\fr{n!}{n_{1}!}$, where $n_{1} = 2$. Thus $\fr{6!}{2} = 360$.}
  \end{minipage}
  
  \smallskip
  
\item \hfill
  \begin{minipage}{.1\linewidth}
    \framebox(40,40){\gr  }
  \end{minipage}\quad
  \begin{minipage}{.85\linewidth}
    The number of distinct subgroups of size $n$ that can be formed from a larger group of $m$ objects is given by $\fr{m!}{n!(n-m)!}$.
    %{\gr (The correct answer is $\fr{m!}{n!(m-n)!}$)}
  \end{minipage}

  \smallskip

  \item \hfill
  \begin{minipage}{.1\linewidth}
    \framebox(40,40){\gr  }
  \end{minipage}\quad
  \begin{minipage}{.85\linewidth}
    The events $E_{1}$ and $E_{2}$ are independent. If $P(E_{1}) = 0.4$ and $P(E_{1}E_{2}) = 0.04$, then $P(E_{2}) = 0.1$.
    %{\gr ($P(E_{2}) = P(E_{1}E_{2})/P(E_{1}) = 0.04/0.4 = 0.1$)}
  \end{minipage}
  \end{enumerate}
% \item \hfill
%   \begin{minipage}{.1\linewidth}
%     \framebox(40,40){\gr T}
%   \end{minipage}\quad
%   \begin{minipage}{.85\linewidth}
%     The number of ways 6 books can be arranged on a bookshelf is 720. If two of the books are identical, then the total number of distinct arrangements is 360. {\gr The permutations are $\fr{n!}{n_{1}!}$, where $n_{1} = 2$. Thus $\fr{6!}{2} = 360$.}
%   \end{minipage}
  
  %\smallskip
  
% \item \hfill
%   \begin{minipage}{.1\linewidth}
%     \framebox(40,40){\gr  }
%   \end{minipage}\quad
%   \begin{minipage}{.85\linewidth}
%     The number of tails $X$ in 50 tosses of a coin is a continuous random variable.
%   \end{minipage}

%   \smallskip

%   \item \hfill
%   \begin{minipage}{.1\linewidth}
%     \framebox(40,40){\gr  }
%   \end{minipage}\quad
%   \begin{minipage}{.85\linewidth}
%   The length of time $T$ between car accidents at a busy intersection is a continuous random variable.
%   \end{minipage}
%   \end{enumerate}

  % \smallskip
  
% \item \hfill
%   \begin{minipage}{.1\linewidth}
%     \framebox(40,40){}
%   \end{minipage}\quad
%   \begin{minipage}{.85\linewidth}
%     A good clustering solution should maximize the total within-cluster variation. %False
%   \end{minipage}

  \eject

   \section*{Problem 2 \textit{(8 points)}}
Data collected at elementary schools in DeKalb County, GA, suggest that each
year roughly 25\% of students miss exactly one day of school, 15\% miss 2 days, and 30\% miss 3 or more days
due to sickness.
\vspace{-2ex}
\begin{quote}
    \subsubsection*{\bl Example}\small
    \bl   Let $X$ be the number of school days missed in a year. The probability that a student chosen at random misses 2 or more days of school in a year is given by:
    \begin{equation*}
      P(X \ge 2) = P(X=2) + P(X \ge 3) = 0.15 + 0.30 = 0.45
  \end{equation*}
\end{quote}
\begin{enumerate}[\bf (a)]
\item  What is the probability that a student chosen at random does not miss any days of school due to sickness
  this year?
 \vspace{18ex}
\item What is the probability that a student chosen at random misses no more than one day?
 \vspace{20ex}

\item What is the probability that a student chosen at random misses at least one day?
 \vspace{20ex}

 \item What is the probability that a student chosen at random misses one or two days of school a year?
 \vspace{25ex}

 

\end{enumerate}

% \section*{Problem 2 \textit{(17 points)}}
% For a construction project, you are given the following conditional
% probabilities: $P(E|G)$, $P(E|N)$, $P(E|BC)$ and $P(E|B\ol{C})$, where $E$
% denotes project completion, $G$ good weather, $N$ normal weather $B$ bad
% weather, and $C$ the launching of a crash program to speed up completion.
% \begin{align*}
%   P(E|G) &= 1 \\
%   P(E|N) &= 0.9 \\
%   P(E|BC) &=  0.8 \\
%   P(E|B\ol{C}) &= 0.2 \\
%   P(G) & = 0.2 \\
%   P(N) &= 0.4 \\
%   P(B) &= 0.4 \\
%   P(C) &= 0.5 
% \end{align*}

% \begin{enumerate}[(a)]
% \item How would you chracterize the event $(E|G)$? \pts{1}
%   \vspace{5ex}

% \item How would characterize the events $E|BC$ and $E|B\ol{C}$? \pts{1}
%   \vspace{5ex}

% \item $P(G) + P(N) + P(B) = 1$. How would you characterize the 3 events $G$, $N$ and $B$? \pts{1}
%   \vspace{5ex}

% \item What is the probability of the complement of $C$, i.e.\ $\ol{C}$? \pts{1}
%   \vspace{5ex}

% \item How would you characterize $C$ and $\ol{C}$? \pts{1}
%   \vspace{5ex}

% \item The Venn diagram below represents all the events in this problem. \pts{5} Match the description of each region to the appropriate event.

%        \begin{tikzpicture}[scale=.8]
%     \draw[thick] (0,0) rectangle (10,4) node[above left] {$\bm S$};
% %    \draw[thick] (1.5,0) ellipse  (4 cm and 2 cm) node[] {$\bm{D}$};
%     \filldraw[thick, fill=orange!30!white] (0,0) -- (2,0) -- (2,4) -- (0,4) -- cycle;
%     \filldraw[thick, fill=purple!20!white] (2,0) -- (6,0) -- (6,4) -- (2,4) -- cycle;
%     \filldraw[thick, fill=green!20!white] (6,0) -- (10,0) -- (10,4) -- (6,4) -- cycle;
%     \filldraw[red, thick,pattern = horizontal lines, pattern color=red] (0,0) -- (10,0) -- (10,2) -- (0,2) -- cycle;    
%     \filldraw[blue, thick,pattern=north west lines,pattern color=blue!50!white] (0,0) -- (2,0) -- (9,2) -- (2,4) -- (0,4) -- cycle;
%     % \node[blue] at (3, 2.5) {$\bm E$};
%     % \node[orange] at (1,3.75) {$\bm{G}$};
%     % \node[purple] at (4,3.75) {$\bm{N}$};
%     % \node at (8,3.75) {$\bm{B}$};
%     % \node[red] at (7,1) {$\bm{C}$};
    
%     %\begin{scope}
%     %\clip   (1.5,0) ellipse  (4 cm and 2 cm);
%     %\fill[pattern=north west lines, pattern color=green!50!black] (4,-3) rectangle (7,3);
%     %\end{scope}
% %    \draw[thick] (3,0) circle (2 cm) node[right] {$\bm{E_2}$};
%   \end{tikzpicture}

%   \begin{center}
%     \begin{tabular}{p{5.5in} p{.5in}}
%       \fbox{Orange rectangle to the left (20\% of  sample space)} & \fbox{\bf  B} \\[3mm]
%        \fbox{Purple rectangle in the middle (40\% of sample space)} &  \fbox{\bf C} \\[3mm]
%       \fbox{Green rectangle on the right (40\% of sample space)} & \fbox{\bf  E} \\[3mm]
%        \fbox{Blue pentagon hatched with diagonal lines} &  \fbox{\bf G} \\[3mm]
%        \fbox{Red rectangle hatched with horizontal lines (50\% of sample space)} &  \fbox{\bf N} 
                                                             
%     \end{tabular}
%   \end{center}
%   % \begin{itemize}
%   % \item $G$ is the orange rectangle to the left (20\% of  sample space)
%   % \item $N$ is the purple rectangle in the middle (40\% of sample space)
%   % \item $B$ is the green rectangle on the right (40\% of sample space)
%   % \item $E$ is the blue pentagon hatched with diagonal lines
%   % \item $C$ is the red rectangle hatched with horizontal lines (50\% of sample space)
%   % \end{itemize}

% % \item Draw and label a Venn diagram illustrating the space of events. \pts{3}
% %   \vspace{40ex}
  
% \item Use the theorem of total probability to show that $P(E) = 0.76$.\pts{4}
%   Hints:
%   \begin{itemize}
%   \item $ P(E) = P(E|G)P(G) + P(E|N)P(N) + P(E|BC)P(BC) + P(E|B\ol{C})P(B\ol{C})$
%   \item $B$ and $C$ are independent events.
%   \end{itemize}
%   % \begin{align*}
%   %   P(E) &= P(E|G)P(G) + P(E|N)P(N) + P(E|BC)P(BC) + P(E|B\ol{C})P(B\ol{C}) \\
%   %        &= 1(0.2) + 0.9(0.4) + 0.8(0.4)(0.5) + 0.2(0.4)(0.5) \\
%   %        &= \boxed{0.76}
%   %  \end{align*}
%  \vspace{50ex}
  

% \item Find $P(N|\ol{E})$. \pts{3}
%   % \begin{align*}
%   %   P(N|\ol{E}) &= \fr{P(\ol{E}|N)P(N)}{P(\ol{E})} \\
%   %               &= \fr{(1-0.9)(0.4)}{1-0.76} = \boxed{0.167}
%   % \end{align*}
% \end{enumerate}

% \eject

\section*{Problem 3 \textit{Bayes' Theorem (8 points)}}
Given an earthquake of intensity
\begin{equation}
  \label{eq:10}
  X :  \{\text{light } (L), \text{moderate } (M), \text{important } (I)\}
\end{equation}
and a structure that can be in a state
\begin{equation}
  \label{eq:11}
  Y :   \{\text{damaged} (D), \text{undamaged} (\ol{D})\}
\end{equation}
The likelihood of damage given earthquake intensity is given by the following conditional probabilities:
\begin{align*}
  \label{eq:12}
  P(D|L) &= 0.01 \\
  P(D|M) &= 0.10 \\
  P(D|I) &= 0.60 
  \end{align*}
and the prior probability of each intensity is given by
\begin{align*}
    P(L) &= 0.90 \\
    P(M) &= 0.08 \\
    P(I) &= 0.02 
\end{align*}
% An earthquake has just occurred and damaged buildings have been observed, i.e.\
% \begin{equation}
%   \label{eq:14}
%   y = D
% \end{equation}

\begin{enumerate}[\bf (a)]
\item Draw and label a Venn diagram illustrating the given events. \pts{5}
 \vspace{40ex}

\item Find the total probability $P(D)$. \pts{3}
  \vspace{20ex}
\end{enumerate}

\eject
\section*{Problem 4 \textit{Bayes' Theorem (continued; 7 points)}}
Use the quantities provided and the results from Problem to answer the following questions.
\begin{enumerate}[\bf (a)]
\item Use Bayes' Theorem to find the posterior probabilities $P(L|D)$, $P(M|D)$ and $P(I|D)$.\pts{6}
  \vspace{50ex}

\item Show that the probabilities $P(L|D)$, $P(M|D)$ and $P(I|D)$ sum up to 1. \pts{1}
  \vspace{15ex}


\end{enumerate}

% \item $P(A_3|E)P(E)$ \marginpar{\it[1pt]} % Bayes Theorem
  
%   \begin{tikzpicture}[scale=.8]
%     \draw[thick] (-4,-4) rectangle (7,3) node[above left] {$\bm S$};
%     \draw[thick] (1.5,0) ellipse  (4 cm and 2 cm) node[] {$\bm{E}$};
%     \draw (-1,3) -- (0,-4);
%     \draw (4,3) -- (4,-4);
%     \node at (-2,-3) {$\bm{A_1}$};
%     \node at (2,-3) {$\bm{A_2}$};
%     \node at (5,-3) {$\bm{A_3}$};
% %    \draw[thick] (3,0) circle (2 cm) node[right] {$\bm{E_2}$};
%  \end{tikzpicture}

 
%   \begin{tikzpicture}[scale=.8]
%     \draw[thick]  (-4,-3) rectangle (7,3); % node[above left] {$\bm S$};
%     \draw[thick, pattern=north west lines,pattern color=green!50!black] (0,0) circle (2 cm) node[left] {$\bm{A}$};
%     \draw[thick,  pattern=north west lines,pattern color=green!50!black] (3,0) circle (2 cm) node[right] {$\bm{B}$};
%   \end{tikzpicture}

%   % \begin{tikzpicture}[scale=.8]
%   %   \draw[thick]  (-4,-3) rectangle (7,3); % node[above left] {$\bm S$};
%   %   \draw[thick] (0,0) circle (2 cm) node[left] {$\bm{A}$};
%   %   \draw[thick] (3,0) circle (2 cm) node[right] {$\bm{B}$};
%   % \end{tikzpicture}

%   \bigskip
  
% \item $E_1\cap E_2$ \marginpar{\it[1]}
  
%   \begin{tikzpicture}[scale=.8]
%     \draw[thick]  (-4,-3) rectangle (7,3); % node[above left] {$\bm S$};
%     \draw[thick] (0,0) circle (2 cm) node[left] {$\bm{E_1}$};
%     \draw[thick] (3,0) circle (2 cm) node[right] {$\bm{E_2}$};

%     \begin{scope}
%       \clip (0,0) circle (2 cm) node[left] {$\bm{E_1}$};
%       \fill[ pattern=north west lines,pattern color=green!50!black](3,0) circle (2 cm) node[right] {$\bm{E_2}$};
%     \end{scope}
%   \end{tikzpicture}


   
 
  % \begin{tikzpicture}[scale=.8]
  %   \draw[thick]  (-4,-5.5) rectangle (7,3); % node[above left] {$\bm S$};
  %   \draw[thick] (0,0) circle (2 cm) node[left] {$\bm{A}$};
  %   \draw[thick] (3,0) circle (2 cm) node[right] {$\bm{B}$};
  %   \draw[thick] (1.5,-2.6) circle (2 cm) node[] {$\bm{C}$};

  % \end{tikzpicture}
  
    



% \eject
% \section*{Problem 4  \textit{Summary Statistics and Probabilities (22 points)}}
% \subsubsection*{Objectives}
% \begin{itemize}
% \item Learn how to import datasets

% \item Compute probabilities

% \item Create descriptive plots of datasets

% \item Compute summary statistics 
% \end{itemize}

% \subsubsection*{Requirements}
% You must have a working installation of Python, preferably with the most recent
% version of Jupyter Lab/Notebook, or MATLAB.


% \subsubsection*{Submission instructions }
% Be sure to carefully follow the submission instructions.
% The grader should be able to run your Jupyter Notebook and see all the required outcomes.
% You only need to submit the \texttt{.ipynb} or \texttt{.m} file but you need to make sure that when running your notebook on your computer, the data files are in the same folder as your notebook. This way, the relative paths for the files will work when we run your notebook to grade your work.
% Any errors in your notebook/script will result in points deducted if the required output is not visible.
% Use the template provided to write your code. Feel free to modify it to your
% liking.

% Append your name (\texttt{LASTNAME\_FIRSTNAME\_}) to \texttt{PS4\_Prob4.ipynb} or  \texttt{PS4\_Prob4.m} with an underscore (deleting
% \texttt{\_template} from the filename) and then submit, for example:
% \begin{itemize}
% \item \texttt{OKE\_JIMI\_PS4\_Prob4.ipynb} (for Python/Jupyter) 
% \item  \texttt{OKE\_JIMI\_PS4\_Prob4.m} (for Matlab)
% \end{itemize}

% Note that MATLAB filenames must begin with a letter and contain only letters, numbers and
% underscores.
% Any data or resources provided for an assignment will be available
% to the graders so you only need to upload the \texttt{.ipynb} or \texttt{.m} file.

% \subsubsection*{Data}
% The files \texttt{2013Jan\_Baltimore\_MeanTempF.csv} and
% \texttt{2014Jan\_Baltimore\_MeanTempF.csv} (contained in the archive \texttt{PS3data.zip} contain average daily temperatures (in Fahrenheit)
% recorded in Baltimore during the months January 2013 and January 2014,
% respectively.
% \bigskip



% \noindent \textbf{Only complete one of the versions: Matlab OR Python, but not both.}
% Questions on next 2 pages

% \eject
%   \subsection*{Matlab version}

% \subsubsection*{Tutorial}
%   \textit{Optional}\quad If you have never used MATLAB before, visit this link:
%   \url{https://www.mathworks.com/help/matlab/learn_matlab/desktop.html}
%   and start with the lesson ``Desktop Basics.'' 
 
 
% \begin{enumerate}[(a)]
% \item Use a built-in MATLAB function (\texttt{csvread} is a good one) to read
%   the January 2013 values into an array $X$ and the January 2014 values into an
%   array $Y$. (The first task has been completed in the template.) \pts{1}

% \item Replace \texttt{??} in line 20 with the appropriate value to find the number of days in January 2013 with an average temperature of 30F. \pts{1}

% \item Now uncomment the other lines to \pts{1} find the probability that the mean temperature in January 2013 was 30F and display your answer using the \texttt{fprintf} statement.

% \item Write \pts{1} a one line statement in your script to compute the probability that the mean temperature in January 2013 was \textit{not} 30F and display your answer using \texttt{fprintf}. \pts{2}

% \item Now compute the probability that the \pts{3} mean daily temperature in January 2014 was greater than 32F but \textit{no more} than 35F. Hint: to find the number of occurrences required, use:
%   \begin{quote}
%     \texttt{nnz(Y > 32 \& Y <= 35)}
%   \end{quote}
% Display your answer using \texttt{fprintf}.


%   \item Create the following plots:
%     \begin{enumerate}[\bf (i)]
%     \item Histogram of \texttt{X} \pts{2}
%     \item Histogram of \texttt{Y} \pts{2}
%     \item Boxplot of \texttt{X} \pts{2}
%     \item Boxplot of \texttt{Y} \pts{2}      
%     \item Scatterplot of \texttt{Y} versus \texttt{X} \pts{2}.
%   \end{enumerate}
%   Make sure that in each plot, the $x$-axis and the $y$-axis are both labeled. Also be sure to include titles for each plot.
%   When you or the TA runs your script, all 5 plots must be displayed simultaneously.
%   Points will be deducted in each case where these instructions are not followed.
%   Consult \texttt{example2.m} from Module 1 for examples of creating scatterplots and labeling axes and plots. \texttt{example1.m} shows how to plot histograms.

% \item Use MATLAB \pts{4} to find the mean and standard deviation of the temperatures in both X and Y. Display your answers using two \texttt{fprintf} statements. (The first one has been commented out for you. Note the usage of square brackets when breaking up a long quote in MATLAB over multiple lines.)
%   \end{enumerate}

%   \eject

%   \subsection*{Python version}

% \subsubsection*{Tutorial}
%   \textit{Optional}\quad If you have never used Python or Jupyter Notebook before, visit this link:
%   \url{https://jupyter-notebook.readthedocs.io/en/stable/examples/Notebook/Notebook%20Basics.html} and start with the lesson ``Notebook Basics.''
 

% \subsubsection*{Questions}
% \begin{enumerate}[(a)]
% \item Use a  Python function (\texttt{pd.read\_csv} is a good one) to read
%   the January 2013 values into an array $X$ and the January 2014 values into an
%   array $Y$. (The first task has been completed in the template.) \pts{1}

% \item Replace \texttt{??} with the appropriate value to find the number of days in January 2013 with an average temperature of 30F. \pts{1}

% \item Now uncomment the other lines to \pts{1} find the probability that the mean temperature in January 2013 was 30F and display your answer using the \texttt{print} statement.

% \item Write   a one line statement in your notebook to compute the probability that the mean temperature in January 2013 was \textit{not} 30F and display your answer using \texttt{print}. \pts{2}

% \item Now compute the probability that the \pts{3} mean daily temperature in January 2014 was greater than 32F but \textit{no more} than 35F. Hint: to find the number of occurrences required, use:
%   \begin{quote}
%      \texttt{np.count\_nonzero((Y > 32) \& (Y <= 35))}
%   \end{quote}
% Display your answer using \texttt{print}.


%   \item Create the following plots:
%     \begin{enumerate}[\bf (i)]
%     \item Histogram of \texttt{X} \pts{2}
%     \item Histogram of \texttt{Y} \pts{2}
%     \item Boxplot of \texttt{X} \pts{2}
%     \item Boxplot of \texttt{Y} \pts{2}
%     \item Scatterplot of \texttt{Y} versus \texttt{X} \pts{2}.
%   \end{enumerate}
%   Make sure that in each plot, the $x$-axis and the $y$-axis are both labeled. Also be sure to include titles for each plot.
%   When you or the grader runs your notebook, all 5 plots must be displayed.
%   Points will be deducted in each case where these instructions are not followed.
%   % Consult \texttt{example2.m} from Module 1 for examples of creating scatterplots and labeling axes and plots. \texttt{example1.m} shows how to plot histograms.

% \item Use Python \pts{4} to find the mean and standard deviation of the temperatures in both X and Y. Display your answers using two \texttt{print} statements. (The first one has been commented out for you.) %Note the usage of square brackets when breaking up a long quote in MATLAB over multiple lines.)
%   \end{enumerate}
 

% \vfill

    
    
% \end{document}

\end{document}

%%% Local Variables:
%%% mode: latex
%%% TeX-master: t
%%% End:
