\documentclass[11pt,twoside]{article}
\usepackage{etex}

\raggedbottom

%geometry (sets margin) and other useful packages
\usepackage{geometry}
\geometry{top=1in, left=1in,right=1in,bottom=1in}
 \usepackage{graphicx,booktabs,calc}
 
\usepackage{listings}


% Marginpar width
%Marginpar width
\newcommand{\pts}[1]{\marginpar{ \small\hspace{0pt} \textit{[#1]} } } 
\setlength{\marginparwidth}{.5in}
%\reversemarginpar
%\setlength{\marginparsep}{.02in}

 
%\usepackage{cmbright}lstinputlisting
%\usepackage[T1]{pbsi}


\usepackage{chngcntr,mathtools}
\counterwithin{figure}{section}
\numberwithin{equation}{section}

%\usepackage{listings}

%AMS-TeX packages
\usepackage{amssymb,amsmath,amsthm} 
\usepackage{bm}
\usepackage[mathscr]{eucal}
\usepackage{colortbl}
\usepackage{color}


\usepackage{subfigure,hyperref,enumerate,polynom,polynomial}
\usepackage{multirow,minitoc,fancybox,array,multicol}

\definecolor{slblue}{rgb}{0,.3,.62}
\hypersetup{
    colorlinks,%
    citecolor=blue,%
    filecolor=blue,%
    linkcolor=blue,
    urlcolor=slblue
}

%%%TIKZ
\usepackage{tikz}

\usepackage{pgfplots}
\pgfplotsset{compat=newest}

\usetikzlibrary{arrows,shapes,positioning}
\usetikzlibrary{decorations.markings}
\usetikzlibrary{shadows}
\usetikzlibrary{patterns}
%\usetikzlibrary{circuits.ee.IEC}
\usetikzlibrary{decorations.text}
% For Sagnac Picture
\usetikzlibrary{%
    decorations.pathreplacing,%
    decorations.pathmorphing%
}

\tikzstyle arrowstyle=[black,scale=2]
\tikzstyle directed=[postaction={decorate,decoration={markings,
    mark=at position .65 with {\arrow[arrowstyle]{stealth}}}}]
\tikzstyle reverse directed=[postaction={decorate,decoration={markings,
    mark=at position .65 with {\arrowreversed[arrowstyle]{stealth};}}}]
\tikzstyle dir=[postaction={decorate,decoration={markings,
    mark=at position .98 with {\arrow[arrowstyle]{latex}}}}]
\tikzstyle rev dir=[postaction={decorate,decoration={markings,
    mark=at position .98 with {\arrowreversed[arrowstyle]{latex};}}}]

\usepackage{ctable}

%
%Redefining sections as problems
%
\makeatletter
\newenvironment{exercise}{\@startsection 
	{section}
	{1}
	{-.2em}
	{-3.5ex plus -1ex minus -.2ex}
    	{1.3ex plus .2ex}
    	{\pagebreak[3]%forces pagebreak when space is small; use \eject for better results
	\large\bf\noindent{Exercise 1.\hspace{-1.5ex} }
	}
	}
	%{\vspace{1ex}\begin{center} \rule{0.3\linewidth}{.3pt}\end{center}}
	%\begin{center}\large\bf \ldots\ldots\ldots\end{center}}
\makeatother

%
%Fancy-header package to modify header/page numbering 
%
\usepackage{fancyhdr}
\pagestyle{fancy}
%\addtolength{\headwidth}{\marginparsep} %these change header-rule width
%\addtolength{\headwidth}{\marginparwidth}
%\fancyheadoffset{30pt}
%\fancyfootoffset{30pt}
\fancyhead[LO,RE]{\small Oke}
\fancyhead[RO,LE]{\small Page \thepage} 
\fancyfoot[RO,LE]{\small PS 4} 
\fancyfoot[LO,RE]{\small \scshape CEE 260/MIE 273} 
\cfoot{} 
\renewcommand{\headrulewidth}{0.1pt} 
\renewcommand{\footrulewidth}{0.1pt}
%\setlength\voffset{-0.25in}
%\setlength\textheight{648pt}


\usepackage{paralist}

\newcommand{\osn}{\oldstylenums}
\newcommand{\lt}{\left}
\newcommand{\rt}{\right}
\newcommand{\pt}{\phantom}
\newcommand{\tf}{\therefore}
\newcommand{\?}{\stackrel{?}{=}}
\newcommand{\fr}{\frac}
\newcommand{\dfr}{\dfrac}
\newcommand{\ul}{\underline}
\newcommand{\tn}{\tabularnewline}
\newcommand{\nl}{\newline}
\newcommand\relph[1]{\mathrel{\phantom{#1}}}
\newcommand{\cm}{\checkmark}
\newcommand{\ol}{\overline}
\newcommand{\rd}{\color{red}}
\newcommand{\bl}{\color{blue}}
\newcommand{\pl}{\color{purple}}
\newcommand{\og}{\color{orange!90!black}}
\newcommand{\gr}{\color{green!40!black}}
\newcommand{\nin}{\noindent}
\newcommand{\la}{\lambda}
\renewcommand{\th}{\theta}
\newcommand*\circled[1]{\tikz[baseline=(char.base)]{
            \node[shape=circle,draw,thick,inner sep=1pt] (char) {\small #1};}}

\newcommand{\bc}{\begin{compactenum}[\quad--]}
\newcommand{\ec}{\end{compactenum}}

\newcommand{\n}{\\[2mm]}
%% GREEK LETTERS
\newcommand{\al}{\alpha}
\newcommand{\gam}{\gamma}
\newcommand{\eps}{\epsilon}
\newcommand{\sig}{\sigma}

\newcommand{\p}{\partial}
\newcommand{\pd}[2]{\frac{\partial{#1}}{\partial{#2}}}
\newcommand{\dpd}[2]{\dfrac{\partial{#1}}{\partial{#2}}}
\newcommand{\pdd}[2]{\frac{\partial^2{#1}}{\partial{#2}^2}}
\newcommand{\mr}{\mathbb{R}}
\newcommand{\xs}{x^{*}}
\newenvironment{solution}
{\medskip\par\quad\quad\begin{minipage}[c]{.8\textwidth}\gr}{\medskip\end{minipage}}


\pgfmathdeclarefunction{poiss}{1}{%
  \pgfmathparse{(#1^x)*exp(-#1)/(x!)}%
  }

\pgfmathdeclarefunction{gauss}{2}{%
  \pgfmathparse{1/(#2*sqrt(2*pi))*exp(-((x-#1)^2)/(2*#2^2))}%
}

\pgfmathdeclarefunction{expo}{2}{%
  \pgfmathparse{#1*exp(-#1*#2)}%
}

\usetikzlibrary{math}

% https://tex.stackexchange.com/questions/461758/asymmetric-distribution-gauss-curve
\tikzmath{%
  function h1(\x, \lx) { return (9*\lx + 3*((\lx)^2) + ((\lx)^3)/3 + 9); };
  function h2(\x, \lx) { return (3*\lx - ((\lx)^3)/3 + 4); };
  function h3(\x, \lx) { return (9*\lx - 3*((\lx)^2) + ((\lx)^3)/3 + 7); };
  function skewnorm(\x, \l) {
    \x = (\l < 0) ? -\x : \x;
    \l = abs(\l);
    \e = exp(-(\x^2)/2);
    return (\l == 0) ? 1 / sqrt(2 * pi) * \e: (
      (\x < -3/\l) ? 0 : (
      (\x < -1/\l) ? \e / (8 * sqrt(2 * pi)) * h1(\x, \x*\l) : (
      (\x <  1/\l) ? \e / (4 * sqrt(2 * pi)) * h2(\x, \x*\l) : (
      (\x <  3/\l) ? \e / (8 * sqrt(2 * pi)) * h3(\x, \x*\l) : (
      sqrt(2/pi) * \e)))));
  };
}

\def\cdf(#1)(#2)(#3){0.5*(1+(erf((#1-#2)/(#3*sqrt(2)))))}%
% to be used: \cdf(x)(mean)(variance)

\DeclareMathOperator{\CDF}{cdf}
\tikzset{
    declare function={
        normcdf(\x,\m,\s)=1/(1 + exp(-0.07056*((\x-\m)/\s)^3 - 1.5976*(\x-\m)/\s));
    }
}
%%%%%%%%%%%%%%%%%%%%%%%%%%%%%%%%%%%%%%%%%%%%%%%%%%%
%%%%%%%%%%%%%%%%%%%%%%%%%%%%%%%%%%%%%%%%%%%%%%%%%%%

\begin{document}

\lstset{language=C++,
                basicstyle=\tiny\ttfamily,
                keywordstyle=\color{blue}\ttfamily,
                stringstyle=\color{red}\ttfamily,
                commentstyle=\color{gray}\ttfamily,
                morecomment=[l][\color{gray}]{\#}
}


\thispagestyle{empty}


\nin{\LARGE Problem Set 4 {\gr }}\hfill{\bf Prof. Oke}

\medskip\hrule\medskip

\nin {\small CEE 260/MIE 273: Probability \& Statistics in Civil Engineering
\hfill\textit{ 9.23.2025}}

\bigskip

\nin{\it \textbf{Due September 30, 2025 at 11:59 PM as PDF uploaded via Gradescope.}
  For ease of grading, use this document as your template (either print/write/upload, use \LaTeX or edit using a tablet device)
    \textbf{Show as much work as possible in order to get FULL credit.}
    There are 7 problems with a total of 30 points available.
    \textbf{Important:} If you use Python for any probability computations, briefly write/include the statements you
    used to arrive at your answers. If instead you use probability tables, note this in the respective solution, as well.
}\\


\section*{Problem 1 \textit{(5 points)}}

Respond ``T'' ({\it True})  or  ``F'' (\textit{False}) to the following statements. Use the boxes provided. Each response is worth 1 point.

\begin{enumerate}[\bf (i)]
\item \hfill
  \begin{minipage}{.1\linewidth}
    \framebox(40,40){ \gr  }
  \end{minipage}\quad
  \begin{minipage}{.85\linewidth}
    If $Z$ represents the standard normal variable, then the mean of $Z$ is 1.
   \end{minipage}
  
  \smallskip
  
\item \hfill
  \begin{minipage}{.1\linewidth}
    \framebox(40,40){\gr }
  \end{minipage}\quad
  \begin{minipage}{.85\linewidth}
    To standardize a normally distributed random variable, we find the difference from its mean and divide the result by
    its standard deviation.   
  \end{minipage}

  \smallskip
  
\item \hfill
  \begin{minipage}{.1\linewidth}
    \framebox(40,40){\gr  }
  \end{minipage}\quad
  \begin{minipage}{.85\linewidth}
    The lifetime of a lightbulb is normally distributed with $\mu=1400$ hrs and $\sigma=200$ hrs.
    The 20th percentile of the lifetimes is approximately 1140 hrs.
   \end{minipage}

  \smallskip
  
\item \hfill
  \begin{minipage}{.1\linewidth}
    \framebox(40,40){\gr  }
  \end{minipage}\quad
  \begin{minipage}{.85\linewidth}
    The lifetime of a lightbulb is normally distributed with $\mu=1400$ hrs and $\sigma=200$ hrs.
    The probability that the lightbulb will last more than 1700 hrs is 0.933.
  \end{minipage}

  \smallskip

% \item \hfill
%   \begin{minipage}{.1\linewidth}
%     \framebox(40,40){\gr }
%   \end{minipage}\quad
%   \begin{minipage}{.85\linewidth}
%     If the natural logarithm of a random variable is normally distributed with parameters $\mu$ and $\sigma$, then the
%     variable is lognormally distributed with the same parameters.
%    \end{minipage}
  
%   \smallskip
  
  \item \hfill
    \begin{minipage}{.1\linewidth}
      \framebox(40,40){\gr  }
    \end{minipage}\quad
    \begin{minipage}{.85\linewidth}
      The area under a PDF can be less than or equal to 1.
     \end{minipage}
  
  \smallskip
  
% \item \hfill
%   \begin{minipage}{.1\linewidth}
%     \framebox(40,40){\gr  }
%   \end{minipage}\quad
%   \begin{minipage}{.85\linewidth}
%     A binomial distribution with sufficiently large $n$ can be approximated by a normal distribution with $\mu = np$.
%   \end{minipage}

%   \smallskip

  % \item \hfill
  %   \begin{minipage}{.1\linewidth}
  %     \framebox(40,40){\gr  }
  %   \end{minipage}\quad
  %   \begin{minipage}{.85\linewidth}
  %     The mean of a lognormal distribution is equal to or greater than its median.

  %   \end{minipage}
\end{enumerate}

  % \smallskip
  
% \item \hfill
%   \begin{minipage}{.1\linewidth}
%     \framebox(40,40){}
%   \end{minipage}\quad
%   \begin{minipage}{.85\linewidth}
%     A good clustering solution should maximize the total within-cluster variation. %False
%   \end{minipage}

\eject  

\section*{Problem 2 \textit{(2 points)}}

%\textit{Show brief amount of work for partial credit if answer is wrong. Not required however for full credit. }

%\medskip

   

  \begin{enumerate}[\bf (a)]

  \item   Write down the expression of the probability represented by the shaded portion of the normal PDF below. \pts{1} For example, $P(X \le
    2)$. Note that a dashed vertical boundary indicates ``$>$'' or ``$<$,'' while a solid vertical boundary indicates ``$\ge$'' or
    ``$\le$.''

          \begin{figure}[h!]
    \centering
      \begin{tikzpicture}
    \begin{axis}[no markers, domain=0:10, samples=100,
      axis lines*=left, xlabel=$x$, ylabel=$f_X(x)$,,
      height=6cm, width=10cm,
      %xtick={-3, -2, -1, 0, 1, 2, 3},
      %ytick=\empty,
      enlargelimits=false, clip=false, axis on top,
      grid style={line width=.1pt, draw=gray},
      yticklabel style={
            /pgf/number format/fixed,
            /pgf/number format/fixed zerofill,
            /pgf/number format/precision=2
          },        
      grid = major]
      \addplot [ultra thick, domain=-2:12] {gauss(5,2)};
      \addplot [thick,draw=none, pattern=north west lines,  domain=6:12] {gauss(5,2)} \closedcycle;
      \addplot [dashed, ultra thick,] coordinates {(6,0) (6,{gauss(5,2)})};
    \end{axis}
  \end{tikzpicture}

\end{figure}

      \vspace{2ex}
  \begin{minipage}[]{.1\linewidth}
    {\bf Answer:}
  \end{minipage}\qquad
  \begin{minipage}[]{.8\linewidth}
    \framebox(390,40){\phantom{\Huge t}  }     
  \end{minipage}

  \bigskip
  
  \item   Write down the expression of the probability represented by the shaded portion of the normal PDF below. \pts{1} For example, $P(X \le
    2)$. Note that a dashed vertical boundary indicates ``$>$'' or ``$<$,'' while a solid vertical boundary indicates ``$\ge$'' or
    ``$\le$.''

    \begin{figure}[h!]
      \centering
      \begin{tikzpicture}
        \begin{axis}[no markers, domain=0:10, samples=100,
          axis lines*=left, xlabel=$x$, ylabel=$f_X(x)$,,
          height=6cm, width=10cm,
          % xtick={-3, -2, -1, 0, 1, 2, 3},
          % ytick=\empty,
          enlargelimits=false, clip=false, axis on top,
          grid style={line width=.1pt, draw=gray},
          yticklabel style={
            /pgf/number format/fixed,
            /pgf/number format/fixed zerofill,
            /pgf/number format/precision=2
          },        
          grid = major]
          \addplot [ultra thick, domain=-3:3] {gauss(0,1)};
          \addplot [thick,draw=none, pattern=north west lines,  domain=-3:0] {gauss(0,1)} \closedcycle;
          \addplot [ultra thick] coordinates {(0,0) (0,{gauss(0,1)})};
        \end{axis}
      \end{tikzpicture}

    \end{figure}


      \vspace{2ex}
  \begin{minipage}[]{.1\linewidth}
    {\bf Answer:}
  \end{minipage}\qquad
  \begin{minipage}[]{.8\linewidth}
    \framebox(390,40){\phantom{\Huge t}  }     
  \end{minipage}

  \bigskip
\end{enumerate}

  \eject

\section*{Problem 3 \textit{(2 points)}}
  \begin{enumerate}[\bf (a)]

  
\item Write down the expression of the probability represented by the shaded portion of the normal PDF below. \pts{1}
  For example, $P(X \le 2)$. Note that a dashed vertical boundary indicates ``$>$'' or ``$<$,'' while a solid vertical
  boundary indicates ``$\ge$'' or ``$\le$.''


    \begin{figure}[h!]
      \centering
      \begin{tikzpicture}
        \begin{axis}[no markers, domain=0:10, samples=100,
          axis lines*=left, xlabel=$x$, ylabel=$f_X(x)$,,
          height=6cm, width=10cm,
          % xtick={-3, -2, -1, 0, 1, 2, 3},
          % ytick=\empty,
          enlargelimits=false, clip=false, axis on top,
          grid style={line width=.1pt, draw=gray},
          yticklabel style={
            /pgf/number format/fixed,
            /pgf/number format/fixed zerofill,
            /pgf/number format/precision=2
          },        
          grid = major]
          \addplot [ultra thick, domain=-3:3] {gauss(0,1)};
          \addplot [thick,draw=none, pattern=north west lines,  domain=-1:2] {gauss(0,1)} \closedcycle;
          \addplot [ultra thick] coordinates {(-1,0) (-1,{gauss(0,1)})};
          \addplot [ultra thick, dashed] coordinates {(2,0) (2,{gauss(0,1)})};
        \end{axis}
      \end{tikzpicture}

    \end{figure}

      \vspace{2ex}
  \begin{minipage}[]{.1\linewidth}
    {\bf Answer:}
  \end{minipage}\qquad
  \begin{minipage}[]{.8\linewidth}
    \framebox(390,40){\phantom{\Huge t}  }     
  \end{minipage}

  \bigskip
  \bigskip
  \bigskip
  
\item Below is the PDF of a given normal distribution. \pts{1} What is the median of this distribution?

            \begin{figure}[h!]
    \centering
      \begin{tikzpicture}
    \begin{axis}[no markers, domain=-4:12, samples=100,
      axis lines*=left, xlabel=$x$, ylabel=$f_X(x)$,,
      height=6cm, width=10cm,
      %xtick={-3, -2, -1, 0, 1, 2, 3},
      %ytick=\empty,
      enlargelimits=false, clip=false, axis on top,
      grid style={line width=.1pt, draw=gray},
      yticklabel style={
            /pgf/number format/fixed,
            /pgf/number format/fixed zerofill,
            /pgf/number format/precision=2
          },        
      grid = major]
      \addplot [ultra thick, domain=-4:12] {gauss(4,3)};
%      \addplot [thick,draw=none, pattern=north west lines,  domain=6:12] {gauss(4,3)} \closedcycle;
%      \addplot [dashed, ultra thick,] coordinates {(6,0) (6,{gauss(5,2)})};
    \end{axis}
  \end{tikzpicture}

\end{figure}

  \vspace{2ex}
  \begin{minipage}[]{.1\linewidth}
    {\bf Answer:}
  \end{minipage}\qquad
  \begin{minipage}[]{.8\linewidth}
    \framebox(390,40){\phantom{\Huge t}   }     
  \end{minipage}

\end{enumerate}

  \eject

\section*{Problem 4 \textit{(4 points)}}
{\it In the following problems, show how you arrive at the answer on the graph.}


  \begin{enumerate}[\bf (a)]
 \item Below is the CDF of a given normal distribution. \pts{2} What is the mean of this distribution?

            \begin{figure}[h!]
    \centering
      \begin{tikzpicture}[scale=.8]
    \begin{axis}[no markers, domain=-6:6, samples=100,
      axis lines*=left, xlabel=$x$, ylabel=$F_X(x)$,,
      height=8cm, width=10cm,ymax=1,
      %xtick={-3, -2, -1, 0, 1, 2, 3},
      ytick={0,0.1,...,1},
      enlargelimits=false, clip=false, axis on top,
      grid style={line width=.1pt, draw=gray},
      yticklabel style={
            /pgf/number format/fixed,
            /pgf/number format/fixed zerofill,
            /pgf/number format/precision=2
          },        
      grid = both]
      % addplot [ultra thick, domain=-4:12] {gauss(4,3)};
      \addplot [ultra thick, smooth, black] {normcdf(x,0,2)};
%      \addplot [thick,draw=none, pattern=north west lines,  domain=6:12] {gauss(4,3)} \closedcycle;
%      \addplot [dashed, ultra thick,] coordinates {(6,0) (6,{gauss(5,2)})};
    \end{axis}
  \end{tikzpicture}

\end{figure}

  \vspace{2ex}
  \begin{minipage}[]{.1\linewidth}
    {\bf Answer:}
  \end{minipage}\qquad
  \begin{minipage}[]{.8\linewidth}
    \framebox(390,40){\phantom{\Huge t} }     
  \end{minipage}


  \bigskip

   \item Below is the CDF of a given normal distribution. \pts{2} Estimate the probability $P(X > 3.5)$.

            \begin{figure}[h!]
    \centering
      \begin{tikzpicture}
    \begin{axis}[no markers, domain=-2:8, samples=100,
      axis lines*=left, xlabel=$x$, ylabel=$F_X(x)$,,
      height=8cm, width=17cm,ymax=1,
      xtick={-2,-1.5, -1, ..., 9},
      ytick={0,0.1,...,1.1},
      enlargelimits=false, clip=false, axis on top,
      grid style={line width=.1pt, draw=gray},
      yticklabel style={
            /pgf/number format/fixed,
            /pgf/number format/fixed zerofill,
            /pgf/number format/precision=2
          },        
      grid = both]
      % addplot [ultra thick, domain=-4:12] {gauss(4,3)};
      \addplot [ultra thick, smooth, black] {normcdf(x,3,1)};
%      \addplot [thick,draw=none, pattern=north west lines,  domain=6:12] {gauss(4,3)} \closedcycle;
%      \addplot [dashed, ultra thick,] coordinates {(6,0) (6,{gauss(5,2)})};
    \end{axis}
  \end{tikzpicture}

\end{figure}

  \vspace{2ex}
  \begin{minipage}[]{.1\linewidth}
    {\bf Answer:}
  \end{minipage}\qquad
  \begin{minipage}[]{.8\linewidth}
    \framebox(390,40){\phantom{\Huge t}   }     
  \end{minipage}

\end{enumerate}

  \eject

\section*{Problem 5 \textit{(4 points)}}
{\it In the following problems, show how you arrive at the answer on the graph.}

  \begin{enumerate}[\bf (a)]

     \item Below is the CDF of a given normal distribution. \pts{2} Estimate the quantity $F_{X}^{-1}(0.3)$.

            \begin{figure}[h!]
    \centering
      \begin{tikzpicture}[scale=.8]
    \begin{axis}[no markers, domain=-2:8, samples=100,
      axis lines*=left, xlabel=$x$, ylabel=$F_X(x)$,,
      height=8cm, width=17cm,ymax=1,
      xtick={-2,-1.5, -1, ..., 9},
      ytick={0,0.1,...,1.1},
      enlargelimits=false, clip=false, axis on top,
      grid style={line width=.1pt, draw=gray},
      yticklabel style={
            /pgf/number format/fixed,
            /pgf/number format/fixed zerofill,
            /pgf/number format/precision=2
          },        
      grid = both]
      % addplot [ultra thick, domain=-4:12] {gauss(4,3)};
      \addplot [ultra thick, smooth, black] {normcdf(x,3,1)};
%      \addplot [thick,draw=none, pattern=north west lines,  domain=6:12] {gauss(4,3)} \closedcycle;
%      \addplot [dashed, ultra thick,] coordinates {(6,0) (6,{gauss(5,2)})};
    \end{axis}
  \end{tikzpicture}

\end{figure}

  \vspace{2ex}
  \begin{minipage}[]{.1\linewidth}
    {\bf Answer:}
  \end{minipage}\qquad
  \begin{minipage}[]{.8\linewidth}
    \framebox(390,40){\phantom{\Huge t}  }     
  \end{minipage}

  \bigskip
  \bigskip
  
     \item Below is the CDF of a given normal distribution. \pts{2} Estimate the first quartile of the distribution.

            \begin{figure}[h!]
    \centering
      \begin{tikzpicture}
    \begin{axis}[no markers, domain=-1:3, samples=100,
      axis lines*=left, xlabel=$x$, ylabel=$F_X(x)$,,
      height=8cm, width=16cm,ymax=1,
      xtick={-1, -.5, ..., 3},
      ytick={0,0.125,.25,...,1},
      enlargelimits=false, clip=false, axis on top,
      grid style={line width=.1pt, draw=gray!40},
      major grid style={line width=.7pt, draw=gray},
      yticklabel style={
            /pgf/number format/fixed,
            /pgf/number format/fixed zerofill,
            /pgf/number format/precision=3
          },        
      grid = both,minor tick num=4]
      % addplot [ultra thick, domain=-4:12] {gauss(4,3)};
      \addplot [ultra thick, smooth, black] {normcdf(x,1,.5)};
%      \addplot [thick,draw=none, pattern=north west lines,  domain=6:12] {gauss(4,3)} \closedcycle;
%      \addplot [dashed, ultra thick,] coordinates {(6,0) (6,{gauss(5,2)})};
    \end{axis}
  \end{tikzpicture}

\end{figure}

  \vspace{2ex}
  \begin{minipage}[]{.1\linewidth}
    {\bf Answer:}
  \end{minipage}\qquad
  \begin{minipage}[]{.8\linewidth}
    \framebox(390,40){\phantom{\Huge t}   }     
  \end{minipage}

\end{enumerate}
%   \eject

% \item The \pts{1} graph below is the PDF of an exponentially distributed random variable $T$, given by $f_T(t) = \lambda
%   e^{-\la t}$. What is the value of the parameter $\la$?
 
%   \begin{figure}[h!]
%     \centering
%     \begin{tikzpicture}
%       \begin{axis}[no marks,
%         samples = 100,
%         axis x line=center,
%         axis y line=center,
%         xtick={0,1,2,...,8},
%         ytick={0,0.25,...,1.75},
%         domain = 0:8,
%         xlabel={$t$},
%         ylabel={$f_T(t)$},
%         xlabel style={right},
%         ylabel style={above },
%         ymax=1.8,
%         xmax=8.2,
%         x post scale=1.7, enlargelimits=false,
%         grid=both,
%         yticklabel style={
%           /pgf/number format/fixed,
%           /pgf/number format/fixed zerofill,
%           /pgf/number format/precision=2
%         }
%         ]
%         \addplot+[draw=black, ultra thick,opacity=1] {expo(1.5,x)}; %\addlegendentry{\large $\bm{f_{T_X}}$}
%         \addplot+[draw=black,thick, pattern=north east lines, opacity=1, domain=0:1] {expo(1.5,x)} \closedcycle; 
%       \end{axis}
%     \end{tikzpicture}
%   \end{figure}

%   \vspace{2ex}
%   \begin{minipage}[]{.1\linewidth}
%     {\bf Answer:}
%   \end{minipage}\qquad
%   \begin{minipage}[]{.8\linewidth}
%     \framebox(390,40){\phantom{\Huge t}  }     
%   \end{minipage}

%   \bigskip

%   \bigskip

%   \bigskip
  
% \item What is the probability represented by the shaded area in the figure in part \textbf{(i)}? \pts{1} (A numeric value
%   is expected here, not just a symbolic expression.)

%   \vspace{2ex}
%     \begin{minipage}[]{.1\linewidth}
%     {\bf Answer:}
%   \end{minipage}\qquad
%   \begin{minipage}[]{.8\linewidth}
%     \framebox(390,40){\phantom{\Huge t}   }     
%   \end{minipage}

  
  
 

  \eject
  \section*{Problem 6 \textit{Standard Normal Distribution (8 points)}}
What percent of a standard normal distribution $\mathcal{N} (\mu = 0, \sigma = 1)$ is found in each region? Sketch the accompanying curve along with your answer.
\begin{enumerate}[(a)]
\item $Z < - 1.35$ \marginpar{\it [2 pts]}
  \vspace{30ex}
  
% \item $Z > 1.48$
%   \vspace{20ex}

\item $ -0.4 < Z < 1.5$ \marginpar{\it [3 pts]}
  \vspace{30ex}

\item $|Z| > 2$ \marginpar{\it [3 pts]}
  \vspace{30ex}
\end{enumerate}


\eject

\section*{Problem 7  \textit{Normal Distribution (5 points)}}
The average daily high temperature in June in LA is 77$^{\circ}$F with a standard deviation of $5^{\circ}$F.
Suppose that the temperatures in June closely follow a normal distribution.

\begin{enumerate}[\bf (a)]
\item What is the probability of observing an 83$^{\circ}$F temperature \pts{2} or higher in LA during a randomly chosen day in
  June?
 \vspace{40ex}

\item How cool are the coldest 10\% of the days \pts{3} (days with lowest average high temperature) during June in LA?
 
\end{enumerate}

%\eject

% \section*{Problem 4: Lognormal Distribution \textit{(5 points)}}
% The mean, median and variance of lognormal variable $Y$ with parameters $\mu$ and $\sigma^{2}$ are given by:
% \begin{align}
%   \mathbb{E}(Y) &= e^{\lt(\mu + \fr{\sigma^{2}}{2}\rt)} \\[2mm]
%   \text{Median}(Y)&= e^{\mu}\\[2mm]
%   \mathbb{V}(Y) &= e^{\lt(\sigma^{2}-1\rt)}e^{\lt(2\mu + \sigma^{2}\rt)}
% \end{align}
% Given that the lifetime in days of an electronic component is lognormally distributed with $\mu = 1.1$ and $\sigma = 0.5$.

% \begin{enumerate}[\bf (a)]
% \item Find the mean lifetime of the component. \pts{1}
%  \vspace{20ex}
 
  
% \item Find the median lifetime of the component. \pts{1}
%  \vspace{20ex}
  
% \item Find the probability that a component lasts between 3 and 5 days. \pts{3}
%   \eject
%   ~
% \end{enumerate}
\end{document}

%%% Local Variables:
%%% mode: latex
%%% TeX-master: t
%%% End:
