\documentclass[11pt,twoside]{article}
\usepackage{etex}

\raggedbottom

%geometry (sets margin) and other useful packages
\usepackage{geometry}
\geometry{top=1in, left=1in,right=1in,bottom=1in}
 \usepackage{graphicx,booktabs,calc}
 
\usepackage{listings}


% Marginpar width
%Marginpar width
\newcommand{\pts}[1]{\marginpar{ \small\hspace{0pt} \textit{[#1]} } } 
\setlength{\marginparwidth}{.5in}
%\reversemarginpar
%\setlength{\marginparsep}{.02in}

 
%\usepackage{cmbright}lstinputlisting
%\usepackage[T1]{pbsi}


\usepackage{chngcntr,mathtools}
%\counterwithin{figure}{section}
%\numberwithin{equation}{section}

%\usepackage{listings}

%AMS-TeX packages
\usepackage{amssymb,amsmath,amsthm} 
\usepackage{bm}
\usepackage[mathscr]{eucal}
\usepackage{colortbl}
\usepackage{color}


\usepackage{subfig,hyperref,enumerate,polynom,polynomial}
\usepackage{multirow,minitoc,fancybox,array,multicol}

\definecolor{slblue}{rgb}{0,.3,.62}
\hypersetup{
    colorlinks,%
    citecolor=blue,%
    filecolor=blue,%
    linkcolor=blue,
    urlcolor=slblue
}

%%%TIKZ
\usepackage{tikz}

\usepackage{pgfplots}
\pgfplotsset{compat=newest}

\usetikzlibrary{arrows,shapes,positioning}
\usetikzlibrary{decorations.markings}
\usetikzlibrary{shadows}
\usetikzlibrary{patterns}
%\usetikzlibrary{circuits.ee.IEC}
\usetikzlibrary{decorations.text}
% For Sagnac Picture
\usetikzlibrary{%
    decorations.pathreplacing,%
    decorations.pathmorphing%
}

\tikzstyle arrowstyle=[black,scale=2]
\tikzstyle directed=[postaction={decorate,decoration={markings,
    mark=at position .65 with {\arrow[arrowstyle]{stealth}}}}]
\tikzstyle reverse directed=[postaction={decorate,decoration={markings,
    mark=at position .65 with {\arrowreversed[arrowstyle]{stealth};}}}]
\tikzstyle dir=[postaction={decorate,decoration={markings,
    mark=at position .98 with {\arrow[arrowstyle]{latex}}}}]
\tikzstyle rev dir=[postaction={decorate,decoration={markings,
    mark=at position .98 with {\arrowreversed[arrowstyle]{latex};}}}]

\usepackage{ctable}

%
%Redefining sections as problems
%
\makeatletter
\newenvironment{exercise}{\@startsection 
	{section}
	{1}
	{-.2em}
	{-3.5ex plus -1ex minus -.2ex}
    	{1.3ex plus .2ex}
    	{\pagebreak[3]%forces pagebreak when space is small; use \eject for better results
	\large\bf\noindent{Part 1.\hspace{-1.5ex} }
	}
	}
	%{\vspace{1ex}\begin{center} \rule{0.3\linewidth}{.3pt}\end{center}}
	%\begin{center}\large\bf \ldots\ldots\ldots\end{center}}
\makeatother

%
%Fancy-header package to modify header/page numbering 
%
\usepackage{fancyhdr}
\pagestyle{fancy}
%\addtolength{\headwidth}{\marginparsep} %these change header-rule width
%\addtolength{\headwidth}{\marginparwidth}
%\fancyheadoffset{30pt}
%\fancyfootoffset{30pt}
\fancyhead[LO,RE]{\small Oke}
\fancyhead[RO,LE]{\small Page \thepage} 
\fancyfoot[RO,LE]{\small Project Presentation Guide} 
\fancyfoot[LO,RE]{\small \scshape CEE 697M} 
\cfoot{} 
\renewcommand{\headrulewidth}{0.1pt} 
\renewcommand{\footrulewidth}{0.1pt}
%\setlength\voffset{-0.25in}
%\setlength\textheight{648pt}


\usepackage{paralist}

\newcommand{\osn}{\oldstylenums}
\newcommand{\lt}{\left}
\newcommand{\rt}{\right}
\newcommand{\pt}{\phantom}
\newcommand{\tf}{\therefore}
\newcommand{\?}{\stackrel{?}{=}}
\newcommand{\fr}{\frac}
\newcommand{\dfr}{\dfrac}
\newcommand{\ul}{\underline}
\newcommand{\tn}{\tabularnewline}
\newcommand{\nl}{\newline}
\newcommand\relph[1]{\mathrel{\phantom{#1}}}
\newcommand{\cm}{\checkmark}
\newcommand{\ol}{\overline}
\newcommand{\rd}{\color{red}}
\newcommand{\bl}{\color{blue}}
\newcommand{\pl}{\color{purple}}
\newcommand{\og}{\color{orange!90!black}}
\newcommand{\gr}{\color{green!40!black}}
\newcommand{\nin}{\noindent}
\newcommand{\la}{\lambda}
\renewcommand{\th}{\theta}
\newcommand*\circled[1]{\tikz[baseline=(char.base)]{
            \node[shape=circle,draw,thick,inner sep=1pt] (char) {\small #1};}}

\newcommand{\bc}{\begin{compactenum}[\quad--]}
\newcommand{\ec}{\end{compactenum}}

\newcommand{\n}{\\[2mm]}
%% GREEK LETTERS
\newcommand{\al}{\alpha}
\newcommand{\gam}{\gamma}
\newcommand{\eps}{\epsilon}
\newcommand{\sig}{\sigma}

\newcommand{\p}{\partial}
\newcommand{\pd}[2]{\frac{\partial{#1}}{\partial{#2}}}
\newcommand{\dpd}[2]{\dfrac{\partial{#1}}{\partial{#2}}}
\newcommand{\pdd}[2]{\frac{\partial^2{#1}}{\partial{#2}^2}}
\newcommand{\mr}{\mathbb{R}}
\newcommand{\xs}{x^{*}}
\newenvironment{solution}
{\medskip\par\quad\quad\begin{minipage}[c]{.8\textwidth}}{\medskip\end{minipage}}

\newcommand{\nmfr}[3]{\Phi\left(\frac{{#1} - {#2}}{#3}\right)}
 
%%%%%%%%%%%%%%%%%%%%%%%%%%%%%%%%%%%%%%%%%%%%%%%%%%%
%%%%%%%%%%%%%%%%%%%%%%%%%%%%%%%%%%%%%%%%%%%%%%%%%%%

\begin{document}

\lstset{language=C++,
                basicstyle=\tiny\ttfamily,
                keywordstyle=\color{blue}\ttfamily,
                stringstyle=\color{red}\ttfamily,
                commentstyle=\color{gray}\ttfamily,
                morecomment=[l][\color{gray}]{\#}
}


\thispagestyle{empty}


\nin{\LARGE Project Guidelines}\hfill{\bf Prof. Oke}

\medskip\hrule\medskip

\nin {\small CEE 697M:  Data Mining and Machine Learning for Engineers
\hfill\textit{ 05.17.2023}}

%\nin{\it \small Due April 10, 2020 at 11:59PM. Submit on \href{https://moodle.umass.edu/course/view.php?id=64661}{Moodle}}.\\

% \nin The standard problems are worth a total of \textbf{100 points}. While 25 \textbf{extra credit points} are available from the problems marked [EC], as well as those in the ``FURTHER PROBLEMS'' section, only a maximum of \textbf{15} extra credit points will be awarded.

\bigskip

\noindent Please use the following guidelines as you work on your project and prepare your presentations (date TBD).


\subsubsection*{Format}
I will email the order of presentations in an Announcement a few days before.
You will have \textbf{10 minutes} for your presentation [if you are presenting solo]; and \textbf{14 minutes} if you are in a group.
Following each presentation, a few minutes will be allotted for questions. (Please keep responses brief and to the point.)

\bigskip

\subsubsection*{Structure}
Your presentation should have the following sections. You may find it helpful to also have an Outline slide. For each item, the number of \textit{recommended}
slides  has been placed in square brackets (note that these are recommendations, and not strict slide number requirements).
You are free to reorganize the order of each section within reason, as well as increase/reduce the number of slides as you see fit.
However, keep in mind your time constraints.

\begin{itemize}
\item \textbf{Introduction and/or Motivation} Background of your project (Why is the problem important?  What has been done before, and what is yet unknown?
  Why do you care about it? Why should your audience care?) [1 Slide]
\item \textbf{Objectives:} State the research questions/aims of your work. (What does your project accomplish?) [1 Slide]
\item \textbf{Data:} Describe your data and sources. [1 Slide]
\item \textbf{Methods:} Specify your model structure[s] (use equations where feasible, e.g.\ splines, and/or diagrams, e.g.\ trees/neural networks). [2 Slides]
\item \textbf{Results:} Show estimates and results, including performance/errors. (What did you find? How good is/are your model[s]?) [2 Slides]
\item \textbf{Conclusion:} Summarize what you did and what you discovered. Briefly state any limitations of your approach and potential future work. [1 Slide]
\item \textbf{Author Contributions:} For those who worked in a group, include 1 slide stating how each member contributed to the project.
\end{itemize}
Use your slide real-estate wisely. \textit{You may also wish to include slides in an appendix}, which you may refer to in response to post-presentation questions.



% \newpage
% \begin{exercise}{\it Variable estimation}
% Choose one of the variable as your dependent variable.
% Using techniques learned in multivariate linear regression, study the dependence of the dependent variable with other relevant variables.
% Justify your explanatory variable selection as well as the fitness of your model (or models).
% Use also ridge regression or kernel density estimation.
% Use plots and tables to support your responses.
% \end{exercise}


% \begin{exercise}{\it Dimensionality reduction}
%   Apply Principal Components Analysis you learned in class.Clearly state your reasoning.
%   Indicate how well the projected data are representative of the original.
%   Describe your results, using tables and plots to show how the indicators contribute to each factor or component.
%   Justify why the factors/components make sense
  
%   % \begin{enumerate}[(a)]
%   % \item 
%   % \end{enumerate}
% \end{exercise}

% \begin{exercise}{\it Clustering}
%   Use any of the clustering approaches discussed in the lectures to assign the cities into characteristic groups.
%   This should be done based on computations in the reduced space obtained in the previous exercise.
%   Clearly show your results using tables and maps.
%   Justify why the clusters/groups you have obtained make sense.
%   Indicate the criteria for measuring the quality of your method of choice.
%   % \begin{enumerate}[(a)]
%   % \item 
%   % \end{enumerate}
% \end{exercise}
% \vfill

\subsubsection*{Submission Instructions}
Following the class presentation, I will email you with comments.
You should address any suggested corrections prior to the submission of your final presentation (submit as PDF), which will be due a few days afterward (TBD).
Along with your PDF, you are also required to include the code you used in generating your results either as an R script/package, Python script/package or Jupyter Lab/Notebook.
Your code should be portable. Thus, you should include your data folder in your submission. Your paths should all be relative. \textit{(I will not necessarily be running your code; so if your dataset is too large to upload, a small sample will suffice.)}
This means your code submission must be an archive (e.g.\ ZIP file). \textbf{Alternatively}, you can submit a link to your code on GitHub or other sharing platform.
Submit this along with your final [revised] presentation.


\subsubsection*{Grading rubric}

\begin{tabular}{m{1in} m{1in} m{3.5in} m{.5in}}\toprule
  \bf Component & \bf Item & \bf Description & \bf Points \\ \midrule
  \textbf{Proposal} & Completeness  & All required elements present & 10 \\\midrule
  \multirow{4}{*}{\textbf{Presentation}} & Organization  & Clear structure and required elements  & 5 \\\cmidrule{2-4}
                &  \multirow{3}{*}{Content} & Adequate model description  & 5\\\cmidrule{3-4}
                & & Adequate data description  & 5\\\cmidrule{3-4}
                & & Thoroughness of results & 5 \\\cmidrule{2-4}
                &  \multirow{2}{*}{Methods}  & At least 2 modeling approaches used & 5 \\ \cmidrule{3-4}
                % \ref{} (or 1 with prior approval) %OR (b) 2 datasets used; 
                         &    & Correctness and appropriateness  & 5 \\\midrule
%  & Question & Adequate responses to post-presentation questions & 5 \\
 % & Final version submitted with corrections addressed (if any)& 5 \\\midrule
  \textbf{Code} &  Organization & Organized and runnable, with reproducible results & 10 \\ \midrule
  \bf Total & & & \bf 50 \\\bottomrule
\end{tabular}

%Slides for your presentation should also be included with your submission as a PDF. Clearly indicate the responsibilities assigned to each person in your group.
 


\end{document}

%%% Local Variables:
%%% mode: latex
%%% TeX-master: t
%%% End:
