\documentclass[11pt,twoside]{article}
\usepackage{etex}

\raggedbottom

%geometry (sets margin) and other useful packages
\usepackage{geometry}
\geometry{top=1in, left=1in,right=1in,bottom=1in}
 \usepackage{graphicx,booktabs,calc}
 
\usepackage{listings}


% Marginpar width
%Marginpar width
\newcommand{\pts}[1]{\marginpar{ \small\hspace{0pt} \textit{[#1]} } } 
\setlength{\marginparwidth}{.5in}
%\reversemarginpar
%\setlength{\marginparsep}{.02in}

 
%\usepackage{cmbright}lstinputlisting
%\usepackage[T1]{pbsi}


\usepackage{chngcntr,mathtools}
%\counterwithin{figure}{section}
%\numberwithin{equation}{section}

%\usepackage{listings}

%AMS-TeX packages
\usepackage{amssymb,amsmath,amsthm} 
\usepackage{bm}
\usepackage[mathscr]{eucal}
\usepackage{colortbl}
\usepackage{color}


\usepackage{subfig,hyperref,enumerate,polynom,polynomial}
\usepackage{multirow,minitoc,fancybox,array,multicol}

\definecolor{slblue}{rgb}{0,.3,.62}
\hypersetup{
    colorlinks,%
    citecolor=blue,%
    filecolor=blue,%
    linkcolor=blue,
    urlcolor=slblue
}

%%%TIKZ
\usepackage{tikz}

\usepackage{pgfplots}
\pgfplotsset{compat=newest}

\usetikzlibrary{arrows,shapes,positioning}
\usetikzlibrary{decorations.markings}
\usetikzlibrary{shadows}
\usetikzlibrary{patterns}
%\usetikzlibrary{circuits.ee.IEC}
\usetikzlibrary{decorations.text}
% For Sagnac Picture
\usetikzlibrary{%
    decorations.pathreplacing,%
    decorations.pathmorphing%
}

\tikzstyle arrowstyle=[black,scale=2]
\tikzstyle directed=[postaction={decorate,decoration={markings,
    mark=at position .65 with {\arrow[arrowstyle]{stealth}}}}]
\tikzstyle reverse directed=[postaction={decorate,decoration={markings,
    mark=at position .65 with {\arrowreversed[arrowstyle]{stealth};}}}]
\tikzstyle dir=[postaction={decorate,decoration={markings,
    mark=at position .98 with {\arrow[arrowstyle]{latex}}}}]
\tikzstyle rev dir=[postaction={decorate,decoration={markings,
    mark=at position .98 with {\arrowreversed[arrowstyle]{latex};}}}]

\usepackage{ctable}

%
%Redefining sections as problems
%
\makeatletter
\newenvironment{exercise}{\@startsection 
	{section}
	{1}
	{-.2em}
	{-3.5ex plus -1ex minus -.2ex}
    	{1.3ex plus .2ex}
    	{\pagebreak[3]%forces pagebreak when space is small; use \eject for better results
	\large\bf\noindent{Part 1.\hspace{-1.5ex} }
	}
	}
	%{\vspace{1ex}\begin{center} \rule{0.3\linewidth}{.3pt}\end{center}}
	%\begin{center}\large\bf \ldots\ldots\ldots\end{center}}
\makeatother

%
%Fancy-header package to modify header/page numbering 
%
\usepackage{fancyhdr}
\pagestyle{fancy}
%\addtolength{\headwidth}{\marginparsep} %these change header-rule width
%\addtolength{\headwidth}{\marginparwidth}
%\fancyheadoffset{30pt}
%\fancyfootoffset{30pt}
\fancyhead[LO,RE]{\small Oke}
\fancyhead[RO,LE]{\small Page \thepage} 
\fancyfoot[RO,LE]{\small } 
\fancyfoot[LO,RE]{\small \scshape CEE 616} 
\cfoot{} 
\renewcommand{\headrulewidth}{0.1pt} 
\renewcommand{\footrulewidth}{0.1pt}
%\setlength\voffset{-0.25in}
%\setlength\textheight{648pt}


\usepackage{paralist}

\newcommand{\osn}{\oldstylenums}
\newcommand{\lt}{\left}
\newcommand{\rt}{\right}
\newcommand{\pt}{\phantom}
\newcommand{\tf}{\therefore}
\newcommand{\?}{\stackrel{?}{=}}
\newcommand{\fr}{\frac}
\newcommand{\dfr}{\dfrac}
\newcommand{\ul}{\underline}
\newcommand{\tn}{\tabularnewline}
\newcommand{\nl}{\newline}
\newcommand\relph[1]{\mathrel{\phantom{#1}}}
\newcommand{\cm}{\checkmark}
\newcommand{\ol}{\overline}
\newcommand{\rd}{\color{red}}
\newcommand{\bl}{\color{blue}}
\newcommand{\pl}{\color{purple}}
\newcommand{\og}{\color{orange!90!black}}
\newcommand{\gr}{\color{green!40!black}}
\newcommand{\nin}{\noindent}
\newcommand{\la}{\lambda}
\renewcommand{\th}{\theta}
\newcommand*\circled[1]{\tikz[baseline=(char.base)]{
            \node[shape=circle,draw,thick,inner sep=1pt] (char) {\small #1};}}

\newcommand{\bc}{\begin{compactenum}[\quad--]}
\newcommand{\ec}{\end{compactenum}}

\newcommand{\n}{\\[2mm]}
%% GREEK LETTERS
\newcommand{\al}{\alpha}
\newcommand{\gam}{\gamma}
\newcommand{\eps}{\epsilon}
\newcommand{\sig}{\sigma}

\newcommand{\p}{\partial}
\newcommand{\pd}[2]{\frac{\partial{#1}}{\partial{#2}}}
\newcommand{\dpd}[2]{\dfrac{\partial{#1}}{\partial{#2}}}
\newcommand{\pdd}[2]{\frac{\partial^2{#1}}{\partial{#2}^2}}
\newcommand{\mr}{\mathbb{R}}
\newcommand{\xs}{x^{*}}
\newenvironment{solution}
{\medskip\par\quad\quad\begin{minipage}[c]{.8\textwidth}}{\medskip\end{minipage}}

\newcommand{\nmfr}[3]{\Phi\left(\frac{{#1} - {#2}}{#3}\right)}
 
%%%%%%%%%%%%%%%%%%%%%%%%%%%%%%%%%%%%%%%%%%%%%%%%%%%
%%%%%%%%%%%%%%%%%%%%%%%%%%%%%%%%%%%%%%%%%%%%%%%%%%%

\begin{document}

\lstset{language=C++,
                basicstyle=\tiny\ttfamily,
                keywordstyle=\color{blue}\ttfamily,
                stringstyle=\color{red}\ttfamily,
                commentstyle=\color{gray}\ttfamily,
                morecomment=[l][\color{gray}]{\#}
}


\thispagestyle{empty}


\nin{\LARGE Project Proposal}\hfill{\bf Prof. Oke}

\medskip\hrule\medskip

\nin {\small CEE 616: Probabilistic Machine Learning
\hfill\textit{ 10.16.2025}}

\nin{\it \small Due Nov 25, 2025 at 11:59PM}.\\
 
\subsubsection*{Objectives}
\label{sec:obj}
Through the course project, you will use the various methods you have learned and are currently learning in this class to analyze, predict or make inferences from a dataset of interest.

The purpose of this proposal is to afford you the opportunity to detail your project plans and receive feedback on the general direction of your  approach. 
Your proposal should be between \textbf{1 and 1.5 pages} in length. If you are unsure of where to start or would like a
dataset to work with, please let me know, as I can provide you with some options. You can work on your
project individually or in groups of \textbf{no larger than 3.}

\bigskip

\subsubsection*{General Instructions}
Your proposal should have the following sections.
\begin{itemize}
\item \textbf{Introduction:} Include a brief introductory/motivational paragraph of your project. {\it (Why are you interested in this topic? Why is it important?)}
\item \textbf{Objectives:} In a paragraph or list, outline the specific goals of your project \textit{(What process are you trying to understand? What are you trying to predict?)}
\item \textbf{Data:} Briefly describe the dataset you intend to use. \textit{(Specify the dimensions, i.e.\ number of observations and features). }
\item \textbf{Methods:} List or describe at least \textbf{2 modeling approaches} (supervised or unsupervised) from this course that you plan to apply in this project and their relevance to your data. \textit{(E.g.\ You could incorporate clustering and SVM in one framework. Or you could estimate an ANN and a GAM on a dataset and compare their performance.)}
  \item \textbf{Results:} Summarize your expected outcomes \textit{(from your proposed model(s)/modeling framework).}
\end{itemize}




% \newpage
% \begin{exercise}{\it Variable estimation}
% Choose one of the variable as your dependent variable.
% Using techniques learned in multivariate linear regression, study the dependence of the dependent variable with other relevant variables.
% Justify your explanatory variable selection as well as the fitness of your model (or models).
% Use also ridge regression or kernel density estimation.
% Use plots and tables to support your responses.
% \end{exercise}


% \begin{exercise}{\it Dimensionality reduction}
%   Apply Principal Components Analysis you learned in class.Clearly state your reasoning.
%   Indicate how well the projected data are representative of the original.
%   Describe your results, using tables and plots to show how the indicators contribute to each factor or component.
%   Justify why the factors/components make sense
  
%   % \begin{enumerate}[(a)]
%   % \item 
%   % \end{enumerate}
% \end{exercise}

% \begin{exercise}{\it Clustering}
%   Use any of the clustering approaches discussed in the lectures to assign the cities into characteristic groups.
%   This should be done based on computations in the reduced space obtained in the previous exercise.
%   Clearly show your results using tables and maps.
%   Justify why the clusters/groups you have obtained make sense.
%   Indicate the criteria for measuring the quality of your method of choice.
%   % \begin{enumerate}[(a)]
%   % \item 
%   % \end{enumerate}
% \end{exercise}
% \vfill

\subsubsection*{Submission Instructions}
\begin{enumerate}
\item Submit a PDF of your proposal on Moodle by \textbf{November 25} using the naming format:\\ \texttt{<LastName>-Proposal.pdf}.
\item If you choose to work in a group, ensure that all your last names are in the PDF filename. Each member of the group should  submit
  a copy of the same the PDF on  Moodle (same filename).  
\item I am happy to meet anytime before Nov 25 to discuss your ideas or answer questions. 
\end{enumerate}
%Slides for your presentation should also be included with your submission as a PDF. Clearly indicate the responsibilities assigned to each person in your group.
 


\end{document}

%%% Local Variables:
%%% mode: latex
%%% TeX-master: t
%%% End:
