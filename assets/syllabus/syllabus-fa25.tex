\documentclass[11pt,twoside]{article}
%\usepackage{etex}
\newcommand{\num}{6{} }

\raggedbottom
\usepackage{soul}
%\usepackage[tracking]{microtype}

%\usepackage[sc,osf]{mathpazo}   % With old-style figures and real smallcaps.
%\linespread{1.025}              % Palatino leads a little more leading

% Euler for math and numbers
%\usepackage[euler-digits,small]{eulervm}
%\AtBeginDocument{\renewcommand{\hbar}{\hslash}}

%geometry (sets margin) and other useful packages
\usepackage{geometry}
\geometry{top=1in, left=1in,right=1in,bottom=1in}
 \usepackage{graphicx,booktabs,calc}

%=== GRAPHICS PATH ===========
%\graphicspath{{./140408-Images/}}
% Marginpar width
%Marginpar width
\newcommand{\pts}[1]{\marginpar{ \small\hspace{0pt} \textit{[#1]} } } 
\setlength{\marginparwidth}{.5in}
%\reversemarginpar
%\setlength{\marginparsep}{.02in}

%% Fonts
% \usepackage{fourier}
% \usepackage[T1]{pbsi}

\usepackage{lmodern}
\usepackage[T1]{fontenc}

\usepackage{rotating}
%% Cite Title
% \usepackage[style=numeric,url=false,eprint=false,maxbibnames=99,maxcitenames=2]{biblatex}
% %\addbibresource{bib/references.bib}
% \addbibresource{../bib/references.bib}
% %%% Counters
\usepackage{chngcntr,mathtools}
\counterwithout{figure}{section}
\counterwithout{table}{section}

\numberwithin{equation}{section}

%% Captions
\usepackage{caption}
\captionsetup{
  labelsep=quad,
  justification=raggedright,
  labelfont=sc
}

%AMS-TeX packages
\usepackage{amssymb,amsmath,amsthm} 
\usepackage{bm}
\usepackage[mathscr]{eucal}
\usepackage{colortbl}
\usepackage{color}


\usepackage{epstopdf,subfigure,hyperref,enumerate,polynom,polynomial}
\usepackage{multirow,minitoc,fancybox,array,multicol}

\definecolor{slblue}{rgb}{0,.3,.62}
\hypersetup{
    colorlinks,%
    citecolor=blue,%
    filecolor=blue,%
    linkcolor=blue,
    urlcolor=slblue
}

%%%TIKZ
\usepackage{tikz}
\usepackage{pgfplots}
\usepackage{pgfplotstable}
\pgfplotsset{compat=newest}

\usetikzlibrary{arrows,shapes,positioning}
\usetikzlibrary{decorations.markings}
\usetikzlibrary{shadows,automata}
\usetikzlibrary{patterns}
%\usetikzlibrary{circuits.ee.IEC}
\usetikzlibrary{decorations.text}
% For Sagnac Picture
\usetikzlibrary{%
    decorations.pathreplacing,%
    decorations.pathmorphing%
}

%
%Redefining sections as problems
%
\makeatletter
\newenvironment{question}{\@startsection 
	{section}
	{1}
	{-.2em}
	{-3.5ex plus -1ex minus -.2ex}
    	{1.3ex plus .2ex}
    	{\pagebreak[3]%forces pagebreak when space is small; use \eject for better results
	\large\bf\noindent{Question }
	}
	}
	%{\vspace{1ex}\begin{center} \rule{0.3\linewidth}{.3pt}\end{center}}
	%\begin{center}\large\bf \ldots\ldots\ldots\end{center}}
\makeatother

%
%Fancy-header package to modify header/page numbering 
%
%\renewcommand{\chaptermark}[1]{ \markboth{#1}{} }
\renewcommand{\sectionmark}[1]{ \markright{#1}{} }

\usepackage{fancyhdr,etoolbox}
\pagestyle{fancy}
\newcommand{\headrulecolor}[1]{\patchcmd{\headrule}{\hrule}{\color{#1}\hrule}{}{}}
\newcommand{\footrulecolor}[1]{\patchcmd{\footrule}{\hrule}{\color{#1}\hrule}{}{}}
%\addtolength{\headwidth}{\marginparsep} %these change header-rule width
%\addtolength{\headwidth}{\marginparwidth}
%\fancyheadoffset{30pt}
%\fancyfootoffset{30pt}
\fancyhead[LO,RE]{\small\color{gray}  \it \nouppercase{\leftmark}}
\fancyhead[RO,LE]{\small\color{gray} \thepage} 
\fancyfoot[RO,LE]{\small\color{gray} CEE 260/MIE 273} 
\fancyfoot[LO,RE]{\small\color{gray} Prof.\ Oke} 
\cfoot{} 
\renewcommand{\headrulewidth}{0.1pt} 
\renewcommand{\footrulewidth}{0.1pt}
%\setlength\voffset{-0.25in}
%\setlength\textheight{648pt}
\footrulecolor{gray!50}
\headrulecolor{gray!50}
\usepackage{paralist}


%%% FORMAT PYTHON CODE
\usepackage{listings}
% Default fixed font does not support bold face
\DeclareFixedFont{\ttb}{T1}{txtt}{bx}{n}{8} % for bold
\DeclareFixedFont{\ttm}{T1}{txtt}{m}{n}{8}  % for normal

% Custom colors
\usepackage{color}
\definecolor{deepblue}{rgb}{0,0,0.5}
\definecolor{deepred}{rgb}{0.6,0,0}
\definecolor{deepgreen}{rgb}{0,0.5,0}

%\usepackage{listings}

% Python style for highlighting
\newcommand\pythonstyle{\lstset{
language=Python,
basicstyle=\footnotesize\ttm,
otherkeywords={self},             % Add keywords here
keywordstyle=\footnotesize\ttb\color{deepblue},
emph={MyClass,__init__},          % Custom highlighting
emphstyle=\footnotesize\ttb\color{deepred},    % Custom highlighting style
stringstyle=\color{deepgreen},
frame=tb,                         % Any extra options here
showstringspaces=false            % 
}}

% Python environment
\lstnewenvironment{python}[1][]
{
\pythonstyle
\lstset{#1}
}
{}

% Python for external files
\newcommand\pythonexternal[2][]{{
\pythonstyle
\lstinputlisting[#1]{#2}}}

% Python for inline
\newcommand\pythoninline[1]{{\pythonstyle\lstinline!#1!}}


\newcommand{\osn}{\oldstylenums}
\newcommand{\dg}{^{\circ}}
\newcommand{\lt}{\left}
\newcommand{\rt}{\right}
\newcommand{\pt}{\phantom}
\newcommand{\tf}{\therefore}
\newcommand{\?}{\stackrel{?}{=}}
\newcommand{\fr}{\frac}
\newcommand{\dfr}{\dfrac}
%\newcommand{\ul}{\underline}
\newcommand{\tn}{\tabularnewline}
\newcommand{\nl}{\newline}
\newcommand\relph[1]{\mathrel{\phantom{#1}}}
\newcommand{\cm}{\checkmark}
\newcommand{\ol}{\overline}
\newcommand{\rd}{\color{red}}
\newcommand{\bl}{\color{blue}}
\newcommand{\pl}{\color{purple}}
\newcommand{\og}{\color{orange!90!black}}
\newcommand{\gr}{\color{green!40!black}}
\newcommand{\nin}{\noindent}
\newcommand{\la}{\lambda}
\renewcommand{\th}{\theta}
\newcommand{\al}{\alpha}
\newcommand{\G}{\Gamma}
\newcommand*\circled[1]{\tikz[baseline=(char.base)]{
            \node[shape=circle,draw,thick,inner sep=1pt] (char) {\small #1};}}

\newcommand{\bc}{\begin{compactenum}[\quad--]}
\newcommand{\ec}{\end{compactenum}}

\newcommand{\p}{\partial}
\newcommand{\pd}[2]{\frac{\partial{#1}}{\partial{#2}}}
\newcommand{\dpd}[2]{\dfrac{\partial{#1}}{\partial{#2}}}
\newcommand{\pdd}[2]{\frac{\partial^2{#1}}{\partial{#2}^2}}

\newcommand{\zkr}{Z(k_{r})}
\newcommand{\zkb}{Z(k_{b})}
\newcommand{\zkl}{Z(k_{l})}
\newcommand{\ztot}{Z_{\mathrm{tot}}}
\newcommand{\alp}{\alpha_{\text{prec}}}
\newcommand{\alo}{\alpha_{\text{orig}}}
\newcommand{\als}{\alpha_{\text{succ}}}
\newcommand{\sep}{\mathrm{sec}_{\text{prec}}}
\newcommand{\seo}{\mathrm{sec}_{\text{orig}}}
\newcommand{\ses}{\mathrm{sec}_{\text{succ}}}
\newcommand{\dir}{\mathrm{dir}}
\newcommand{\ali}{\alpha_{i}}
\newcommand{\nw}{N\"{o}llenburg and Wolff}
\newcommand{\incg}{\includegraphics}
\newenvironment{solution}
{\medskip\par\quad\quad\begin{minipage}[c]{.8\textwidth}}{\medskip\end{minipage}}


%%%%%%%%%%%%%%%%%%%%%%%%%%%%%%%%%%%%%%%%%%%%%%%%%%%
%%%%%%%%%%%%%%%%%%%%%%%%%%%%%%%%%%%%%%%%%%%%%%%%%%%

\begin{document}

\title{CEE 260/MIE 273 \\
  Probability and Statistics in Engineering \\
Fall 2025  Course Syllabus
}
% \author{Jimi Oke}
\date{}
\maketitle

\thispagestyle{empty}

\tableofcontents
%\listoftables

\eject
\section{Personnel and Logistics}

\subsection{Venue and Class Time}
All classes will be held Engineering Lab II Room 119. %  Lectures will be available on livestream for situations where
% in-person class is not possible or feasible.  
They will also be recorded and available for download via Canvas (using Echo360).  You are expected to
attend all lectures, except by prior notification.  I will also upload slides in advance of each lecture.  Periodically,
I will also make typeset notes available to you, particularly for exam review sessions.  I encourage you to use the Canvas
Forum for discussions on homework, problem sets and other
academic/course matters.

\begin{quote}\small
  \textbf{Day \& Time:} Tuesdays 1:00PM -- 2:15PM\\
  \textbf{Physical Location:} ELab II Room 119\\
 % \textbf{Zoom:}  See Moodle page \url{https://umass.moonami.com/mod/lti/view.php?id=1308345}\\
 \textbf{Gradescope Entry Code:} J6W65J\\
\textbf{Project Presentations:} December 11 (10:30AM -- 12:30PM)
\end{quote}

\subsection{Instructor}
\begin{quote}\small
\textbf{Name:} Dr. Jimi Oke\footnote{Pronounced ``aw-KEH''; Pronouns: He/Him/His} \\
\textbf{Email:} jimi@umass.edu\\
\textbf{Office:} Marston 214D\\
\textbf{Office Hours:} Wednesdays and Thursdays (2:20 -- 3:20PM)
\end{quote}

\subsection{Teaching Assistant}
The Teaching Assistant will be responsible for coordinating the course and holding additional office hours for homework help.
\begin{quote}\small
   \textbf{Name:} Mohammed Abdalazeem\\
   \textbf{Email:} mamohammed@umass.edu\\
   \textbf{Tutorials:} Fridays (10:00AM -- 11:00AM; Location: TBD)
   \end{quote}

\subsection{Graders}
Two graders  will be responsible for evaluating and assigning grades to problem sets:
\begin{compactitem}\small
	\item Colin O'Brien
	\item Rillary Madruga Ferreira 
\end{compactitem} 

%\subsection{Supplemental Instruction Leader}
%TBA
% \textbf{Kevin Brooks} (kpbrooks@umass.edu)
% \begin{itemize}
% \item Facilitate two tutorial sessions weekly (time/link TBD)
% \item Provide partial homework/problem set assistance as feasible
% \end{itemize}
% \begin{quote}
%   \it You don't need a question to come to office hours. You can just come by to say hello.
%   \footnote{Greg Mankiw, Professor of Economics at Harvard, 2017. \url{https://news.harvard.edu/gazette/story/2017/12/professors-examine-the-realities-of-office-hours/}.}
% \end{quote}


\section{Course Information}
\subsection{Description}
The Probability and Statistics in  Engineering course is designed to introduce students to
the field of probability and statistics, and demonstrate its
importance and utility in the solution of problems specifically of interest to civil,
environmental, mechanical and industrial engineering.
Core areas covered by this course are
\begin{compactitem}
\item basic probability concepts
\item the role of uncertainty in engineering design
\item sampling and inference (hypothesis testing, confidence intervals, ANOVA)
\item probability distribution model fitting and testing
\item regression and correlation analyses
\end{compactitem}

\subsection{Objectives}
CEE 260/MIE 273 aims to introduce you to statistical methods in engineering and develop
your ability to analytically apply these methods in your practice of engineering.  

\subsection{Outcomes}
In this course, you will:
\begin{compactitem}
\item   Understand the fundamental concepts of probability, such as independence, expectation,
error propagation, and density functions.
\item Identify, apply and evaluate the proper probability model for different systems.
\item Use statistical methods to describe processes and make inferences about systems from
data.
\item Perform regression analyses, test hypotheses, and calculate confidence intervals for solving
engineering problems.
\item Develop and apply computational and numerical approaches to quantify uncertainty.
\item Gain proficiency with Python for statistical analysis.
\end{compactitem}

\subsection{Textbook}

{\bf Required Text:}
Diez, David, Cetinkaya-Rundel, Mine, and Barr, Christopher. OpenIntro Statistics (4th Ed.) 2019.
\url{https://leanpub.com/openintro-statistics}\\
%\footnote{Shown as ``DCB'' in the {\bf Reading} column in Section 4 (page 4).}

\noindent {\bf Supplementary Reading:}
William Navidi. Statistics for Engineers and Scientists. Fifth Edition, McGraw-Hill Education 2020. \\

\noindent Other resource: Kunin et al. Seeing Theory (\url{https://seeing-theory.brown.edu/index.html})
\subsection{Prerequisite}
MATH 132 (or equivalent).

\section{Policies}
My goal is to introduce you to the rudiments of probability and statistics within an engineering context.
I will use slides in the classroom, and annotate them electronically when possible.
These slides will be available to you prior to the lecture.
I will endeavor to foster an equitable and inclusive learning environment that will spark your curiosity and challenge you learn actively.
I strongly urge you to come to class prepared, having done the reading, ready to reflect on your homework or problem set and engage with new material.
I will ask frequent questions of you, and will also expect you to ask as many questions as possible.
Further specifics on class policies and values are as follows.

\subsection{Overview of Assessments}
You will be evaluated based on your in-class activities, problem sets, a midterm and a project.
See \autoref{tab:comp} for a breakdown of the grades.

\begin{table}[h!]
  \centering
  \caption{Course components and grade breakdown}
  \label{tab:comp}  
  \begin{tabular}{l c  r}\toprule
    \bf Assessment & \bf Number & \bf \% \\ \midrule
    In-Class Activities  & 23 & 23 \\
    Problem Sets [PS]    & 9  & 45 \\
    Midterm Exam         & 1 &   12 \\
    Regression Project   & 1 &   20 \\ \bottomrule
  \end{tabular}
\end{table}


Students are assessed individually, and there is no pre-determined grade spread in any subject.
Consistent with this, after drop date, students who remain in this class are not in jeopardy of seeing their grades change due to the change in class composition.
There will be no grading on a curve. 
Individual grades will be based on the scale shown in \autoref{tab:scale}.

\begin{table}[h!]
   \centering
 \caption{Grading scale}
 \label{tab:scale}
   \begin{tabular}{l l}\toprule
   Grade & Range (\%) \\\midrule
   A & 93--100\%   \\
   A$-$ & 90--92 \\
   B$+$ & 87--89 \\
   B &   83--86 \\
   B$-$ & 80--82 \\
   C$+$  & 77--79\\
   C   & 73--76 \\
   C$-$ & 70-72 \\
   D & 60-69 \\
   F &  $\le$ 60\\\bottomrule
 \end{tabular}
\end{table}

\subsection{In-class activities}
During each lecture, there will be an activity to be completed either individually or in a group. Activities will include short readings, interactive quizzes/polls and/or worksheets.
In order to participate fully, \textbf{please bring your laptop to all lectures}.


\subsection{Problem sets}
Problem sets (PS) will be assigned weekly and due on Tuesdays at 11:59pm submitted via Canvas.
Solutions will be posted after the due date.
Each PS will also include a guided Python problem (provided as a Jupyter Notebook) with clearly stated objectives.  
In each case, the steps will be laid out as clearly as possible.  You should
be able to complete these without further external guidance.  
These assignments will enable you master some key probabilistic and statistical functions in Python, along with practice in visualization. 
You will also practice applying a concept you learned in class while completing these assignments.

In each assignment, you will be required to submit a %\texttt{.m} (MATLAB), \texttt{.py} (Python) or 
\texttt{.ipynb} (Jupyter Notebook) with your responses.
Instructions will be clearly laid out on each assignment, so please follow them to a T.  This will ensure that your work
can be efficiently and fairly evaluated.

NOTE: {\rd \bf Late problem sets} \textbf{will not be graded}, excepting emergencies or illness (of which adequate proof must be provided) or {\bf  prior permission} for exigencies. 


\subsection{Exam}
The mid-term will be conducted as a  take-home open-book exam. You will typically have 2 hours to complete it.
It will be submitted as a PDF document.

\subsection{Project}
The project will be a structured assignment focused on linear regression. 
You will execute this in a group of 5 students.
More detailed instructions will be provided after the mid-term.
Deliverables will include a proposal, a report and a presentation.





%\appendix
%\eject





\section{Schedule}
The table below summarizes the schedule of topics, reading and assignments for this course.
Please study carefully.
Be sure you have an accessible and working installation of MATLAB or Python by the end of the first week of class.
Let me or the TA know promptly if you have any questions.
%Chapter 10 (Ang \& Tang) and the OpenIntro (DCB) chapters will be provided on Moodle.

\begin{table}[h!]\small
	\centering
	\begin{tabular}{p{1.8cm} p{1.2cm} p{7cm}  p{1.5cm} p{3cm}}\toprule
		\bf Date & \bf Module & \bf Topic &  \bf Reading & \bf Assignments \\\midrule
		\rowcolor{gray!30}    \multicolumn{5}{l}{\bf M1 Introduction} \\\midrule
		Tu, Sep 2 & 1a  & Data and Sampling   & 1 &  \\
		Th, Sep 4 & 1b  & Summarizing Data       & 2          & \\
		Tu, Sep 9  & 1c & Case Studies and Experiments      & 2  & PS1 due\\    
		\midrule
		\rowcolor{gray!30}    \multicolumn{5}{l}{\bf M2 Probability} \\\midrule
		Th, Sep 11  & 2a & Events \& Set Operations          & 3.1    &  \\ 
		Tu, Sep 16 &  2b & Theory of Probability           & 3.2         &   PS2 due  \\  
		Th, Sep 18 & 2d &Conditional Probability \& Bayes' Theorem                 &   3.3-4          &  \\ 
		\midrule
		\rowcolor{gray!30}    \multicolumn{5}{l}{\bf M3 Probability Distributions} \\\midrule
		Tu, Sep 23 & 3a & Introduction           & 3.5   &  PS3 due  \\   
		Th, Sep 25 & 3b & Normal Distribution    & 4.1  &   \\ 
		Tu, Sep 30 & 3c & Lognormal and Exponential Distributions   & TBD   & PS4 due \\ 
		Th, Oct 2 & 3d & Binomial Distribution & 4.3 &  \\ 
		Tu, Oct 7 & 3e & Poisson Distribution         &    4.5        & PS5 due \\ 
		Th, Oct 9 & 3f & Review/Applications         &          & \\ 
		\midrule
		Tu, Oct 14 &   & MIDTERM EXAM (Take Home: 2hrs)                &            &  \\
		\midrule
		\rowcolor{gray!30}    \multicolumn{5}{l}{\bf M4 Inference Foundations} \\\midrule
		Th, Oct 16 &  4a & Point Estimates and Sampling Variability                  &      5.1      &  \\
		Tu, Oct 21  & 4b & Confidence Intervals for a Proportion             & 5.2         & \\
		Th, Oct 23  & 4c & Hypothesis Testing for a Proportion      & 5.3          & {PS6 due}  \\ 
		\midrule
		\rowcolor{gray!30}    \multicolumn{5}{l}{\bf M5 Inference for Categorical Data} \\\midrule
		Tu, Oct 28  & 5a & Inference for Single Proportion                  & 6.1            &  \\  
		Th, Oct 30 & 5b & Difference of Two Proportions       & 6.2        & {PS7 due}   \\ 
		\rowcolor{gray!30}  Tu, Nov 4 & & NO CLASS (Election Day) && \\
		Th, Nov 6  & 5c & Goodness of Fit Testing (Chi-square)    & 6.3-4      &  \\ \midrule
		\rowcolor{gray!30}    \multicolumn{5}{l}{\bf M6 Inference for Numerical Data} \\\midrule           
		Tu, Nov 11  & 6a & One-sample means   & 6.1--6.4   &  {PS7 due} \\
		Th, Nov 13   & 6b & Inference with Two Samples       & 7.2-3   &  \\  
		Tu, Nov 18   & 6c & Power and ANOVA      & 7.4-5 &  {PS8 due} \\ 
		\midrule
		\rowcolor{gray!30} \multicolumn{5}{l}{\bf M7 Linear Regression} \\\midrule
		Th, Nov 20 & 7a & Correlation and Least Squares              & 8.1-2       &   \\		
		Tu, Nov 25  & 7b & Inference for Regression              & 8.4       &  {PS9 due} \\
		\rowcolor{gray!30} Th, Nov 27 &  & THANKSGIVING RECESS & &\\
		Tu, Dec 2  & 7c & Multiple Regression  & 9.1-4  &   \\ 
		Th, Dec 4  &  & Review/Project Work     &  &  \\
		\rowcolor{gray!30}Tu, Dec 9  &  & NO CLASS     &   &  \\		
		\midrule
		Th, Dec 11  &       & PROJECT PRESENTATIONS && \\  \bottomrule
	\end{tabular}
	%\caption{} %Proposed syllabus (full version)}
\label{tab:compsyl}
\end{table}

\eject
\section{Values}
\subsection{Academic Honesty Policy Statement}
Since the integrity of the academic enterprise of any institution of higher education requires
honesty in scholarship and research, academic honesty is required of all students at the
University of Massachusetts Amherst. Academic dishonesty including but not limited to
cheating, fabrication, plagiarism, and facilitating dishonesty, is prohibited in all programs of
the University. Appropriate sanctions may be imposed on any student who has committed
an act of academic dishonesty. Instructors should take reasonable steps to address academic
misconduct. Any person who has reason to believe that a student has committed academic
dishonesty should bring such information to the attention of the appropriate course
instructor as soon as possible. Instances of academic dishonesty not related to a specific
course should be brought to the attention of the appropriate department Head or Chair. The
procedures outlined below are intended to provide an efficient and orderly process by which
action may be taken if it appears that academic dishonesty has occurred and by which
students may appeal such actions. Since student are expected to be familiar with this policy
and the commonly accepted standards of academic integrity, ignorance of such standards is
not normally sufficient evidence of lack of intent.
For more information about what constitutes academic dishonesty, please see the Dean of
Students' website: \url{http://umass.edu/dean_students/honesty/}

\subsection{Disability Statement}
The University of Massachusetts Amherst is committed to making reasonable, effective and
appropriate accommodations to meet the needs of students with disabilities and help create a
barrier-free campus. If you are in need of accommodation for a documented disability,
register with Disability Services to have an accommodation letter sent to your faculty. It is
your responsibility to initiate these services and to communicate with faculty ahead of time
to manage accommodations in a timely manner. For more information, consult the
Disability Services website at \url{http://www.umass.edu/disability/}.


\subsection{Title IX}

In accordance with Title IX of the Education Amendments of 1972 that prohibits gender-based discrimination in educational settings that receive federal funds, the University of Massachusetts Amherst is committed to providing a safe learning environment for all students, free from all forms of discrimination, including sexual assault, sexual harassment, domestic violence, dating violence, stalking, and retaliation. This includes interactions in person or online through digital platforms and social media. Title IX also protects against discrimination on the basis of pregnancy, childbirth, false pregnancy, miscarriage, abortion, or related conditions, including recovery. There are resources here on campus to support you. A summary of the available Title IX resources (confidential and non-confidential) can be found at the following link: 
\url{https://www.umass.edu/titleix/resources}. You do not need to make a formal report to access them. If you need immediate support, you are not alone. Free and confidential support is available 24 hours a day/7 days a week/365 days a year at the SASA Hotline 413-545-0800.



\subsection{Policy Against Discrimination, Harassment, and Related Interpersonal Violence}

In synergy with the aspirations of our department and the college, I strive to ensure that my classroom is a place where all students can not only succeed, but also thrive. Federal and state laws as well as University policies provide several protections to support these efforts. 

The University of Massachusetts Amherst, through this Policy Against Discrimination, Harassment, and Related Interpersonal Violence prohibits unlawful discrimination, harassment, and retaliation on the basis of race, color, religion, caste, creed, sex, sex stereotypes, sex characteristics, sexual orientation, gender identity and expression, pregnancy and pregnancy-related condition(s), age, marital status, national origin, mental or physical disability, political belief or affiliation, veteran status , genetic information, natural and protective hairstyle, and any other legally protected class of individuals protected from discrimination under state or federal law in any aspect of the access to, admission, employment, or treatment in the University's educational program and activity. The University affirms its commitment to provide a welcoming and respectful work and educational environment, in which all individuals within the University community may benefit from each other's experiences and foster mutual respect and appreciation of divergent views. Any member of the campus community, guest, or other person who acts to deny or limit the access to educational, employment, residential, and/or social programs or activities, benefits, and/or opportunities of any (other) member of the campus community, guest, or visitor on the basis of their actual or perceived membership in classes protected by this Policy will be in violation of this Policy.



\section{Wellness and Success}
You are not alone at UMass---many people care about your well-being and many resources are available to help you thrive and succeed.
During this time, you may be experiencing pressures related to as health, money, family, and academic concerns or stress and trauma from societal inequities and violence.
Coursework is challenging and classes are not the only demand in your life. 
You have resilience and are already using effective strategies to help you achieve your educational goals.
Take stock of these and consider what new steps or resources could be helpful.
Getting enough sleep, exercising, eating well, and connecting with others are all antidotes to stress.
If you are struggling academically, reach out to your instructors and advisors prior to deadlines and before the demands of exams, papers, and projects reach their peak. 
Students experiencing challenges including stress, anxiety, difficulty concentrating, loneliness, and trauma, or who feel down or alienated, can find it helpful to connect with one or more of the many supportive resources on campus that stand ready to assist you. You matter at UMass. 

\subsection{Academic Advice and Support}
\begin{itemize}
\item Academic Dean: \url{https://www.umass.edu/registrar/students/list-academic-deans}
\item  Academic Advisor: \url{https://www.umass.edu/gateway/academics/academic-advising}
\item  Writing Center: \url{https://www.umass.edu/writingcenter/}
\item  Learning Resource Center: \url{https://www.umass.edu/lrc/}
\end{itemize}

\subsection{Single-Stop Resources}
\begin{itemize}
\item  Referrals for personal, financial, or life challenges that interfere with college success and well-being: \url{http://www.umass.edu/studentlife/single-stop}
 %  \item  Expanded resources for support during COVID-19: \url{https://www.umass.edu/dean_students/support-resources}

   \end{itemize}


   \subsection{Communities of Support}
\begin{itemize}
\item  Residential Life Support for on campus students; help addressing roommate disputes and residence hall quality of life: \url{https://www.umass.edu/living/}
   \item  Off Campus Student Life (OCSL): \url{https://www.umass.edu/offcampuslife/}; (413) 577-1005\\
Community connections and programs for students living off-campus
   \item  International Programs Office (IPO): \url{https://www.umass.edu/ipo/}; (413) 545-2710\\
Networking and assistance for international students and scholars at UMass and UMass students studying abroad
   \item  Center for Multicultural Advancement and Student Success (CMASS): \url{https://www.umass.edu/cmass/}; (413) 545-2517\\
Mentoring, workshops, advocacy, scholarship and internship opportunities, graduate school preparation and career development
   \item  Stonewall Center: \url{https://www.umass.edu/stonewall/}; (413) 545-4824\\
Programming, advocacy, and support for LGBTQIA+ students and allies
   \item  Student Parent Programs: \url{https://www.umass.edu/ofr/}; (413) 545-0865\\
Support for students with children
   \item  Student Veteran Resource Center:\\ \url{https://www.umass.edu/veterans/student-veteran-resource-center-svrc}; (413) 545-0939\\
A welcoming place for veterans, active military members, and their families to study, network, learn, seek support, and get help with benefits
   \item  Center for Women and Community: \url{https://www.umass.edu/cwc};  (413) 545-0883\\
Information and referrals, community education, general counseling, and empowerment-based support groups for survivors of all genders
   \item  Men and Masculinities Center: \url{https://www.umass.edu/masculinities/}; (413) 577-4636
Supports male student success and the development of healthy masculinities
   \item  Office of Religious and Spiritual Life: \url{https://www.umass.edu/orsl/}; (413) 545-9642\\
Educational programs, advocacy, dialogue, interfaith programs and service
   \item  Center for Health Promotion: \url{https://www.umass.edu/studentlife/health-safety/chp}; (413) 577-5181\\
     Peer wellness coaching, alcohol screening and brief intervention, support for students in recovery
   \end{itemize}
\subsection{Offices that Can Help}
\begin{itemize}
\item  Center for Counseling and Psychological Health: \url{https://www.umass.edu/counseling/} (413) 545-2337.  After hours: (877) 831-7421\\
24/7 emergency crisis intervention, support groups and workshops, online therapy and resources, brief psychotherapy and referrals
   \item  Dean of Students Office: \url{https://www.umass.edu/dean_students/} (413) 545-2684\\
Advice and support for managing challenging or crisis related matters
   \item  UMass Police Department: \url{https://www.umass.edu/umpd/} (413) 545-2121.  Emergency: (413) 545-3111 or 911\\
Immediate emergency response, anonymous tip reporting, theft prevention, community safety, and self-defense programs and training
   \item  University Health Services: \url{https://www.umass.edu/uhs/} (413) 577-5000\\
24/7 medical advice and triage, walk-in clinic, nutritional counseling, sports medicine, and more
   \item  Disability Services: \url{https://www.umass.edu/disability/} (413) 545-0892\\
Help registering and accommodating students with all types of disabilities
   \item  Student Legal Services Office: \url{https://www.umass.edu/slso/} (413) 545-1995\\
Confidential legal counseling and advice for all fee-paying students with any legal matter
   \item  Psychological Services Center: \url{https://www.umass.edu/psc/} (413) 545-0041\\
Individual, couples and group therapy and assessment services
   \item  Ombuds Office: \url{https://www.umass.edu/ombuds/} (413) 545-0867\\
Facilitation and informal mediation; resolution of grade disputes
   \item  Equal Opportunity Office: \url{https://www.umass.edu/equalopportunity/} (413) 545-3464\\
Upholds university’s commitment to access and opportunity for all
\end{itemize}


\end{document}
%%% Local Variables:
%%% mode: latex
%%% TeX-master: t
%%% End:
